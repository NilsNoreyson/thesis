\documentclass[11pt]{article}

    \usepackage[breakable]{tcolorbox}
    \usepackage{parskip} % Stop auto-indenting (to mimic markdown behaviour)
    
    \usepackage{iftex}
    \ifPDFTeX
    	\usepackage[T1]{fontenc}
    	\usepackage{mathpazo}
    \else
    	\usepackage{fontspec}
    \fi

    % Basic figure setup, for now with no caption control since it's done
    % automatically by Pandoc (which extracts ![](path) syntax from Markdown).
    \usepackage{graphicx}
    % Maintain compatibility with old templates. Remove in nbconvert 6.0
    \let\Oldincludegraphics\includegraphics
    % Ensure that by default, figures have no caption (until we provide a
    % proper Figure object with a Caption API and a way to capture that
    % in the conversion process - todo).
    \usepackage{caption}
    \DeclareCaptionFormat{nocaption}{}
    \captionsetup{format=nocaption,aboveskip=0pt,belowskip=0pt}

    \usepackage[Export]{adjustbox} % Used to constrain images to a maximum size
    \adjustboxset{max size={0.9\linewidth}{0.9\paperheight}}
    \usepackage{float}
    \floatplacement{figure}{H} % forces figures to be placed at the correct location
    \usepackage{xcolor} % Allow colors to be defined
    \usepackage{enumerate} % Needed for markdown enumerations to work
    \usepackage{geometry} % Used to adjust the document margins
    \usepackage{amsmath} % Equations
    \usepackage{amssymb} % Equations
    \usepackage{textcomp} % defines textquotesingle
    % Hack from http://tex.stackexchange.com/a/47451/13684:
    \AtBeginDocument{%
        \def\PYZsq{\textquotesingle}% Upright quotes in Pygmentized code
    }
    \usepackage{upquote} % Upright quotes for verbatim code
    \usepackage{eurosym} % defines \euro
    \usepackage[mathletters]{ucs} % Extended unicode (utf-8) support
    \usepackage{fancyvrb} % verbatim replacement that allows latex
    \usepackage{grffile} % extends the file name processing of package graphics 
                         % to support a larger range
    \makeatletter % fix for grffile with XeLaTeX
    \def\Gread@@xetex#1{%
      \IfFileExists{"\Gin@base".bb}%
      {\Gread@eps{\Gin@base.bb}}%
      {\Gread@@xetex@aux#1}%
    }
    \makeatother

    % The hyperref package gives us a pdf with properly built
    % internal navigation ('pdf bookmarks' for the table of contents,
    % internal cross-reference links, web links for URLs, etc.)
    \usepackage{hyperref}
    % The default LaTeX title has an obnoxious amount of whitespace. By default,
    % titling removes some of it. It also provides customization options.
    \usepackage{titling}
    \usepackage{longtable} % longtable support required by pandoc >1.10
    \usepackage{booktabs}  % table support for pandoc > 1.12.2
    \usepackage[inline]{enumitem} % IRkernel/repr support (it uses the enumerate* environment)
    \usepackage[normalem]{ulem} % ulem is needed to support strikethroughs (\sout)
                                % normalem makes italics be italics, not underlines
    \usepackage{mathrsfs}
    

    
    % Colors for the hyperref package
    \definecolor{urlcolor}{rgb}{0,.145,.698}
    \definecolor{linkcolor}{rgb}{.71,0.21,0.01}
    \definecolor{citecolor}{rgb}{.12,.54,.11}

    % ANSI colors
    \definecolor{ansi-black}{HTML}{3E424D}
    \definecolor{ansi-black-intense}{HTML}{282C36}
    \definecolor{ansi-red}{HTML}{E75C58}
    \definecolor{ansi-red-intense}{HTML}{B22B31}
    \definecolor{ansi-green}{HTML}{00A250}
    \definecolor{ansi-green-intense}{HTML}{007427}
    \definecolor{ansi-yellow}{HTML}{DDB62B}
    \definecolor{ansi-yellow-intense}{HTML}{B27D12}
    \definecolor{ansi-blue}{HTML}{208FFB}
    \definecolor{ansi-blue-intense}{HTML}{0065CA}
    \definecolor{ansi-magenta}{HTML}{D160C4}
    \definecolor{ansi-magenta-intense}{HTML}{A03196}
    \definecolor{ansi-cyan}{HTML}{60C6C8}
    \definecolor{ansi-cyan-intense}{HTML}{258F8F}
    \definecolor{ansi-white}{HTML}{C5C1B4}
    \definecolor{ansi-white-intense}{HTML}{A1A6B2}
    \definecolor{ansi-default-inverse-fg}{HTML}{FFFFFF}
    \definecolor{ansi-default-inverse-bg}{HTML}{000000}

    % commands and environments needed by pandoc snippets
    % extracted from the output of `pandoc -s`
    \providecommand{\tightlist}{%
      \setlength{\itemsep}{0pt}\setlength{\parskip}{0pt}}
    \DefineVerbatimEnvironment{Highlighting}{Verbatim}{commandchars=\\\{\}}
    % Add ',fontsize=\small' for more characters per line
    \newenvironment{Shaded}{}{}
    \newcommand{\KeywordTok}[1]{\textcolor[rgb]{0.00,0.44,0.13}{\textbf{{#1}}}}
    \newcommand{\DataTypeTok}[1]{\textcolor[rgb]{0.56,0.13,0.00}{{#1}}}
    \newcommand{\DecValTok}[1]{\textcolor[rgb]{0.25,0.63,0.44}{{#1}}}
    \newcommand{\BaseNTok}[1]{\textcolor[rgb]{0.25,0.63,0.44}{{#1}}}
    \newcommand{\FloatTok}[1]{\textcolor[rgb]{0.25,0.63,0.44}{{#1}}}
    \newcommand{\CharTok}[1]{\textcolor[rgb]{0.25,0.44,0.63}{{#1}}}
    \newcommand{\StringTok}[1]{\textcolor[rgb]{0.25,0.44,0.63}{{#1}}}
    \newcommand{\CommentTok}[1]{\textcolor[rgb]{0.38,0.63,0.69}{\textit{{#1}}}}
    \newcommand{\OtherTok}[1]{\textcolor[rgb]{0.00,0.44,0.13}{{#1}}}
    \newcommand{\AlertTok}[1]{\textcolor[rgb]{1.00,0.00,0.00}{\textbf{{#1}}}}
    \newcommand{\FunctionTok}[1]{\textcolor[rgb]{0.02,0.16,0.49}{{#1}}}
    \newcommand{\RegionMarkerTok}[1]{{#1}}
    \newcommand{\ErrorTok}[1]{\textcolor[rgb]{1.00,0.00,0.00}{\textbf{{#1}}}}
    \newcommand{\NormalTok}[1]{{#1}}
    
    % Additional commands for more recent versions of Pandoc
    \newcommand{\ConstantTok}[1]{\textcolor[rgb]{0.53,0.00,0.00}{{#1}}}
    \newcommand{\SpecialCharTok}[1]{\textcolor[rgb]{0.25,0.44,0.63}{{#1}}}
    \newcommand{\VerbatimStringTok}[1]{\textcolor[rgb]{0.25,0.44,0.63}{{#1}}}
    \newcommand{\SpecialStringTok}[1]{\textcolor[rgb]{0.73,0.40,0.53}{{#1}}}
    \newcommand{\ImportTok}[1]{{#1}}
    \newcommand{\DocumentationTok}[1]{\textcolor[rgb]{0.73,0.13,0.13}{\textit{{#1}}}}
    \newcommand{\AnnotationTok}[1]{\textcolor[rgb]{0.38,0.63,0.69}{\textbf{\textit{{#1}}}}}
    \newcommand{\CommentVarTok}[1]{\textcolor[rgb]{0.38,0.63,0.69}{\textbf{\textit{{#1}}}}}
    \newcommand{\VariableTok}[1]{\textcolor[rgb]{0.10,0.09,0.49}{{#1}}}
    \newcommand{\ControlFlowTok}[1]{\textcolor[rgb]{0.00,0.44,0.13}{\textbf{{#1}}}}
    \newcommand{\OperatorTok}[1]{\textcolor[rgb]{0.40,0.40,0.40}{{#1}}}
    \newcommand{\BuiltInTok}[1]{{#1}}
    \newcommand{\ExtensionTok}[1]{{#1}}
    \newcommand{\PreprocessorTok}[1]{\textcolor[rgb]{0.74,0.48,0.00}{{#1}}}
    \newcommand{\AttributeTok}[1]{\textcolor[rgb]{0.49,0.56,0.16}{{#1}}}
    \newcommand{\InformationTok}[1]{\textcolor[rgb]{0.38,0.63,0.69}{\textbf{\textit{{#1}}}}}
    \newcommand{\WarningTok}[1]{\textcolor[rgb]{0.38,0.63,0.69}{\textbf{\textit{{#1}}}}}
    
    
    % Define a nice break command that doesn't care if a line doesn't already
    % exist.
    \def\br{\hspace*{\fill} \\* }
    % Math Jax compatibility definitions
    \def\gt{>}
    \def\lt{<}
    \let\Oldtex\TeX
    \let\Oldlatex\LaTeX
    \renewcommand{\TeX}{\textrm{\Oldtex}}
    \renewcommand{\LaTeX}{\textrm{\Oldlatex}}
    % Document parameters
    % Document title
    \title{Numerical calculations of SMOx based gas sensors with Python}
    
    
    
    
    
% Pygments definitions
\makeatletter
\def\PY@reset{\let\PY@it=\relax \let\PY@bf=\relax%
    \let\PY@ul=\relax \let\PY@tc=\relax%
    \let\PY@bc=\relax \let\PY@ff=\relax}
\def\PY@tok#1{\csname PY@tok@#1\endcsname}
\def\PY@toks#1+{\ifx\relax#1\empty\else%
    \PY@tok{#1}\expandafter\PY@toks\fi}
\def\PY@do#1{\PY@bc{\PY@tc{\PY@ul{%
    \PY@it{\PY@bf{\PY@ff{#1}}}}}}}
\def\PY#1#2{\PY@reset\PY@toks#1+\relax+\PY@do{#2}}

\expandafter\def\csname PY@tok@w\endcsname{\def\PY@tc##1{\textcolor[rgb]{0.73,0.73,0.73}{##1}}}
\expandafter\def\csname PY@tok@c\endcsname{\let\PY@it=\textit\def\PY@tc##1{\textcolor[rgb]{0.25,0.50,0.50}{##1}}}
\expandafter\def\csname PY@tok@cp\endcsname{\def\PY@tc##1{\textcolor[rgb]{0.74,0.48,0.00}{##1}}}
\expandafter\def\csname PY@tok@k\endcsname{\let\PY@bf=\textbf\def\PY@tc##1{\textcolor[rgb]{0.00,0.50,0.00}{##1}}}
\expandafter\def\csname PY@tok@kp\endcsname{\def\PY@tc##1{\textcolor[rgb]{0.00,0.50,0.00}{##1}}}
\expandafter\def\csname PY@tok@kt\endcsname{\def\PY@tc##1{\textcolor[rgb]{0.69,0.00,0.25}{##1}}}
\expandafter\def\csname PY@tok@o\endcsname{\def\PY@tc##1{\textcolor[rgb]{0.40,0.40,0.40}{##1}}}
\expandafter\def\csname PY@tok@ow\endcsname{\let\PY@bf=\textbf\def\PY@tc##1{\textcolor[rgb]{0.67,0.13,1.00}{##1}}}
\expandafter\def\csname PY@tok@nb\endcsname{\def\PY@tc##1{\textcolor[rgb]{0.00,0.50,0.00}{##1}}}
\expandafter\def\csname PY@tok@nf\endcsname{\def\PY@tc##1{\textcolor[rgb]{0.00,0.00,1.00}{##1}}}
\expandafter\def\csname PY@tok@nc\endcsname{\let\PY@bf=\textbf\def\PY@tc##1{\textcolor[rgb]{0.00,0.00,1.00}{##1}}}
\expandafter\def\csname PY@tok@nn\endcsname{\let\PY@bf=\textbf\def\PY@tc##1{\textcolor[rgb]{0.00,0.00,1.00}{##1}}}
\expandafter\def\csname PY@tok@ne\endcsname{\let\PY@bf=\textbf\def\PY@tc##1{\textcolor[rgb]{0.82,0.25,0.23}{##1}}}
\expandafter\def\csname PY@tok@nv\endcsname{\def\PY@tc##1{\textcolor[rgb]{0.10,0.09,0.49}{##1}}}
\expandafter\def\csname PY@tok@no\endcsname{\def\PY@tc##1{\textcolor[rgb]{0.53,0.00,0.00}{##1}}}
\expandafter\def\csname PY@tok@nl\endcsname{\def\PY@tc##1{\textcolor[rgb]{0.63,0.63,0.00}{##1}}}
\expandafter\def\csname PY@tok@ni\endcsname{\let\PY@bf=\textbf\def\PY@tc##1{\textcolor[rgb]{0.60,0.60,0.60}{##1}}}
\expandafter\def\csname PY@tok@na\endcsname{\def\PY@tc##1{\textcolor[rgb]{0.49,0.56,0.16}{##1}}}
\expandafter\def\csname PY@tok@nt\endcsname{\let\PY@bf=\textbf\def\PY@tc##1{\textcolor[rgb]{0.00,0.50,0.00}{##1}}}
\expandafter\def\csname PY@tok@nd\endcsname{\def\PY@tc##1{\textcolor[rgb]{0.67,0.13,1.00}{##1}}}
\expandafter\def\csname PY@tok@s\endcsname{\def\PY@tc##1{\textcolor[rgb]{0.73,0.13,0.13}{##1}}}
\expandafter\def\csname PY@tok@sd\endcsname{\let\PY@it=\textit\def\PY@tc##1{\textcolor[rgb]{0.73,0.13,0.13}{##1}}}
\expandafter\def\csname PY@tok@si\endcsname{\let\PY@bf=\textbf\def\PY@tc##1{\textcolor[rgb]{0.73,0.40,0.53}{##1}}}
\expandafter\def\csname PY@tok@se\endcsname{\let\PY@bf=\textbf\def\PY@tc##1{\textcolor[rgb]{0.73,0.40,0.13}{##1}}}
\expandafter\def\csname PY@tok@sr\endcsname{\def\PY@tc##1{\textcolor[rgb]{0.73,0.40,0.53}{##1}}}
\expandafter\def\csname PY@tok@ss\endcsname{\def\PY@tc##1{\textcolor[rgb]{0.10,0.09,0.49}{##1}}}
\expandafter\def\csname PY@tok@sx\endcsname{\def\PY@tc##1{\textcolor[rgb]{0.00,0.50,0.00}{##1}}}
\expandafter\def\csname PY@tok@m\endcsname{\def\PY@tc##1{\textcolor[rgb]{0.40,0.40,0.40}{##1}}}
\expandafter\def\csname PY@tok@gh\endcsname{\let\PY@bf=\textbf\def\PY@tc##1{\textcolor[rgb]{0.00,0.00,0.50}{##1}}}
\expandafter\def\csname PY@tok@gu\endcsname{\let\PY@bf=\textbf\def\PY@tc##1{\textcolor[rgb]{0.50,0.00,0.50}{##1}}}
\expandafter\def\csname PY@tok@gd\endcsname{\def\PY@tc##1{\textcolor[rgb]{0.63,0.00,0.00}{##1}}}
\expandafter\def\csname PY@tok@gi\endcsname{\def\PY@tc##1{\textcolor[rgb]{0.00,0.63,0.00}{##1}}}
\expandafter\def\csname PY@tok@gr\endcsname{\def\PY@tc##1{\textcolor[rgb]{1.00,0.00,0.00}{##1}}}
\expandafter\def\csname PY@tok@ge\endcsname{\let\PY@it=\textit}
\expandafter\def\csname PY@tok@gs\endcsname{\let\PY@bf=\textbf}
\expandafter\def\csname PY@tok@gp\endcsname{\let\PY@bf=\textbf\def\PY@tc##1{\textcolor[rgb]{0.00,0.00,0.50}{##1}}}
\expandafter\def\csname PY@tok@go\endcsname{\def\PY@tc##1{\textcolor[rgb]{0.53,0.53,0.53}{##1}}}
\expandafter\def\csname PY@tok@gt\endcsname{\def\PY@tc##1{\textcolor[rgb]{0.00,0.27,0.87}{##1}}}
\expandafter\def\csname PY@tok@err\endcsname{\def\PY@bc##1{\setlength{\fboxsep}{0pt}\fcolorbox[rgb]{1.00,0.00,0.00}{1,1,1}{\strut ##1}}}
\expandafter\def\csname PY@tok@kc\endcsname{\let\PY@bf=\textbf\def\PY@tc##1{\textcolor[rgb]{0.00,0.50,0.00}{##1}}}
\expandafter\def\csname PY@tok@kd\endcsname{\let\PY@bf=\textbf\def\PY@tc##1{\textcolor[rgb]{0.00,0.50,0.00}{##1}}}
\expandafter\def\csname PY@tok@kn\endcsname{\let\PY@bf=\textbf\def\PY@tc##1{\textcolor[rgb]{0.00,0.50,0.00}{##1}}}
\expandafter\def\csname PY@tok@kr\endcsname{\let\PY@bf=\textbf\def\PY@tc##1{\textcolor[rgb]{0.00,0.50,0.00}{##1}}}
\expandafter\def\csname PY@tok@bp\endcsname{\def\PY@tc##1{\textcolor[rgb]{0.00,0.50,0.00}{##1}}}
\expandafter\def\csname PY@tok@fm\endcsname{\def\PY@tc##1{\textcolor[rgb]{0.00,0.00,1.00}{##1}}}
\expandafter\def\csname PY@tok@vc\endcsname{\def\PY@tc##1{\textcolor[rgb]{0.10,0.09,0.49}{##1}}}
\expandafter\def\csname PY@tok@vg\endcsname{\def\PY@tc##1{\textcolor[rgb]{0.10,0.09,0.49}{##1}}}
\expandafter\def\csname PY@tok@vi\endcsname{\def\PY@tc##1{\textcolor[rgb]{0.10,0.09,0.49}{##1}}}
\expandafter\def\csname PY@tok@vm\endcsname{\def\PY@tc##1{\textcolor[rgb]{0.10,0.09,0.49}{##1}}}
\expandafter\def\csname PY@tok@sa\endcsname{\def\PY@tc##1{\textcolor[rgb]{0.73,0.13,0.13}{##1}}}
\expandafter\def\csname PY@tok@sb\endcsname{\def\PY@tc##1{\textcolor[rgb]{0.73,0.13,0.13}{##1}}}
\expandafter\def\csname PY@tok@sc\endcsname{\def\PY@tc##1{\textcolor[rgb]{0.73,0.13,0.13}{##1}}}
\expandafter\def\csname PY@tok@dl\endcsname{\def\PY@tc##1{\textcolor[rgb]{0.73,0.13,0.13}{##1}}}
\expandafter\def\csname PY@tok@s2\endcsname{\def\PY@tc##1{\textcolor[rgb]{0.73,0.13,0.13}{##1}}}
\expandafter\def\csname PY@tok@sh\endcsname{\def\PY@tc##1{\textcolor[rgb]{0.73,0.13,0.13}{##1}}}
\expandafter\def\csname PY@tok@s1\endcsname{\def\PY@tc##1{\textcolor[rgb]{0.73,0.13,0.13}{##1}}}
\expandafter\def\csname PY@tok@mb\endcsname{\def\PY@tc##1{\textcolor[rgb]{0.40,0.40,0.40}{##1}}}
\expandafter\def\csname PY@tok@mf\endcsname{\def\PY@tc##1{\textcolor[rgb]{0.40,0.40,0.40}{##1}}}
\expandafter\def\csname PY@tok@mh\endcsname{\def\PY@tc##1{\textcolor[rgb]{0.40,0.40,0.40}{##1}}}
\expandafter\def\csname PY@tok@mi\endcsname{\def\PY@tc##1{\textcolor[rgb]{0.40,0.40,0.40}{##1}}}
\expandafter\def\csname PY@tok@il\endcsname{\def\PY@tc##1{\textcolor[rgb]{0.40,0.40,0.40}{##1}}}
\expandafter\def\csname PY@tok@mo\endcsname{\def\PY@tc##1{\textcolor[rgb]{0.40,0.40,0.40}{##1}}}
\expandafter\def\csname PY@tok@ch\endcsname{\let\PY@it=\textit\def\PY@tc##1{\textcolor[rgb]{0.25,0.50,0.50}{##1}}}
\expandafter\def\csname PY@tok@cm\endcsname{\let\PY@it=\textit\def\PY@tc##1{\textcolor[rgb]{0.25,0.50,0.50}{##1}}}
\expandafter\def\csname PY@tok@cpf\endcsname{\let\PY@it=\textit\def\PY@tc##1{\textcolor[rgb]{0.25,0.50,0.50}{##1}}}
\expandafter\def\csname PY@tok@c1\endcsname{\let\PY@it=\textit\def\PY@tc##1{\textcolor[rgb]{0.25,0.50,0.50}{##1}}}
\expandafter\def\csname PY@tok@cs\endcsname{\let\PY@it=\textit\def\PY@tc##1{\textcolor[rgb]{0.25,0.50,0.50}{##1}}}

\def\PYZbs{\char`\\}
\def\PYZus{\char`\_}
\def\PYZob{\char`\{}
\def\PYZcb{\char`\}}
\def\PYZca{\char`\^}
\def\PYZam{\char`\&}
\def\PYZlt{\char`\<}
\def\PYZgt{\char`\>}
\def\PYZsh{\char`\#}
\def\PYZpc{\char`\%}
\def\PYZdl{\char`\$}
\def\PYZhy{\char`\-}
\def\PYZsq{\char`\'}
\def\PYZdq{\char`\"}
\def\PYZti{\char`\~}
% for compatibility with earlier versions
\def\PYZat{@}
\def\PYZlb{[}
\def\PYZrb{]}
\makeatother


    % For linebreaks inside Verbatim environment from package fancyvrb. 
    \makeatletter
        \newbox\Wrappedcontinuationbox 
        \newbox\Wrappedvisiblespacebox 
        \newcommand*\Wrappedvisiblespace {\textcolor{red}{\textvisiblespace}} 
        \newcommand*\Wrappedcontinuationsymbol {\textcolor{red}{\llap{\tiny$\m@th\hookrightarrow$}}} 
        \newcommand*\Wrappedcontinuationindent {3ex } 
        \newcommand*\Wrappedafterbreak {\kern\Wrappedcontinuationindent\copy\Wrappedcontinuationbox} 
        % Take advantage of the already applied Pygments mark-up to insert 
        % potential linebreaks for TeX processing. 
        %        {, <, #, %, $, ' and ": go to next line. 
        %        _, }, ^, &, >, - and ~: stay at end of broken line. 
        % Use of \textquotesingle for straight quote. 
        \newcommand*\Wrappedbreaksatspecials {% 
            \def\PYGZus{\discretionary{\char`\_}{\Wrappedafterbreak}{\char`\_}}% 
            \def\PYGZob{\discretionary{}{\Wrappedafterbreak\char`\{}{\char`\{}}% 
            \def\PYGZcb{\discretionary{\char`\}}{\Wrappedafterbreak}{\char`\}}}% 
            \def\PYGZca{\discretionary{\char`\^}{\Wrappedafterbreak}{\char`\^}}% 
            \def\PYGZam{\discretionary{\char`\&}{\Wrappedafterbreak}{\char`\&}}% 
            \def\PYGZlt{\discretionary{}{\Wrappedafterbreak\char`\<}{\char`\<}}% 
            \def\PYGZgt{\discretionary{\char`\>}{\Wrappedafterbreak}{\char`\>}}% 
            \def\PYGZsh{\discretionary{}{\Wrappedafterbreak\char`\#}{\char`\#}}% 
            \def\PYGZpc{\discretionary{}{\Wrappedafterbreak\char`\%}{\char`\%}}% 
            \def\PYGZdl{\discretionary{}{\Wrappedafterbreak\char`\$}{\char`\$}}% 
            \def\PYGZhy{\discretionary{\char`\-}{\Wrappedafterbreak}{\char`\-}}% 
            \def\PYGZsq{\discretionary{}{\Wrappedafterbreak\textquotesingle}{\textquotesingle}}% 
            \def\PYGZdq{\discretionary{}{\Wrappedafterbreak\char`\"}{\char`\"}}% 
            \def\PYGZti{\discretionary{\char`\~}{\Wrappedafterbreak}{\char`\~}}% 
        } 
        % Some characters . , ; ? ! / are not pygmentized. 
        % This macro makes them "active" and they will insert potential linebreaks 
        \newcommand*\Wrappedbreaksatpunct {% 
            \lccode`\~`\.\lowercase{\def~}{\discretionary{\hbox{\char`\.}}{\Wrappedafterbreak}{\hbox{\char`\.}}}% 
            \lccode`\~`\,\lowercase{\def~}{\discretionary{\hbox{\char`\,}}{\Wrappedafterbreak}{\hbox{\char`\,}}}% 
            \lccode`\~`\;\lowercase{\def~}{\discretionary{\hbox{\char`\;}}{\Wrappedafterbreak}{\hbox{\char`\;}}}% 
            \lccode`\~`\:\lowercase{\def~}{\discretionary{\hbox{\char`\:}}{\Wrappedafterbreak}{\hbox{\char`\:}}}% 
            \lccode`\~`\?\lowercase{\def~}{\discretionary{\hbox{\char`\?}}{\Wrappedafterbreak}{\hbox{\char`\?}}}% 
            \lccode`\~`\!\lowercase{\def~}{\discretionary{\hbox{\char`\!}}{\Wrappedafterbreak}{\hbox{\char`\!}}}% 
            \lccode`\~`\/\lowercase{\def~}{\discretionary{\hbox{\char`\/}}{\Wrappedafterbreak}{\hbox{\char`\/}}}% 
            \catcode`\.\active
            \catcode`\,\active 
            \catcode`\;\active
            \catcode`\:\active
            \catcode`\?\active
            \catcode`\!\active
            \catcode`\/\active 
            \lccode`\~`\~ 	
        }
    \makeatother

    \let\OriginalVerbatim=\Verbatim
    \makeatletter
    \renewcommand{\Verbatim}[1][1]{%
        %\parskip\z@skip
        \sbox\Wrappedcontinuationbox {\Wrappedcontinuationsymbol}%
        \sbox\Wrappedvisiblespacebox {\FV@SetupFont\Wrappedvisiblespace}%
        \def\FancyVerbFormatLine ##1{\hsize\linewidth
            \vtop{\raggedright\hyphenpenalty\z@\exhyphenpenalty\z@
                \doublehyphendemerits\z@\finalhyphendemerits\z@
                \strut ##1\strut}%
        }%
        % If the linebreak is at a space, the latter will be displayed as visible
        % space at end of first line, and a continuation symbol starts next line.
        % Stretch/shrink are however usually zero for typewriter font.
        \def\FV@Space {%
            \nobreak\hskip\z@ plus\fontdimen3\font minus\fontdimen4\font
            \discretionary{\copy\Wrappedvisiblespacebox}{\Wrappedafterbreak}
            {\kern\fontdimen2\font}%
        }%
        
        % Allow breaks at special characters using \PYG... macros.
        \Wrappedbreaksatspecials
        % Breaks at punctuation characters . , ; ? ! and / need catcode=\active 	
        \OriginalVerbatim[#1,codes*=\Wrappedbreaksatpunct]%
    }
    \makeatother

    % Exact colors from NB
    \definecolor{incolor}{HTML}{303F9F}
    \definecolor{outcolor}{HTML}{D84315}
    \definecolor{cellborder}{HTML}{CFCFCF}
    \definecolor{cellbackground}{HTML}{F7F7F7}
    
    % prompt
    \makeatletter
    \newcommand{\boxspacing}{\kern\kvtcb@left@rule\kern\kvtcb@boxsep}
    \makeatother
    \newcommand{\prompt}[4]{
        \ttfamily\llap{{\color{#2}[#3]:\hspace{3pt}#4}}\vspace{-\baselineskip}
    }
    

    
    % Prevent overflowing lines due to hard-to-break entities
    \sloppy 
    % Setup hyperref package
    \hypersetup{
      breaklinks=true,  % so long urls are correctly broken across lines
      colorlinks=true,
      urlcolor=urlcolor,
      linkcolor=linkcolor,
      citecolor=citecolor,
      }
    % Slightly bigger margins than the latex defaults
    
    \geometry{verbose,tmargin=1in,bmargin=1in,lmargin=1in,rmargin=1in}
    
    

\begin{document}
    
    \maketitle
    
    

    
    \tableofcontents 
\setcounter{section}{0}

    \hypertarget{abstract}{%
\section{Abstract}\label{abstract}}

The investigation of thick film semiconducting metaloxide gas sensors
(SMOX) offers for a scientist many secrets to reveal. Hidden in the
mystery of semiconductors most process are not available by simple
measuring techniques. The general procedure of investigating a material
follows the following lines:

A measuring technique is applied and based on the state-of-the-art
understanding of the transduction mechanism the results are interpreted.
The risk of getting to wrong conclusions are fairly high since the
amount of assumption in the logical conclusion-path are quit high.

A major key figure measured is often the resistance of the sensor.
Therefore one important elements in this logical path of conclusion is
how electrical properties of the semiconductor result in a resistance.
Further more the how (relative) changes of the properties result in
changes in the resistance. With this knowledge other processes related
i.e.~to chemical surface reaction my be understood better.

    \hypertarget{general-aspects-of-sno_2-based-gas-sensors}{%
\section{\texorpdfstring{General Aspects of \(SnO_2\) based gas
sensors}{General Aspects of SnO\_2 based gas sensors}}\label{general-aspects-of-sno_2-based-gas-sensors}}

\hypertarget{common-preparation-routes-of-sno_2-materials}{%
\subsection{\texorpdfstring{Common preparation routes of \(SnO_{2}\)
materials}{Common preparation routes of SnO\_\{2\} materials}}\label{common-preparation-routes-of-sno_2-materials}}

\hypertarget{sol-gel}{%
\subsubsection{SOl gel}\label{sol-gel}}

\hypertarget{morphologies-and-shape}{%
\paragraph{Morphologies and shape}\label{morphologies-and-shape}}

\hypertarget{flame-spray-pyrolysis}{%
\subsubsection{Flame spray pyrolysis}\label{flame-spray-pyrolysis}}

\hypertarget{morphologies-and-shape-1}{%
\paragraph{Morphologies and shape}\label{morphologies-and-shape-1}}

\hypertarget{bulk-properties-of-the-sno_2}{%
\subsection{Bulk properties of the
SnO\_\{2\}}\label{bulk-properties-of-the-sno_2}}

\hypertarget{work-function}{%
\subsection{Work function}\label{work-function}}

\hypertarget{kelvin-probe}{%
\subsubsection{Kelvin Probe}\label{kelvin-probe}}

\hypertarget{surface-reactions}{%
\subsection{Surface reactions}\label{surface-reactions}}

\hypertarget{modelling-for-the-transduction-and-reaction}{%
\subsection{Modelling for the transduction and
reaction}\label{modelling-for-the-transduction-and-reaction}}

\hypertarget{general-principle-of-measurement}{%
\subsubsection{General principle of
measurement}\label{general-principle-of-measurement}}

\hypertarget{schottky-approximation}{%
\subsubsection{Schottky approximation}\label{schottky-approximation}}

\hypertarget{boltzmann-approximation}{%
\subsubsection{Boltzmann Approximation}\label{boltzmann-approximation}}

\hypertarget{effective-density-of-states-n_c}{%
\subsubsection{Effective density of states
N\_\{C\}}\label{effective-density-of-states-n_c}}

\hypertarget{current-models-state-of-the-art}{%
\subsubsection{Current models, State of the
art}\label{current-models-state-of-the-art}}

    \hypertarget{motivation}{%
\section{Motivation}\label{motivation}}

The research on semiconducting metal oxide gas sensors was focusing in
the past mostly on scenarios, where oxygen is the most dominant
reactive, gaseous, species in the proximity of the sensor. The adsorbed
oxygen at the surface of the semiconductor lead to an interaction with
the charges inside the thick film grains. Due to the trapping of the
charge carrier by oxygen, a depletion layer is formed in the surface
region of the semiconductor. Based on this depletion layer assumption,
many investigations have successfully lead to a deep understanding of
the sensing mechanism.

Nevertheless, the assumption the existence of an depletion layer is not
always valid. Recent experimental results have shown, that even under
atmospheric conditions which are common in real live, the dominant
impact of oxygen may be gone. It could be shown, that under conditions
of 50\% r.h. and low concentrations of CO (\textgreater1 ppm) in
synthetic air, the depletion layer was gone and a accumulation layer
manifested. The results are shown in Figure

\begin{figure}
\centering
\includegraphics{media/pics/external_plots/nitroline_switch_julia.jpg}
\caption{Nitrogen switch}
\end{figure}

With the absence of the depletion layer also most simplifications are
not valid anymore. Mainly the validity of the Schottky and Boltzmann
approximation are not given anymore. Facing those facts the equations to
describe the transduction mechanism for a specific electrical band
configurations needed a more general descriptions, which includes a
depletion and accumulation layer controlled transduction mechanism.

To describe such a semiconductor the set of differential equations is
available in literature. Since the previously used simplifications for
such problems are not valid in the generalized case, finding an
analytically solution exceeded by far my intellectual capabilities.This
is why a I chose to go a different. Find a numerical solution for the
problem by using computational power

While working in the field of SMOX allready some years, I was used to
describe transduction processes by assigning different parts of an
analytical solution to properties of the sensor. With a numerical
solution this is not possible anymore and this was certainly a drawback
of this method. But on the other side the relation between an effect and
a intrinsic property can be studied also in detail, when numerically
solving the equations for multiple values of the intrinsic property. The
influence of may be studied and an insight about the principles can be
gained. Therefore the goal was first to break the problem of describing
a SMOX sensor into smaller parts discreet parts and second trying to
solve each of it individually.

In the upcoming chapters I will describe the different part and how they
have been simplified, solved individually and combined again. Each part
will also hold a section where experimental data is compared with the
numerically gained results.

    \hypertarget{numerical-calculation-of-semiconductors-gas-sensors}{%
\section{Numerical calculation of semiconductors gas
sensors}\label{numerical-calculation-of-semiconductors-gas-sensors}}

    \hypertarget{introduction}{%
\subsection{Introduction}\label{introduction}}

To elaborate the modeling of sensing, different equations, many of which
are described in {[}sec:current models{]} such as the shape dependent
Poisson equation or the combination between the Poisson equation, the
electro-neutrality equation and the geometry dependent electrical
current path, must be solved. In most cases, this is an extensive
mathematical effort and therefore, the numerical computing environments
Python will be used to derive numerical solutions for equations, which
cannot be solved analytically, as shown below. Depending on the grains
size the charge trapping at the surface has different impact on the
potential and charge distribution inside the grain. For large grains
compared to their Debye length (\(L_D\)), a charge transfer at the
surface may be leave the bulk unaffected. In contrast to large grains,
relative small grains may be affected through the whole grain (Figure
{[}fig:Potential-for-spheriacal,small\_1ld{]}).

To understand the influence of a the charge transfer at the surface the
resulting potential distribution which propagates inside the grain is
from main importance. With this knowledge the free charge carrier
concentration and a position dependent resistivity can be defined.

When investigating the total resistance of one grain, the pathway
through the grain plays a major role. The total resistance may vary a
lot based on the an-isotropic resistivity distribution.

In the simplest case of symmetric contacts on opposite sites, a
graphical representation of the results from a numerical simulation of
one grain with a small depletion layer is shown in (Figure 3). The
resulting total resistance of one grain as a function of the surface
potential can then be used as an approximation for the resistance of a
full multi-grain - sensor. (Figure
{[}fig:grain\_simulation\_small\_depl{]}). The relation between surface
potential change and resistance change may also be obtained
experimentally as described in the previous paragraph with the
Kelvin-Probe-Method (REF).By comparing the Kelvin-Probe-Data with the
results from the numerical simulation a better understanding of the
influence of the grain size on the performance can be gained. The
fitting of the numerical parameters of the simulation to experimental
data or generally fitting of data with an appropriate function can
easily be performed in the presented environment.

    \hypertarget{semiconductor-properties-of-the-smox-grains}{%
\subsection{Semiconductor properties of the SMOX
grains}\label{semiconductor-properties-of-the-smox-grains}}

The advantages of industrialized production techniques are inline with
general advantages of the SMOX-based sensor technologie. Both are:

• upscalable

• highly reproducible

• low cost

Besides the benefits for the industrialization of such a material, also
the resulting morphological are beneficial for a good sensor
performance. The typical spherical grains with a narrow size
distribution result in high surface to volume ration. Besides the high
surface area, the high number of grain-grain contacts have a positive
effect on the sensitivity of the sensor, due to the high number of back
to back Schottky barriers. As described in {[}sec:current models{]}such
barriers are of major importance for the sensing properties.

In the literature other geometries claim exceptional performances for
multiple other shapes. As from hollow spheres to nano-rods, often the
mechanism which explains the measured increase in performance is not
explained. Since the aim is to gain a fundamental understanding of the
shape influence while staying close to a industrialize material such
exotic shapes I will keep the focus in this work on spherical grains,
which are commonly used in commercial products. Nevertheless the
techniques described in my work should be transferable to arbitrary and
more complex shapes.

Besides the shape also the defect concentration and stochastic
composition varies a lot with the preparation process. One additional
goal of this thesis is to gain a better understanding about the relation
of these two properties and the sensor performance.

    \hypertarget{choice-of-geometric-model}{%
\subsection{Choice of geometric model}\label{choice-of-geometric-model}}

Both favorable preparation methods mentioned in the latter paragraph
have been investigated under a REM (Figure {[}fig:REM\_grain\_SMOX{]}).
The SMOX grains can be well approximated as spherical grains. The
typical diameters of the grains are from 5nm to 200nm. The benefit by
choosing of such a shape with a the rotational symmetry is the reduction
of the complexity for the numerical calculation. Therefore the
approximation of the SMOX particles as spheres was chosen. The second
benefit of choosing materials prepared by rather standardized
preparation routes is the availability of multiple different materials
form varying laboratories around the world. These materials may vary in
sizes and defect concentration but are often similar in shape. This fact
is favorable when the numerical results are compared and validated with
experimental data.

\href{WHAT!}{Picture of a typical SMOX-Sensor with the grains visible}

Other available materials with more specialized shapes as hallow spheres
or fibers do exist, but will not be investigated in the research. In the
first place the complex numerical description of such geometries will
increase the calculation duration. Also the limited availability and
variety in parameters as diameter, doping level, band gap, material
composition are not favorable for to check the numerical model with
experimental data.

    \hypertarget{poissons-equation}{%
\subsection{Poisson's equation}\label{poissons-equation}}

Reactions at the surface (see {[}sec:Surface-reactions{]}) result in a
charge transfer between the bulk of a grain and the surface. This
modification of the charge density distribution inside the grain causes
again a change in resistivity . By moving charges to/away from the
surface, electrical potentials through the grain are generated. The
electrical potential at the surface and therefore the work function of
the semiconductor changes (see {[}subsec:Work-function{]}).

Previous studies have shown, that a direct relation between surface
potential, surface charge and resistance exists. The latter studies
initially define certain approximations which have been adapted and
reasonable for the investigated cases, but do not allow a predictions
outside the boundaries of the pre-assumption. Also the direct impact of
size and geometric on the transduction is not taken fully into
consideration. In order to have a more general model of the SMOX
materials and to include the geometric effects, the charge distribution
has to be solved in a more general way.

Identical to the previous studies the relation between surface potential
and charge distribution has to be solved initially. This relation is
defined by the Poisson-Law:

\begin{align}
\nabla\phi=-\frac{\rho}{\epsilon\epsilon_{0}} \label{eq1}\tag{Poisson}
\end{align}

\(\phi\)=electrical potential, \(\rho\)=free charge density,
\(\epsilon\)=vacuum permittivity, \(\epsilon_{0}\)=relative
permittivity.

TODO TEM analysis of the crystal quality of one grain indicate a high
crystalline quality. Therefore one can assume that \(\epsilon\) does not
vary inside the grain. The charge destiny \(\rho\) on the other side is
directly influenced by the charge transfer. Since the transfer of
charges to the surface influences the work function and the energetic
position of the conduction band, \(\rho\) is a function of \(\phi\).
\(\phi\) again depends on the position in the grain. At the surface
\(\phi(r=r_{S})\) corresponds to the surface potential \(\phi_s\), while
in the center \(\phi(r=0)=\phi_b\) may have a different value. The exact
shape of \(\phi(r)\) is gained from solving the Poisson equation
\((\ref{eq1})\).

It will be assumed, that the reactions take place a the surface of the
crystal and bulk diffusion will be neglected. Even if there are reports
of oxygen bulk diffusion for certain materials (Quelle), this is not the
general behavior and does not apply apply for \(SnO_{2}\). Since all
surface sites will be accessible by the gas, the solution of
{[}eq:poisson\_eq{]}should have a rotational symmetry. In the case of
multiple grains with grain-grain contacts this assumptions needs to be
validated again.

In case of an rotational symmetric shape of a SMOX grain, equation
{[}eq:poisson\_eq{]}can be expressed as a ordinary differential equation
of only the radius:

\((\frac{1}{r²})\frac{d}{dr}\frac{r²d\phi(r)}{dr}=-\frac{\rho(r)}{\epsilon\epsilon_{0}}\)

Description of the coordinate system of one spherical symmetric SMOX
grain\ldots.

    \hypertarget{charge-density}{%
\subsection{Charge density}\label{charge-density}}

As mentioned, the scope of my work was to investigate the transduction
mechanism. Specially including the phenomena of a switch from a
depletion- to accumulation layer controlled transduction mechanism. The
previously shown conduction mechanism switch an experiment is shown,
where such a switch occurs under application relevant environmental
conditions (50\% r.h. and \textasciitilde3 ppm CO). The findings are
described in detail in \cite{Barsan2015}.

To summarize the findings it can be said, that the resistance drops
below the level obtained under inert conditions in nitrogen, which means
that the material is getting more conductive that it is initially. The
root cause of the higher conductivity is the increased number of free
charge carrier in the conduction band. In contrast to a depletion layer
controlled transduction mechanism a accumulation layer is formed while
charges are pushed into the conduction band.

In cases of depletion layer controlled transduction the
Schottky-Approximation as described in the beginning was proven to be a
effective way to describe and simplify the Poisson equation. In the case
of a accumulation layer the assumption of a fully depleted space charge
layer is not valid anymore.

It should be mentioned, that a common second approximation often used
together with Schottky's approximation is Boltzmann's approximation. The
Boltzmann approximation is valid if the energetic difference between the
energy E and the Fermi level energy \(E_{Fermi}\) is high enough:
\begin{align}
E-E_{Fermi}\gg3k_{B}T
\end{align} Then the Fermi-Dirac distibution \(f(E)\) can be expressed
with the Boltzmann distribution b(E):

\begin{align}
f(E)=\frac{1}{exp(\frac{E-E_{Fermi}}{k_{B}T})+1}\xrightarrow{Boltzmann\,Conditions}b(E)=exp(-\frac{(E-E_{Fermi})}{k_{B}T})
\end{align}

Based on the findings, that the flat band situation is reached in
application relevant conditions, the Boltzmann approximation i not valid
anymore. Operando Kelvin Probe (see: {[}subsec:Kelvin-Probe{]})
experiments indicate, that the surface potential may drop up to 1eV
below the nitrogen level \cite{Barsan2015}. The typical difference
\(E_{Conduction}-E_{Fermi}\) is between 50meV and 300meV. It is
reasonable to expect, that the conduction band may even cross the Fermi
level and therefore also the conditions necessary for the Boltzmann
approximation do not exist. In the upcomming calculations I will
calculate and show relative error, when using the Boltzmann
approximation for a relevant range of energy bendings of the conduction
band.

\cite{Barsan2011a} and its validity for the case of an accumulation
layer will be shown in the upcoming calculations.

To unify all transduction mechanisms into on numerical calculation the
Fermi-Dirac distribution, without further simplification, is used to
calculate the charge distributions.

We will begin by rewrite the Fermi-Dirac equation to a suitable format,
which will reflect the the occupation probability at energies relative
to the initial conduction band position \(E_C\):

Ferimi-Dirac: \begin{align}
f(E)=\frac{1}{exp(\frac{E-E_{Fermi}}{k_{B}T})+1}=\frac{1}{exp(\frac{E-E_{C}+E_{C}-E_{Fermi}}{k_{B}T})+1}=\frac{1}{exp(\frac{E_{C}-E_{Fermi}}{k_{B}T})*exp(\frac{E-E_{C}}{k_{B}T})+1}
\end{align}

The density of states is given by {[}\#Sze2007{]}for and electron in the
conduction band with the effective mass m\^{}\{*\}as followed: DERIVE
NUMERICALLY!

\begin{align}
N_{E_{C}}\left(E\right)=\frac{\sqrt{2}}{\Pi^{^{2}}}\frac{\sqrt{E-E_{C}}}{\hbar^{3}}m^{*^{\frac{3}{2}}}=4\Pi*\frac{\left(2*m^{*}\right)^{\frac{3}{2}}}{h^{3}}*\sqrt{E-E_{C}}
\end{align}

With the occupation probability \(f(E)\) and the density of states
\(N(E)\) the number of charges in the conduction band can be calculated:
\begin{align}
n\left(E_{C}\right)=\intop_{E_{C}}^{inf}N_{E_{C}}\left(E\right)*f\left(E\right)dE
\end{align}

Typically this equation is simplified to the following form. Such an
analytical equation is useful for further alaytical calculations but is
not necessary for our numerical approach:

\begin{align}
n(E_{C})=N_{C}exp\left(\frac{E_{F}-E_{C}}{k_{B}T}\right)\label{n}\tag{$n(E_C)$}
\end{align}

with
\(N_{C}=2\left(\frac{2\Pi m_{e}^{*}k_{B}T}{h²}\right)^{\frac{3}{2}}\),
the effective density of states in the conduction band. It is worth-wise
to mention, that this simplification is only valid if also the Boltzmann
approximation is valid.

Equation {[}eq:integral\_n{]} was solved numerically and compared with
the results obtained with the common approximations(
\(m_{e}^{*}=0.3m_{e}\) for \(SnO_{2}\) \cite{Batzill2005} was chosen .

Previous publications prove, that above a operation temperature of
300°C, all donors are ionized and available as free charge carriers in
the conduction band \cite{Barsan2015}. If some of those electrons are
trapped at the surface due to surface reaction, a positive charge
remains localized in the crystal at the donors position. Additionally
the energetic position of the conduction band increases with electron
trapped at the surface. Out of the combination of conduction band shift
\(E=E_{C}-E_{C_{b}}\) and equation \ref{n}, one can calculate now the
free charge carrier density \(\rho\) from Poisson's equation.

In the case of an unaffected bulk, \(n(E_{C_{b}})\equiv n_{b}\) is the
density of electrons in the conduction band. In case of a charge
transfer to/from the surface, the number of electrons in the conduction
band will change. The relation between the density of charges in the
conduction band \(n(E_{C})\) and the shifted, new energetic position of
the conduction band \(E_{C}\) is fixed by equation \ref{n}. The
difference between \(n_{b}\) and \(n(E_{C})\) is the density of the
positive, ionized donors remaining in the crystal. Those remaining
donors are the cause of the electrical screening of the surface
potential. The decay of the energetic conduction band position from the
surface energy back to the `bulk position' depends directly on that
number. With this relation a energy dependent charge density can be
formulated as followed:

\begin{align}
\frac{\rho(E)}{e}=n(E_{C_b})-n(E) = n_{b}-n(E)
\end{align}

Equation \ref{n} becomes then:

\begin{align}
\left(\frac{1}{r²}\right)\frac{d}{dr}\frac{r²d\phi}{dr}=-\frac{\rho\left(E\left(r\right)\right)}{\epsilon\epsilon_{0}}=-\frac{e\left(n\left(E_{C_{b}}\right)-n\left(E\right)\right)}{\epsilon\epsilon_{0}}=-\frac{e\left(n_{b}-n\left(E\right)\right)}{\epsilon\epsilon_{0}}\label{poisson_spherical}\tag{Poisson spherical}
\end{align}

With \(E=V*e=(\Phi_{0}-\Phi)*e\)

``In discussions of semiconductos, it is useful to define a''band
bending" function \(V\) such that \(eV\) is related to the potential
energy of an electron''\cite{S.RoyMorrison1977}::

\(V=\phi_{b}-\phi,\;E=V*e\)

With this relation equation (\ref{poisson_spherical}) becomes:

\begin{align}
\left(\frac{1}{r²}\right)\frac{d}{dr}\frac{r²dV}{dr}=\frac{e\left(n_{b}-n\left(E\right)\right)}{\epsilon\epsilon_{0}}\label{poisson_spher_v}\tag{Poisson spherical (V)}
\end{align}

The goal will be to calculate how potential at the surface is
electrically screen by the remaining positive charges in side the grain.
An important parameter for such calculations is the Debye length. In
cases where the Boltzmann approximation is valid, the Debye length can
be approximated with the following formula. In this case the given
length is the distance needed to screen a potential \(V\) until its
value reaches \(\frac{V}{e}\)

\begin{align}
L_{D}=\sqrt{\frac{\epsilon\epsilon_{0}k_{B}T}{n_{b}e^{2}}}
\end{align}

With the definition of the Debye length I can now transform the relevant
variables of the calculation.

\begin{itemize}
\item
  The distance inside the grain r is expressed in units of the Debye
  length \(L_{D}\) :

  \(r^{*}=\frac{r}{L_{D}}, \frac{dr^{*}}{dr}=\frac{1}{L_{D}}\longrightarrow dr=dr^{*}*L_{D}\)
\item
  The position of the conduction band inside the grain in units of the
  \(\frac{k_{B}T}{e}\):

  \(V^{*}=\frac{e}{k_{B}T}*V, \frac{dV^{*}}{dV}=\frac{e}{k_{B}T}\longrightarrow dV=dV^{*}*\frac{k_{B}T}{e}\)
\item
  And the number of free charge carries in units of the intrinsic number
  of charges \(n_b\):

  \(n^{*}(V^{*})=\frac{n(V)}{n_{b}}\)
\end{itemize}

By substituting those unit-less parameters in equation
(\ref{poisson_spher_v}), one obtains the a unit-less Poisson equation
suitable for the numerical calculations:

\begin{align}
\frac{1}{r^{*2}}\frac{d}{dr^{*}}{r^{*}}^{2}\frac{dV^{*}}{dr^{*}}=1-n^{*}(V^{*})\label{poisson_no_units}\tag{Unitless Poisson equation}
\end{align}

This step of substituting the equation with unit less parameters is not
obligatory for the the numerical calculations. As I will show in the
next part of the thesis, the numerical calculation is also be possible
with the initial spherical Poisson equation (\ref{poisson_spher_v}).
Therefore the material specific parameters need to be given to the
algorithm. The downside would be, that for every new material with any
parameter changing, the calculation would need to be redone. The benefit
of the latter derived equation is that it is valid for multiple
combinations of intrinsic parameters. The solution would only depend on
three parameters:

\begin{itemize}
\tightlist
\item
  Grainsize \(R\) in units of \(L_{D}\)
\item
  Temperature \(T\) as in \(\frac{k_{B}T}{e}\)
\item
  The doping level of the semiconductor described with \(n_b\)
\end{itemize}

The advantage of the numerical approch is now, that for typical values
of these parameters the solution are computed and used for further
understanding of their influence on sensing with SMOX material. Typical
values of the relevant parameters are:

\begin{itemize}
\tightlist
\item
  Typical grainsizes reach from 0.1 to 100 \(L_D\)
\item
  Typical temperatures are in the range of 100°C to 400°C
\item
  The doping level \(n_b\) range from \(10^{19}\frac{1}{m^3}\) to
  \(10^{25}\frac{1}{m^3}\)
\end{itemize}

This indicates just the typical materials, but solutions for other
parameters are alos possible. For the scope of my work I will
nevertheless concentrate on the gives ranges.

\hypertarget{poisson-equation-as-system-of-odes}{%
\subsection{Poisson equation as system of
ODEs}\label{poisson-equation-as-system-of-odes}}

The Python SciPy package \cite{Jones} will now be used to numerically
solve the derived equations. The odesolvers of scipy solve first order
ODEs, or systems of first order ODES. To solve a second order ODE, we
must convert it by changes of variables to a system of first order ODES.

Equation (\ref{poisson_no_units}) is an ODE of second order, so it first
had to be converted by changes of variables to a system of first order
ODES.

Practically a functions needs to be written, which gets as input an
array of values and returns returns an array of the derived values:

EXPLAIN HOW OED SOLVER WORKS

\begin{align}
derive\_func\left(V^{*},\frac{dV^{*}}{dr^{*}}\right)\longrightarrow\frac{dV^{*}}{dr^{*}},\frac{dV^{*2}}{dr^{*2}}
\end{align}

The the second input term \(\frac{dV^{*}}{dr^{*}}\) corresponds already
to the first output term. So no special work needs to be done here. But
also the second output parameter can be calculated with the given input
parameters by using unitless poisson equation (\ref{poisson_no_units}).

\begin{align}
\frac{1}{r^{*2}}\frac{d}{dr^{*}}{r^{*}}^{2}\frac{dV^{*}}{dr^{*}}=1-n^{*}(V^{*})=\frac{2r^{*}}{r^{*2}}\frac{dV^{*}}{dr^{*}}+\frac{r^{*2}}{r^{*2}}\frac{d²V^{*}}{dr^{*2}}=\frac{2}{r^{*}}\frac{dV^{*}}{dr^{*}}+\frac{d²V^{*}}{dr^{*2}}=1-n^{*}(V^{*}) 
\end{align}

\begin{align}
\frac{d²V^{*}}{dr^{*2}}=1-n^{*}(V^{*})-\frac{2}{r^{*}}\frac{dV^{*}}{dr^{*}}\label{second_derivative}\tag{Second derivative}
\end{align}

The odesolver needs beside the derive-function additional parameters.
Namely a set of initial start values for \(V^{*}\) and
\(\frac{dV^{*}}{dr^{*}}\) and boundaries between the solver should
calcualte the solution. Since we first want to calculate the shape of
the conduction band for different surface potentials \(V^{*}\), the
initial parameter is already defined \(V^{*}\) . Also the boundaries
should be the boundaries of the grain, r\^{}* between 0 and the
grainsize R\^{}*.

\(odesolver(derive\_func,[V^{*}{}_{init},\frac{dV^{*}}{dr^{*}}_{init}],r)\longrightarrow V^{*}(r^{*}),\frac{dV^{*}}{dr*}(r*)\)

If such a function can be defined, the solver is able to solve the
system of equation interactively starting from a given initial
condition.

    \hypertarget{constants}{%
\subsection{Constants}\label{constants}}

For the numerical calculations I will need to use some constants to
refer to. To structure this notebook it is favorable to concentrate the
definition at a single point and refer always back to this definition.
This reduces the potential error of typos when using some constants over
and over again.

One way to generate a global object which groups the relevant
information together and allows to access them easily are classes. Such
classes do not only store the relevant information but offer also some
useful functionalities related to the stored information.

In the following code block, the class is defined with the statement
\texttt{class}. What the follows \texttt{class} statement is the name of
the class. By convention class names start with a capital letter. Inside
the \texttt{class} initially one function called \texttt{\_\_init\_\_}
is defined with the \texttt{def\ \_\_init\_\_(self):} statement. This
special function is always automatically executed when an instance of
the class is created. Here I define some constants, which are relevant
for course of this thesis. I also add some useful functions, e.g.~the
conversion from Celsius to Kelvin and vise versa. Such functions will be
of mayor importance when doing the knowledge transfer gained in the
`semiconductor regime' (Kelvin is the useful scale here) to application
relevant conditions (where Celsius is the usual temperature scale).

    \begin{tcolorbox}[breakable, size=fbox, boxrule=1pt, pad at break*=1mm,colback=cellbackground, colframe=cellborder]
\prompt{In}{incolor}{ }{\boxspacing}
\begin{Verbatim}[commandchars=\\\{\}]
\PY{k+kn}{from} \PY{n+nn}{scipy} \PY{k+kn}{import} \PY{n}{constants} \PY{k}{as} \PY{n}{sciConst}
\PY{k}{class} \PY{n+nc}{Constant}\PY{p}{:}
    \PY{k}{def} \PY{n+nf+fm}{\PYZus{}\PYZus{}init\PYZus{}\PYZus{}}\PY{p}{(}\PY{n+nb+bp}{self}\PY{p}{)}\PY{p}{:}
        \PY{n+nb+bp}{self}\PY{o}{.}\PY{n}{K0} \PY{o}{=} \PY{n}{sciConst}\PY{o}{.}\PY{n}{convert\PYZus{}temperature}\PY{p}{(}\PY{l+m+mi}{0}\PY{p}{,}\PY{l+s+s1}{\PYZsq{}}\PY{l+s+s1}{C}\PY{l+s+s1}{\PYZsq{}}\PY{p}{,}\PY{l+s+s1}{\PYZsq{}}\PY{l+s+s1}{K}\PY{l+s+s1}{\PYZsq{}}\PY{p}{)}
        \PY{n+nb+bp}{self}\PY{o}{.}\PY{n}{kB}\PY{o}{=}\PY{n}{sciConst}\PY{o}{.}\PY{n}{k}
        \PY{n+nb+bp}{self}\PY{o}{.}\PY{n}{EPSILON\PYZus{}0} \PY{o}{=} \PY{n}{sciConst}\PY{o}{.}\PY{n}{epsilon\PYZus{}0}
        \PY{n+nb+bp}{self}\PY{o}{.}\PY{n}{E\PYZus{}CHARGE} \PY{o}{=} \PY{n}{sciConst}\PY{o}{.}\PY{n}{elementary\PYZus{}charge}
        \PY{n+nb+bp}{self}\PY{o}{.}\PY{n}{h} \PY{o}{=} \PY{n}{sciConst}\PY{o}{.}\PY{n}{h}
        \PY{n+nb+bp}{self}\PY{o}{.}\PY{n}{MASS\PYZus{}E} \PY{o}{=} \PY{n}{sciConst}\PY{o}{.}\PY{n}{electron\PYZus{}mass}
        \PY{n+nb+bp}{self}\PY{o}{.}\PY{n}{NA}\PY{o}{=} \PY{n}{sciConst}\PY{o}{.}\PY{n}{N\PYZus{}A}
        \PY{n+nb+bp}{self}\PY{o}{.}\PY{n}{VOL\PYZus{}mol} \PY{o}{=} \PY{l+m+mf}{22.4}
        \PY{n+nb+bp}{self}\PY{o}{.}\PY{n}{mole\PYZus{}per\PYZus{}l} \PY{o}{=} \PY{n+nb+bp}{self}\PY{o}{.}\PY{n}{NA}\PY{o}{/}\PY{n+nb+bp}{self}\PY{o}{.}\PY{n}{VOL\PYZus{}mol}
        
    \PY{k}{def} \PY{n+nf}{K\PYZus{}to\PYZus{}C}\PY{p}{(}\PY{n+nb+bp}{self}\PY{p}{,} \PY{n}{K}\PY{p}{)}\PY{p}{:}
        \PY{k}{return} \PY{n}{sciConst}\PY{o}{.}\PY{n}{convert\PYZus{}temperature}\PY{p}{(}\PY{n}{K}\PY{p}{,} \PY{l+s+s1}{\PYZsq{}}\PY{l+s+s1}{K}\PY{l+s+s1}{\PYZsq{}}\PY{p}{,} \PY{l+s+s1}{\PYZsq{}}\PY{l+s+s1}{C}\PY{l+s+s1}{\PYZsq{}}\PY{p}{)}

    \PY{k}{def} \PY{n+nf}{C\PYZus{}to\PYZus{}K}\PY{p}{(}\PY{n+nb+bp}{self}\PY{p}{,} \PY{n}{C}\PY{p}{)}\PY{p}{:}
        \PY{k}{return} \PY{n}{sciConst}\PY{o}{.}\PY{n}{convert\PYZus{}temperature}\PY{p}{(}\PY{n}{C}\PY{p}{,} \PY{l+s+s1}{\PYZsq{}}\PY{l+s+s1}{C}\PY{l+s+s1}{\PYZsq{}}\PY{p}{,} \PY{l+s+s1}{\PYZsq{}}\PY{l+s+s1}{K}\PY{l+s+s1}{\PYZsq{}}\PY{p}{)}
    
    \PY{k}{def} \PY{n+nf}{eV\PYZus{}to\PYZus{}J}\PY{p}{(}\PY{n+nb+bp}{self}\PY{p}{,}\PY{n}{eV}\PY{p}{)}\PY{p}{:}
        \PY{k}{return} \PY{n}{eV}\PY{o}{*}\PY{n+nb+bp}{self}\PY{o}{.}\PY{n}{E\PYZus{}CHARGE}
    
    \PY{k}{def} \PY{n+nf}{J\PYZus{}to\PYZus{}eV}\PY{p}{(}\PY{n+nb+bp}{self}\PY{p}{,}\PY{n}{J}\PY{p}{)}\PY{p}{:}
        \PY{k}{return} \PY{n}{J}\PY{o}{/}\PY{n+nb+bp}{self}\PY{o}{.}\PY{n}{E\PYZus{}CHARGE}
    
    
\PY{n}{CONST} \PY{o}{=} \PY{n}{Constant}\PY{p}{(}\PY{p}{)}
\end{Verbatim}
\end{tcolorbox}

    Once the basic constants are defined, creating also a simplified
numerical representation of the investigated semiconducting material.
The specific properties of the semiconductor will result in individual
properties as the charge distribution. As explained in the theoretical
section, the charge distribution \(n\), depends on the position of the
conduction band. Again start with setting up a numerical python lab,
which will output all results ``inline'' with this document. As shown in
the introduction this is done by using the magic command
\texttt{\%pylab\ inline}

    \begin{tcolorbox}[breakable, size=fbox, boxrule=1pt, pad at break*=1mm,colback=cellbackground, colframe=cellborder]
\prompt{In}{incolor}{2}{\boxspacing}
\begin{Verbatim}[commandchars=\\\{\}]
\PY{o}{\PYZpc{}}\PY{k}{pylab} inline
\end{Verbatim}
\end{tcolorbox}

    \begin{Verbatim}[commandchars=\\\{\}]
Populating the interactive namespace from numpy and matplotlib
    \end{Verbatim}

    \hypertarget{materials}{%
\subsection{Materials}\label{materials}}

    \hypertarget{solving-integrals-numerically}{%
\subsubsection{Solving integrals
numerically}\label{solving-integrals-numerically}}

\hypertarget{introduction}{%
\paragraph{Introduction}\label{introduction}}

Many observables in nature can be predicted with the solution an
integral of a certain function. In this section I will make a short
excursion on how to solve integrals numerically.

From my experience, many students and researches excellently trained to
define the set of equations describing their problems mathematically.
Seconds, the evaluation of the individual equations with multiple
variables imposes (in most cases) no problem. Third, it is also part of
the common knowledge, that the integral of any function is equivalent to
the area below the curve. A college of mine once told me, that she learn
calculating the integral in school by: 1. drawing the function for
multiple points on a paper 2. combine them with a line 3. count the
squares below the curve

Nothing else is done, when solving integrals numerically.

On the other side solving an integral analytically requires in many
cases advanced mathematical skills and often
approximation/simplifications are introduces to be able to solve the
problem. Those tasks are often hard to master for many people (I
continuously fail at solving integral) and the simplifications often
reduce the solution to specific boundary conditions (as for example the
Boltzmann approximation).

As mentioned solving integrals numerically is fairly easy, even if one
might not feel very comfortable with counting squares. But even if
counting is not an option, there are modern tools to solve this task
very efficiently! If haven't been introduced yet, here they come!

So if the elements to be integrated can evaluated for each point between
the boundaries of the integral, not much is in the way to solve the
integral numerically. Here a simple example of solving:
\(\intop_{3}^{5}x{{}^5}dx\)

The analytical solution solution is:
\(\left[1/6x^6\right]_{3}^{5}=1/6*5^6-1/6*3^6\backsimeq2482.67\)

    \hypertarget{the-quad-function}{%
\paragraph{\texorpdfstring{The \texttt{quad}
function}{The quad function}}\label{the-quad-function}}

The \texttt{quad} function from the \texttt{scipy.integrate} package
will be used to integrate the given function. The \texttt{quad} needs as
comma separated inputs the 1. function to integrate 2. the lower
integration boundary 3. the upper integration boundary

The help file of \texttt{quad} says: \textgreater Integrate func from
\texttt{a} to \texttt{b} (possibly infinite interval) using a technique
from the Fortran library QUADPACK.

This description reveals, that the Fortran library QUADPACK is used in
the background. So nothing new is shown here from the ``scientific''
point of view. I'd rather like to point out, how easy this can be
applied in a Jupyter notebook. From discussion with colleagues I know,
that the biggest challenge is how to technically implement the numerical
solving algorithm in Python. So here it comes:

    \begin{tcolorbox}[breakable, size=fbox, boxrule=1pt, pad at break*=1mm,colback=cellbackground, colframe=cellborder]
\prompt{In}{incolor}{3}{\boxspacing}
\begin{Verbatim}[commandchars=\\\{\}]
\PY{o}{\PYZpc{}}\PY{k}{pylab} inline
\end{Verbatim}
\end{tcolorbox}

    \begin{Verbatim}[commandchars=\\\{\}]
Populating the interactive namespace from numpy and matplotlib
    \end{Verbatim}

    \begin{tcolorbox}[breakable, size=fbox, boxrule=1pt, pad at break*=1mm,colback=cellbackground, colframe=cellborder]
\prompt{In}{incolor}{4}{\boxspacing}
\begin{Verbatim}[commandchars=\\\{\}]
\PY{k+kn}{from} \PY{n+nn}{scipy}\PY{n+nn}{.}\PY{n+nn}{integrate} \PY{k+kn}{import} \PY{n}{quad}

\PY{k}{def} \PY{n+nf}{f}\PY{p}{(}\PY{n}{x}\PY{p}{)}\PY{p}{:}
    \PY{c+c1}{\PYZsh{} return x to the power 5}
    \PY{k}{return} \PY{n}{x}\PY{o}{*}\PY{o}{*}\PY{l+m+mi}{5}

\PY{n}{num\PYZus{}sol}\PY{p}{,} \PY{n}{num\PYZus{}error} \PY{o}{=} \PY{n}{quad}\PY{p}{(}\PY{n}{f}\PY{p}{,}\PY{l+m+mi}{3}\PY{p}{,}\PY{l+m+mi}{5}\PY{p}{)}
\PY{n}{ana\PYZus{}sol} \PY{o}{=} \PY{l+m+mi}{1}\PY{o}{/}\PY{l+m+mi}{6}\PY{o}{*}\PY{p}{(}\PY{l+m+mi}{5}\PY{o}{*}\PY{o}{*}\PY{l+m+mi}{6}\PY{o}{\PYZhy{}}\PY{l+m+mi}{3}\PY{o}{*}\PY{o}{*}\PY{l+m+mi}{6}\PY{p}{)}

\PY{n+nb}{print}\PY{p}{(}\PY{l+s+sa}{f}\PY{l+s+s1}{\PYZsq{}}\PY{l+s+s1}{Numerical solution: }\PY{l+s+si}{\PYZob{}num\PYZus{}sol:.2f\PYZcb{}}\PY{l+s+s1}{ +\PYZhy{} }\PY{l+s+si}{\PYZob{}num\PYZus{}error:.2f\PYZcb{}}\PY{l+s+s1}{\PYZsq{}}\PY{p}{)}
\PY{n+nb}{print}\PY{p}{(}\PY{l+s+sa}{f}\PY{l+s+s1}{\PYZsq{}}\PY{l+s+s1}{Analytical solution: }\PY{l+s+si}{\PYZob{}ana\PYZus{}sol:.2f\PYZcb{}}\PY{l+s+s1}{\PYZsq{}}\PY{p}{)}
\end{Verbatim}
\end{tcolorbox}

    \begin{Verbatim}[commandchars=\\\{\}]
Numerical solution: 2482.67 +- 0.00
Analytical solution: 2482.67
    \end{Verbatim}

    In this case, the numerical and the analytical solution result in the
same results. How about a more complex problem? Let's look at the
``Normal distribution'': \begin{align}
f(x)=\frac{1}{\sqrt{2\Pi}}*e^{-\frac{x²}{2}}
\end{align} Since the probability distribution is normalized the
integral from \(-\infty\) to \(\infty\) is 1: \begin{align}
\intop_{-\infty}^{\infty}f(x)dx=1
\end{align} The analytical solution of this integral is already a rather
advanced task, but still doable. The numerical results are obtained in
the following cell.

    \begin{tcolorbox}[breakable, size=fbox, boxrule=1pt, pad at break*=1mm,colback=cellbackground, colframe=cellborder]
\prompt{In}{incolor}{5}{\boxspacing}
\begin{Verbatim}[commandchars=\\\{\}]
\PY{k}{def} \PY{n+nf}{f}\PY{p}{(}\PY{n}{x}\PY{p}{)}\PY{p}{:}
    \PY{k}{return} \PY{l+m+mi}{1}\PY{o}{/}\PY{p}{(}\PY{l+m+mi}{2}\PY{o}{*}\PY{n}{pi}\PY{p}{)}\PY{o}{*}\PY{o}{*}\PY{l+m+mf}{0.5}\PY{o}{*}\PY{n}{e}\PY{o}{*}\PY{o}{*}\PY{p}{(}\PY{o}{\PYZhy{}}\PY{p}{(}\PY{n}{x}\PY{o}{*}\PY{o}{*}\PY{l+m+mi}{2}\PY{o}{/}\PY{l+m+mi}{2}\PY{p}{)}\PY{p}{)}

\PY{n}{num\PYZus{}sol}\PY{p}{,} \PY{n}{num\PYZus{}error} \PY{o}{=} \PY{n}{quad}\PY{p}{(}\PY{n}{f}\PY{p}{,} \PY{o}{\PYZhy{}}\PY{n}{np}\PY{o}{.}\PY{n}{inf}\PY{p}{,}\PY{n}{np}\PY{o}{.}\PY{n}{inf}\PY{p}{)}
\PY{n}{ana\PYZus{}sol} \PY{o}{=} \PY{l+m+mi}{1}

\PY{n}{num\PYZus{}sol}\PY{p}{,} \PY{n}{num\PYZus{}error} \PY{o}{=} \PY{n}{quad}\PY{p}{(}\PY{n}{f}\PY{p}{,} \PY{o}{\PYZhy{}}\PY{n}{inf}\PY{p}{,}\PY{n}{inf}\PY{p}{)}
\PY{n}{ana\PYZus{}sol} \PY{o}{=} \PY{l+m+mi}{1}

\PY{n+nb}{print}\PY{p}{(}\PY{l+s+sa}{f}\PY{l+s+s1}{\PYZsq{}}\PY{l+s+s1}{Numerical solution: }\PY{l+s+si}{\PYZob{}num\PYZus{}sol:.12f\PYZcb{}}\PY{l+s+s1}{ +\PYZhy{} }\PY{l+s+si}{\PYZob{}num\PYZus{}error:.12f\PYZcb{}}\PY{l+s+s1}{\PYZsq{}}\PY{p}{)}
\PY{n+nb}{print}\PY{p}{(}\PY{l+s+sa}{f}\PY{l+s+s1}{\PYZsq{}}\PY{l+s+s1}{Analytical solution: }\PY{l+s+si}{\PYZob{}ana\PYZus{}sol:.12f\PYZcb{}}\PY{l+s+s1}{\PYZsq{}}\PY{p}{)}
\end{Verbatim}
\end{tcolorbox}

    \begin{Verbatim}[commandchars=\\\{\}]
Numerical solution: 1.000000000000 +- 0.000000010178
Analytical solution: 1.000000000000
    \end{Verbatim}

    Also here a numerical solution is in line with the expected result from
the analytic solution.

There two examples cover functions, where a analytical solution is
known. In the following parts of this thesis, many analytical solutions
are not known. In this case the shortcut of using a numerical solution
instead of relaying on the exact solution is reasonable.

\hypertarget{side-note}{%
\paragraph{Side-Note:}\label{side-note}}

The \texttt{quad} function does not only give back the numerical
solution, but also an estimate of the absolute error in the result.

The \texttt{print()} statement is used to add the results in the output
of the notebook. The \texttt{print} command requires a text
\emph{string} between its parenthesis. In Python a \emph{string}
consists of multiple characters between quotation marks:
e.g.~\texttt{\textquotesingle{}L3TT3R5\textquotesingle{}}. Additionally
another rather new feature of Python is used here. This feature is
called \emph{formatted strings}. \emph{Formatted strings} are
constructed with an \texttt{f} in front of the \emph{string}:
\texttt{f\textquotesingle{}L3TT3R5\textquotesingle{}}.

For formatted strings variables inside curly parenthesis are then
replaced with their string representation. The formatting of the
representation may be given after \texttt{:}. For example \texttt{.12f}
tells the formatter to represent the variable as a float with 12 digits
after the decimal separator. When reading this as an interactive
notebook, feels free to modify the formatting statement and check the
result.

    \hypertarget{numerical-description-of-the-semiconductor}{%
\subsubsection{Numerical description of the
semiconductor}\label{numerical-description-of-the-semiconductor}}

    \hypertarget{helper-functions-for-semiconductor-calculations}{%
\paragraph{Helper functions for semiconductor
calculations}\label{helper-functions-for-semiconductor-calculations}}

Michael Hübner (page 50) derived in his thesis a way to approximate the
energetic difference between the the level of the Fermi-Level and the
conduction band (in a flat band situation) depending of the temperature
\(T\) and concentration of defects in the bulk of the grain \(n_b\) and
the effective mass of electrons inside the semiconductor \(m_e^*\). Both
\(m_e^*\) and \(n_b\) are difficult to measure experimentally. This
thesis will initially start with reasonable values found in literature
to derive the required set of equations describing the transduction.

For a the later deep analysis those two parameters part of a set of
parameters to be screen for the analysis of their influence on the
overall performance of a sensing material.

Based on Michael Hübners work now possible to derive a value for the
energetic distance between Fermi level and conduction band
\(\Delta E_FE_C\).

Since the energetic distance from the Fermi level mainly defines the
occupation probability of the states in the conduction band, this term
is of major importance. It should be pointed out, that the calculation
in the thesis are based special assumption only valid for \(SnO_2\). The
definition is translated into a Python algorithm and used as an starting
point for further calculations. Besides this function also two other
``helper functions'' are defined which will be used at multiple places
in the upcoming calculations.

    \begin{tcolorbox}[breakable, size=fbox, boxrule=1pt, pad at break*=1mm,colback=cellbackground, colframe=cellborder]
\prompt{In}{incolor}{31}{\boxspacing}
\begin{Verbatim}[commandchars=\\\{\}]
\PY{k+kn}{import} \PY{n+nn}{scipy} 
\PY{k}{def} \PY{n+nf}{calc\PYZus{}kT}\PY{p}{(}\PY{n}{T\PYZus{}C}\PY{p}{)}\PY{p}{:}
    \PY{l+s+sd}{\PYZdq{}\PYZdq{}\PYZdq{}}
\PY{l+s+sd}{    Calculate the kT value for a temp. in °C}
\PY{l+s+sd}{    T\PYZus{}C = Temp in °C}
\PY{l+s+sd}{    \PYZdq{}\PYZdq{}\PYZdq{}}
    
    \PY{n}{kT} \PY{o}{=} \PY{n}{CONST}\PY{o}{.}\PY{n}{kB}\PY{o}{*}\PY{p}{(}\PY{n}{CONST}\PY{o}{.}\PY{n}{C\PYZus{}to\PYZus{}K}\PY{p}{(}\PY{n}{T\PYZus{}C}\PY{p}{)}\PY{p}{)}
    \PY{k}{return} \PY{n}{kT}

\PY{k}{def} \PY{n+nf}{calc\PYZus{}eff\PYZus{}density\PYZus{}of\PYZus{}states}\PY{p}{(}\PY{n}{T\PYZus{}C}\PY{p}{,}\PY{n}{mass\PYZus{}e\PYZus{}eff\PYZus{}factor}\PY{p}{)}\PY{p}{:}
    \PY{l+s+sd}{\PYZdq{}\PYZdq{}\PYZdq{}}
\PY{l+s+sd}{    Calculate the eff. densitiy of states in the conduction band}
\PY{l+s+sd}{    T\PYZus{}C = Temp in °C}
\PY{l+s+sd}{    mass\PYZus{}e\PYZus{}eff\PYZus{}factor = material specific factor to calculate the effective mass from the electron mass}
\PY{l+s+sd}{    \PYZdq{}\PYZdq{}\PYZdq{}}
    
    \PY{n}{kT} \PY{o}{=} \PY{n}{calc\PYZus{}kT}\PY{p}{(}\PY{n}{T\PYZus{}C}\PY{p}{)}
    \PY{n}{MASS\PYZus{}E\PYZus{}EFF} \PY{o}{=} \PY{n}{mass\PYZus{}e\PYZus{}eff\PYZus{}factor}\PY{o}{*}\PY{n}{CONST}\PY{o}{.}\PY{n}{MASS\PYZus{}E}
    \PY{n}{NC} \PY{o}{=} \PY{l+m+mi}{2}\PY{o}{*}\PY{p}{(}\PY{l+m+mi}{2}\PY{o}{*}\PY{n}{np}\PY{o}{.}\PY{n}{pi}\PY{o}{*}\PY{n}{MASS\PYZus{}E\PYZus{}EFF}\PY{o}{*}\PY{n}{kT}\PY{o}{/}\PY{p}{(}\PY{n}{CONST}\PY{o}{.}\PY{n}{h}\PY{o}{*}\PY{o}{*}\PY{l+m+mi}{2}\PY{p}{)}\PY{p}{)}\PY{o}{*}\PY{o}{*}\PY{p}{(}\PY{l+m+mf}{3.0}\PY{o}{/}\PY{l+m+mf}{2.0}\PY{p}{)}
    \PY{k}{return} \PY{n}{NC}

\PY{k}{def} \PY{n+nf}{calc\PYZus{}EDCF\PYZus{}by\PYZus{}temp}\PY{p}{(}\PY{n}{T\PYZus{}C}\PY{p}{,} \PY{n}{ND}\PY{p}{,}\PY{n}{mass\PYZus{}e\PYZus{}eff\PYZus{}factor}\PY{p}{)}\PY{p}{:}
    \PY{l+s+sd}{\PYZdq{}\PYZdq{}\PYZdq{}}
\PY{l+s+sd}{    T\PYZus{}C = Temperature in °C}
\PY{l+s+sd}{    }
\PY{l+s+sd}{    ND = number of donors per m³}
\PY{l+s+sd}{    ND = 9e21 \PYZsh{} 9*10**15 cm**3 Mich Thesis Seite 50}
\PY{l+s+sd}{    }
\PY{l+s+sd}{    mass\PYZus{}e\PYZus{}eff\PYZus{}factor = material specific factor to calculate the effective mass from the electron mass}
\PY{l+s+sd}{    \PYZdq{}\PYZdq{}\PYZdq{}}
    
        
    \PY{n}{kT} \PY{o}{=} \PY{n}{calc\PYZus{}kT}\PY{p}{(}\PY{n}{T\PYZus{}C}\PY{p}{)}
    
    \PY{n}{NC} \PY{o}{=} \PY{n}{calc\PYZus{}eff\PYZus{}density\PYZus{}of\PYZus{}states}\PY{p}{(}\PY{n}{T\PYZus{}C}\PY{p}{,}\PY{n}{mass\PYZus{}e\PYZus{}eff\PYZus{}factor}\PY{p}{)}
    
    \PY{n}{ED1C\PYZus{}eV} \PY{o}{=} \PY{l+m+mf}{0.034}
    \PY{n}{ED2C\PYZus{}eV} \PY{o}{=} \PY{l+m+mf}{0.140}
    
    \PY{n}{a} \PY{o}{=} \PY{n}{np}\PY{o}{.}\PY{n}{exp}\PY{p}{(}\PY{n}{CONST}\PY{o}{.}\PY{n}{eV\PYZus{}to\PYZus{}J}\PY{p}{(}\PY{n}{ED1C\PYZus{}eV}\PY{p}{)}\PY{o}{/}\PY{n}{kT}\PY{p}{)}
    \PY{n}{b} \PY{o}{=} \PY{n}{np}\PY{o}{.}\PY{n}{exp}\PY{p}{(}\PY{n}{CONST}\PY{o}{.}\PY{n}{eV\PYZus{}to\PYZus{}J}\PY{p}{(}\PY{n}{ED2C\PYZus{}eV}\PY{p}{)}\PY{o}{/}\PY{n}{kT}\PY{p}{)}
    \PY{n}{t3} \PY{o}{=} \PY{l+m+mf}{1.0}
    \PY{n}{t2} \PY{o}{=} \PY{p}{(}\PY{l+m+mf}{1.0}\PY{o}{/}\PY{n}{b}\PY{o}{\PYZhy{}}\PY{l+m+mf}{0.5}\PY{o}{*}\PY{n}{NC}\PY{o}{/}\PY{n}{ND}\PY{p}{)}
    \PY{n}{t1} \PY{o}{=} \PY{o}{\PYZhy{}}\PY{l+m+mf}{1.0}\PY{o}{/}\PY{n}{b}\PY{o}{*}\PY{n}{NC}\PY{o}{/}\PY{n}{ND}
    \PY{n}{c} \PY{o}{=} \PY{o}{\PYZhy{}}\PY{l+m+mf}{1.0}\PY{o}{/}\PY{p}{(}\PY{l+m+mi}{2}\PY{o}{*}\PY{n}{a}\PY{o}{*}\PY{n}{b}\PY{p}{)}\PY{o}{*}\PY{n}{NC}\PY{o}{/}\PY{n}{ND}

    \PY{n}{poly\PYZus{}params} \PY{o}{=} \PY{p}{(}\PY{n}{c}\PY{p}{,}\PY{n}{t1}\PY{p}{,} \PY{n}{t2}\PY{p}{,} \PY{n}{t3}\PY{p}{)}


    \PY{n}{solutions}\PY{o}{=}\PY{n}{scipy}\PY{o}{.}\PY{n}{roots}\PY{p}{(}\PY{n}{poly\PYZus{}params}\PY{p}{)}
    \PY{n}{EDCFs} \PY{o}{=} \PY{p}{[}\PY{p}{]}
    \PY{k}{for} \PY{n}{sol} \PY{o+ow}{in} \PY{n}{solutions}\PY{p}{:}
        \PY{k}{if} \PY{n}{sol}\PY{o}{.}\PY{n}{imag} \PY{o}{==} \PY{l+m+mi}{0}\PY{p}{:}
            \PY{n}{EDCF} \PY{o}{=} \PY{n}{np}\PY{o}{.}\PY{n}{log}\PY{p}{(}\PY{n}{sol}\PY{o}{.}\PY{n}{real}\PY{p}{)}
            \PY{n}{EDCFs}\PY{o}{.}\PY{n}{append}\PY{p}{(}\PY{o}{\PYZhy{}}\PY{n}{EDCF}\PY{o}{*}\PY{n}{kT}\PY{o}{/}\PY{n}{CONST}\PY{o}{.}\PY{n}{E\PYZus{}CHARGE}\PY{p}{)}
    \PY{k}{if} \PY{n+nb}{len}\PY{p}{(}\PY{n}{EDCFs}\PY{p}{)}\PY{o}{\PYZgt{}}\PY{l+m+mi}{1}\PY{p}{:}
        \PY{k}{raise} \PY{n+ne}{Exception}\PY{p}{(}\PY{l+s+s1}{\PYZsq{}}\PY{l+s+s1}{Should not be...}\PY{l+s+s1}{\PYZsq{}}\PY{p}{)}
    \PY{k}{else}\PY{p}{:}
        \PY{k}{return} \PY{n}{EDCFs}\PY{p}{[}\PY{l+m+mi}{0}\PY{p}{]}

\PY{n}{T\PYZus{}C} \PY{o}{=} \PY{l+m+mi}{300}
\PY{n}{ND} \PY{o}{=} \PY{l+m+mf}{9e21}
\PY{n}{mass\PYZus{}e\PYZus{}eff\PYZus{}factor} \PY{o}{=}\PY{l+m+mf}{0.3}

\PY{n}{EDCF\PYZus{}eV} \PY{o}{=} \PY{n}{calc\PYZus{}EDCF\PYZus{}by\PYZus{}temp}\PY{p}{(}\PY{n}{T\PYZus{}C}\PY{p}{,} \PY{n}{ND}\PY{p}{,} \PY{n}{mass\PYZus{}e\PYZus{}eff\PYZus{}factor}\PY{p}{)}
\PY{n+nb}{print}\PY{p}{(}\PY{n}{EDCF\PYZus{}eV}\PY{p}{)}
\end{Verbatim}
\end{tcolorbox}

    \begin{Verbatim}[commandchars=\\\{\}]
0.3163543980626456
    \end{Verbatim}

    \hypertarget{define-the-smox-material-class}{%
\paragraph{Define the smox-material
class}\label{define-the-smox-material-class}}

With the helper functions a new class for the describing the actual
semiconducting material. This class should be initialized with the
relevant parameters (in the scope of this thesis). Besides this the
\texttt{class\ Material} should hold some methods to calculate relevant
values as the concentration of charge carries in the conduction band for
multiple positions of the conduction band.

From literature the definition of the charge carrier density is known.
It is the integral of the product of density of states \(g_C\) in the
conduction band and occupation probability \(f_F\). Typically the
integral is solved analytically by introducing some simplifications.
This is possible in cases, where the Boltzmann approximation is valid.
In such cases the integrand can be simplified so that the integral can
be solved. By additionally solving the integral numerically the
solutions can be compared. As shown earlier in this thesis, the
numerical solution of the integral by using the \texttt{quad} function
is easy achieve.

    \hypertarget{how-to-derive-the-charge-density-pdf-copy-source-unknown-sorry}{%
\paragraph{How to derive the charge density (PDF copy, source unknown,
sorry)}\label{how-to-derive-the-charge-density-pdf-copy-source-unknown-sorry}}

\begin{figure}
\centering
\includegraphics{DensityOfStates.png}
\caption{title}
\end{figure}

    \begin{tcolorbox}[breakable, size=fbox, boxrule=1pt, pad at break*=1mm,colback=cellbackground, colframe=cellborder]
\prompt{In}{incolor}{77}{\boxspacing}
\begin{Verbatim}[commandchars=\\\{\}]
\PY{k+kn}{from} \PY{n+nn}{scipy}\PY{n+nn}{.}\PY{n+nn}{integrate} \PY{k+kn}{import} \PY{n}{quad}
\PY{k+kn}{from} \PY{n+nn}{scipy}\PY{n+nn}{.}\PY{n+nn}{interpolate} \PY{k+kn}{import} \PY{n}{interp1d}
\PY{k+kn}{import} \PY{n+nn}{scipy}
\PY{k+kn}{from} \PY{n+nn}{functools} \PY{k+kn}{import} \PY{n}{lru\PYZus{}cache}
\PY{k+kn}{import} \PY{n+nn}{numpy} \PY{k}{as} \PY{n+nn}{np}
\PY{k+kn}{import} \PY{n+nn}{pandas} \PY{k}{as} \PY{n+nn}{pd}



\PY{k}{class} \PY{n+nc}{Material}\PY{p}{:}
    \PY{k}{def} \PY{n+nf+fm}{\PYZus{}\PYZus{}init\PYZus{}\PYZus{}}\PY{p}{(}\PY{n+nb+bp}{self}\PY{p}{,}\PY{n}{T\PYZus{}C}\PY{p}{,}\PY{n}{ND} \PY{o}{=} \PY{l+m+mf}{9e21}\PY{p}{,}
                 \PY{n}{EPSILON} \PY{o}{=} \PY{l+m+mf}{9.86}\PY{p}{,} \PY{n}{mass\PYZus{}e\PYZus{}eff\PYZus{}factor} \PY{o}{=} \PY{l+m+mf}{0.3}\PY{p}{,} \PY{n}{DIFF\PYZus{}EF\PYZus{}EC\PYZus{}evolt} \PY{o}{=} \PY{l+m+mf}{0.3}\PY{p}{)}\PY{p}{:}
        \PY{l+s+sd}{\PYZsq{}\PYZsq{}\PYZsq{}}
\PY{l+s+sd}{        T\PYZus{}C = Temperature of the material}
\PY{l+s+sd}{        ND = number of donors per m³}
\PY{l+s+sd}{        DIFF\PYZus{}EF\PYZus{}EC\PYZus{}evolt = E\PYZus{}condution \PYZhy{} E\PYZus{}Fermi }
\PY{l+s+sd}{        \PYZsq{}\PYZsq{}\PYZsq{}}
        \PY{n+nb+bp}{self}\PY{o}{.}\PY{n}{EPSILON} \PY{o}{=} \PY{n}{EPSILON}
        \PY{n+nb+bp}{self}\PY{o}{.}\PY{n}{ND} \PY{o}{=} \PY{n}{ND}
        \PY{n+nb+bp}{self}\PY{o}{.}\PY{n}{MASS\PYZus{}E\PYZus{}EFF} \PY{o}{=} \PY{n}{mass\PYZus{}e\PYZus{}eff\PYZus{}factor}\PY{o}{*}\PY{n}{CONST}\PY{o}{.}\PY{n}{MASS\PYZus{}E}
        \PY{n+nb+bp}{self}\PY{o}{.}\PY{n}{T\PYZus{}C} \PY{o}{=} \PY{n}{T\PYZus{}C}
        \PY{n+nb+bp}{self}\PY{o}{.}\PY{n}{kT} \PY{o}{=} \PY{n}{calc\PYZus{}kT}\PY{p}{(}\PY{n+nb+bp}{self}\PY{o}{.}\PY{n}{T\PYZus{}C}\PY{p}{)}
        \PY{n+nb+bp}{self}\PY{o}{.}\PY{n}{NC} \PY{o}{=} \PY{n}{calc\PYZus{}eff\PYZus{}density\PYZus{}of\PYZus{}states}\PY{p}{(}\PY{n}{T\PYZus{}C}\PY{p}{,}\PY{n}{mass\PYZus{}e\PYZus{}eff\PYZus{}factor}\PY{p}{)}
        

        \PY{k}{if} \PY{n}{DIFF\PYZus{}EF\PYZus{}EC\PYZus{}evolt}\PY{p}{:}
            \PY{n+nb+bp}{self}\PY{o}{.}\PY{n}{Diff\PYZus{}EF\PYZus{}EC\PYZus{}evolt} \PY{o}{=} \PY{n}{DIFF\PYZus{}EF\PYZus{}EC\PYZus{}evolt}
        \PY{k}{else}\PY{p}{:}
            \PY{n+nb+bp}{self}\PY{o}{.}\PY{n}{Diff\PYZus{}EF\PYZus{}EC\PYZus{}evolt} \PY{o}{=} \PY{n}{calc\PYZus{}EDCF\PYZus{}by\PYZus{}temp}\PY{p}{(}\PY{n}{T\PYZus{}C}\PY{p}{,} \PY{n}{ND}\PY{p}{,} \PY{n}{mass\PYZus{}e\PYZus{}eff\PYZus{}factor}\PY{p}{)}
        \PY{n+nb+bp}{self}\PY{o}{.}\PY{n}{Diff\PYZus{}EF\PYZus{}EC} \PY{o}{=} \PY{n}{CONST}\PY{o}{.}\PY{n}{eV\PYZus{}to\PYZus{}J}\PY{p}{(}\PY{n+nb+bp}{self}\PY{o}{.}\PY{n}{Diff\PYZus{}EF\PYZus{}EC\PYZus{}evolt}\PY{p}{)}

        \PY{n+nb+bp}{self}\PY{o}{.}\PY{n}{nb}\PY{p}{,} \PY{n+nb+bp}{self}\PY{o}{.}\PY{n}{nb\PYZus{}err} \PY{o}{=} \PY{n+nb+bp}{self}\PY{o}{.}\PY{n}{n}\PY{p}{(}\PY{l+m+mi}{0}\PY{p}{)}
        \PY{n+nb+bp}{self}\PY{o}{.}\PY{n}{LD} \PY{o}{=} \PY{n}{np}\PY{o}{.}\PY{n}{sqrt}\PY{p}{(}\PY{p}{(}\PY{n+nb+bp}{self}\PY{o}{.}\PY{n}{EPSILON}\PY{o}{*}\PY{n}{CONST}\PY{o}{.}\PY{n}{EPSILON\PYZus{}0}\PY{o}{*}\PY{n+nb+bp}{self}\PY{o}{.}\PY{n}{kT}\PY{p}{)}
                          \PY{o}{/}\PY{p}{(}\PY{n+nb+bp}{self}\PY{o}{.}\PY{n}{nb}\PY{o}{*}\PY{p}{(}\PY{n}{CONST}\PY{o}{.}\PY{n}{E\PYZus{}CHARGE}\PY{o}{*}\PY{o}{*}\PY{l+m+mi}{2}\PY{p}{)}\PY{p}{)}\PY{p}{)}
    
    \PY{k}{def} \PY{n+nf}{J\PYZus{}to\PYZus{}kT}\PY{p}{(}\PY{n+nb+bp}{self}\PY{p}{,}\PY{n}{J}\PY{p}{)}\PY{p}{:}
        \PY{k}{return} \PY{n}{J}\PY{o}{/}\PY{n+nb+bp}{self}\PY{o}{.}\PY{n}{kT}
    
    \PY{k}{def} \PY{n+nf}{kT\PYZus{}to\PYZus{}J}\PY{p}{(}\PY{n+nb+bp}{self}\PY{p}{,}\PY{n}{E\PYZus{}kT}\PY{p}{)}\PY{p}{:}
        \PY{k}{return} \PY{n}{E\PYZus{}kT}\PY{o}{*}\PY{n+nb+bp}{self}\PY{o}{.}\PY{n}{kT}
    
    \PY{k}{def} \PY{n+nf}{densitiy\PYZus{}of\PYZus{}states}\PY{p}{(}\PY{n+nb+bp}{self}\PY{p}{,}\PY{n}{E}\PY{p}{,} \PY{n}{E\PYZus{}c}\PY{p}{)}\PY{p}{:}
        \PY{k}{return} \PY{l+m+mi}{4}\PY{o}{*}\PY{n}{np}\PY{o}{.}\PY{n}{pi}\PY{o}{*}\PY{p}{(}\PY{l+m+mi}{2}\PY{o}{*}\PY{n+nb+bp}{self}\PY{o}{.}\PY{n}{MASS\PYZus{}E\PYZus{}EFF}\PY{p}{)}\PY{o}{*}\PY{o}{*}\PY{p}{(}\PY{l+m+mf}{3.0}\PY{o}{/}\PY{l+m+mf}{2.0}\PY{p}{)}\PY{o}{/}\PY{n}{CONST}\PY{o}{.}\PY{n}{h}\PY{o}{*}\PY{o}{*}\PY{l+m+mi}{3}\PY{o}{*}\PY{p}{(}\PY{n}{E}\PY{o}{\PYZhy{}}\PY{n}{E\PYZus{}c}\PY{p}{)}\PY{o}{*}\PY{o}{*}\PY{l+m+mf}{0.5}
    
    \PY{k}{def} \PY{n+nf}{fermic\PYZus{}dirac}\PY{p}{(}\PY{n+nb+bp}{self}\PY{p}{,}\PY{n}{E\PYZus{}c}\PY{p}{)}\PY{p}{:}
        \PY{l+s+sd}{\PYZsq{}\PYZsq{}\PYZsq{}}
\PY{l+s+sd}{        Calculate the value for the Fermi\PYZhy{}Dirac distribution for an energetic}
\PY{l+s+sd}{        position relative to the material specific conduction band E\PYZus{}c}
\PY{l+s+sd}{        E = E\PYZus{}c+Diff\PYZus{}EF\PYZus{}EC+E\PYZus{}Fermi}
\PY{l+s+sd}{        So the term in the Fermi\PYZhy{}Dirac distribution E\PYZhy{}E\PYZus{}Fermi will become}
\PY{l+s+sd}{        E\PYZus{}c+Diff\PYZus{}EF\PYZus{}EC+E\PYZus{}Fermi\PYZhy{}E\PYZus{}Fermi = E\PYZus{}c+Diff\PYZus{}EF\PYZus{}EC}
\PY{l+s+sd}{        TODO: THIS SHOULD BE IN THE TEXT ABOVE SOMEWHERE}
\PY{l+s+sd}{        \PYZsq{}\PYZsq{}\PYZsq{}}
        \PY{n}{f}\PY{o}{=}\PY{l+m+mf}{1.0}\PY{o}{/}\PY{p}{(}\PY{l+m+mi}{1}\PY{o}{+}\PY{n}{np}\PY{o}{.}\PY{n}{exp}\PY{p}{(}\PY{p}{(}\PY{n}{E\PYZus{}c}\PY{o}{+}\PY{n+nb+bp}{self}\PY{o}{.}\PY{n}{Diff\PYZus{}EF\PYZus{}EC}\PY{p}{)}\PY{o}{/}\PY{n+nb+bp}{self}\PY{o}{.}\PY{n}{kT}\PY{p}{)}\PY{p}{)}
        \PY{k}{return} \PY{n}{f}

    \PY{k}{def} \PY{n+nf}{n\PYZus{}E}\PY{p}{(}\PY{n+nb+bp}{self}\PY{p}{,}\PY{n}{E}\PY{p}{,}\PY{n}{E\PYZus{}c}\PY{p}{)}\PY{p}{:}
        \PY{k}{if} \PY{n}{E}\PY{o}{\PYZlt{}}\PY{n}{E\PYZus{}c}\PY{p}{:}
            \PY{n}{n} \PY{o}{=} \PY{l+m+mi}{0}
        \PY{k}{else}\PY{p}{:}
            \PY{n}{n} \PY{o}{=} \PY{n+nb+bp}{self}\PY{o}{.}\PY{n}{densitiy\PYZus{}of\PYZus{}states}\PY{p}{(}\PY{n}{E}\PY{p}{,} \PY{n}{E\PYZus{}c}\PY{p}{)}\PY{o}{*}\PY{n+nb+bp}{self}\PY{o}{.}\PY{n}{fermic\PYZus{}dirac}\PY{p}{(}\PY{n}{E}\PY{p}{)}
        \PY{k}{return} \PY{n}{n}
                                   
    \PY{n+nd}{@lru\PYZus{}cache}\PY{p}{(}\PY{n}{maxsize}\PY{o}{=}\PY{l+m+mi}{512}\PY{o}{*}\PY{l+m+mi}{512}\PY{o}{*}\PY{l+m+mi}{512}\PY{p}{)}
    \PY{k}{def} \PY{n+nf}{n}\PY{p}{(}\PY{n+nb+bp}{self}\PY{p}{,} \PY{n}{E\PYZus{}c}\PY{p}{)}\PY{p}{:}
        \PY{n}{n}\PY{p}{,} \PY{n}{n\PYZus{}err} \PY{o}{=} \PY{n}{quad}\PY{p}{(}\PY{k}{lambda} \PY{n}{E}\PY{p}{:}\PY{n+nb+bp}{self}\PY{o}{.}\PY{n}{n\PYZus{}E}\PY{p}{(}\PY{n}{E}\PY{p}{,} \PY{n}{E\PYZus{}c}\PY{p}{)}\PY{p}{,}\PY{n}{E\PYZus{}c}\PY{p}{,}\PY{n}{E\PYZus{}c}\PY{o}{+}\PY{n+nb+bp}{self}\PY{o}{.}\PY{n}{kT}\PY{o}{*}\PY{l+m+mi}{100}\PY{p}{)}
        \PY{k}{return} \PY{n}{n}\PY{p}{,} \PY{n}{n\PYZus{}err}

    
\PY{n}{T\PYZus{}C} \PY{o}{=} \PY{l+m+mi}{300}
\PY{n}{ND} \PY{o}{=} \PY{l+m+mf}{9e21}
\PY{n}{mass\PYZus{}e\PYZus{}eff\PYZus{}factor} \PY{o}{=}\PY{l+m+mf}{0.3}

\PY{n}{EDCF\PYZus{}eV} \PY{o}{=} \PY{n}{calc\PYZus{}EDCF\PYZus{}by\PYZus{}temp}\PY{p}{(}\PY{n}{T\PYZus{}C}\PY{p}{,} \PY{n}{ND}\PY{p}{,} \PY{n}{mass\PYZus{}e\PYZus{}eff\PYZus{}factor}\PY{p}{)}

\PY{n+nb}{print}\PY{p}{(}\PY{l+s+sa}{f}\PY{l+s+s1}{\PYZsq{}}\PY{l+s+s1}{For SnO2 at }\PY{l+s+si}{\PYZob{}T\PYZus{}C\PYZcb{}}\PY{l+s+s1}{°C with a defect concentration of }\PY{l+s+si}{\PYZob{}ND\PYZcb{}}\PY{l+s+s1}{ 1/m³, the value of EDCF\PYZus{}eV is }\PY{l+s+si}{\PYZob{}EDCF\PYZus{}eV\PYZcb{}}\PY{l+s+s1}{ eV}\PY{l+s+s1}{\PYZsq{}}\PY{p}{)}    
\PY{n}{material} \PY{o}{=} \PY{n}{Material}\PY{p}{(}\PY{n}{T\PYZus{}C}\PY{p}{,}\PY{n}{DIFF\PYZus{}EF\PYZus{}EC\PYZus{}evolt}\PY{o}{=}\PY{n}{EDCF\PYZus{}eV}\PY{p}{)}
\end{Verbatim}
\end{tcolorbox}

    \begin{Verbatim}[commandchars=\\\{\}]
For SnO2 at 300°C with a defect concentration of 9e+21 1/m³, the value of
EDCF\_eV is 0.3163543980626456 eV
    \end{Verbatim}

    \hypertarget{hint}{%
\paragraph{Hint:}\label{hint}}

\texttt{@lru\_cache(maxsize=512*512*512)} is a decorator for the
function n(self, E\_c).

\begin{quote}
``By definition, a decorator is a function that takes another function
and extends the behavior of the latter function without explicitly
modifying it.'' (https://realpython.com/primer-on-python-decorators/)
\end{quote}

This Python decorator is used to speed up the calculation process. The
lru\_cache (``Last Recently Used'') is used to cache the input and
output of a certain function. As the description of the function says:

\begin{quote}
``It can save time when an expensive or I/O bound function is
periodically called with the same arguments''
\end{quote}

Since in our numerical calc. we will often need to derive the charge
density the \texttt{@lru\_cache} is of great use here. The maxsize
argument in the brackets defines the maximal size of the cache in the
memory of the computer.

    \hypertarget{free-charge-carrier-conc.-using-the-boltzmann-approximation}{%
\paragraph{Free charge carrier conc. using the Boltzmann
approximation}\label{free-charge-carrier-conc.-using-the-boltzmann-approximation}}

Besides the full numerical solution, also the solutions derived from the
Boltzmann approximations need to be defined. As described in
{[}REFERENCE{]} this breaks down to the following functions:

    \begin{tcolorbox}[breakable, size=fbox, boxrule=1pt, pad at break*=1mm,colback=cellbackground, colframe=cellborder]
\prompt{In}{incolor}{78}{\boxspacing}
\begin{Verbatim}[commandchars=\\\{\}]
\PY{k}{def} \PY{n+nf}{boltzmann\PYZus{}acc}\PY{p}{(}\PY{n}{material}\PY{p}{,} \PY{n}{E\PYZus{}c}\PY{p}{)}\PY{p}{:}
    \PY{k}{return} \PY{n}{np}\PY{o}{.}\PY{n}{exp}\PY{p}{(}\PY{o}{\PYZhy{}}\PY{p}{(}\PY{n}{E\PYZus{}c}\PY{o}{+}\PY{n}{material}\PY{o}{.}\PY{n}{Diff\PYZus{}EF\PYZus{}EC}\PY{p}{)}\PY{o}{/}\PY{p}{(}\PY{n}{material}\PY{o}{.}\PY{n}{kT}\PY{o}{*}\PY{l+m+mi}{2}\PY{p}{)}\PY{p}{)}

\PY{k}{def} \PY{n+nf}{boltzmann}\PY{p}{(}\PY{n+nb+bp}{self}\PY{p}{,}\PY{n}{E\PYZus{}c}\PY{p}{)}\PY{p}{:}
    \PY{k}{return} \PY{n}{np}\PY{o}{.}\PY{n}{exp}\PY{p}{(}\PY{o}{\PYZhy{}}\PY{p}{(}\PY{n}{E\PYZus{}c}\PY{o}{+}\PY{n}{material}\PY{o}{.}\PY{n}{Diff\PYZus{}EF\PYZus{}EC}\PY{p}{)}\PY{o}{/}\PY{n}{material}\PY{o}{.}\PY{n}{kT}\PY{p}{)}

\PY{k}{def} \PY{n+nf}{densitiy\PYZus{}of\PYZus{}states}\PY{p}{(}\PY{n+nb+bp}{self}\PY{p}{,}\PY{n}{E}\PY{p}{,} \PY{n}{E\PYZus{}c}\PY{p}{)}\PY{p}{:}

    \PY{k}{return} \PY{l+m+mi}{4}\PY{o}{*}\PY{n}{np}\PY{o}{.}\PY{n}{pi}\PY{o}{*}\PY{p}{(}\PY{l+m+mi}{2}\PY{o}{*}\PY{n}{material}\PY{o}{.}\PY{n}{MASS\PYZus{}E\PYZus{}EFF}\PY{p}{)}\PY{o}{*}\PY{o}{*}\PY{p}{(}\PY{l+m+mf}{3.0}\PY{o}{/}\PY{l+m+mf}{2.0}\PY{p}{)}\PY{o}{/}\PY{n}{CONST}\PY{o}{.}\PY{n}{h}\PY{o}{*}\PY{o}{*}\PY{l+m+mi}{3}\PY{o}{*}\PY{p}{(}\PY{n}{E}\PY{o}{\PYZhy{}}\PY{n}{E\PYZus{}c}\PY{p}{)}\PY{o}{*}\PY{o}{*}\PY{l+m+mf}{0.5}

\PY{k}{def} \PY{n+nf}{n\PYZus{}boltzmann}\PY{p}{(}\PY{n+nb+bp}{self}\PY{p}{,}\PY{n}{E\PYZus{}c}\PY{p}{)}\PY{p}{:}

    \PY{k}{return} \PY{n}{boltzmann}\PY{p}{(}\PY{n}{material}\PY{p}{,}\PY{n}{E\PYZus{}c}\PY{p}{)}\PY{o}{*}\PY{n}{material}\PY{o}{.}\PY{n}{NC}

\PY{k}{def} \PY{n+nf}{n\PYZus{}boltzmann\PYZus{}acc}\PY{p}{(}\PY{n+nb+bp}{self}\PY{p}{,}\PY{n}{E\PYZus{}c}\PY{p}{)}\PY{p}{:}

    \PY{k}{return} \PY{n}{boltzmann\PYZus{}acc}\PY{p}{(}\PY{n}{material}\PY{p}{,}\PY{n}{E\PYZus{}c}\PY{p}{)}\PY{o}{*}\PY{n}{material}\PY{o}{.}\PY{n}{NC}
\end{Verbatim}
\end{tcolorbox}

    \hypertarget{compare-the-numerical-solution-with-the-approximations}{%
\paragraph{Compare the numerical solution with the
approximations}\label{compare-the-numerical-solution-with-the-approximations}}

With all the definitions in place, the different solutions can be
compared. This will be done be representing the charge carrier
concentration \(n\) for the different solutions for multiple positions
of the conduction band \(E_C\) in units of \(kT\).

    \begin{tcolorbox}[breakable, size=fbox, boxrule=1pt, pad at break*=1mm,colback=cellbackground, colframe=cellborder]
\prompt{In}{incolor}{79}{\boxspacing}
\begin{Verbatim}[commandchars=\\\{\}]
\PY{k}{def} \PY{n+nf}{plot\PYZus{}material\PYZus{}char}\PY{p}{(}\PY{n}{mat}\PY{p}{)}\PY{p}{:}
    \PY{n}{ns} \PY{o}{=} \PY{p}{[}\PY{p}{]}
    \PY{n}{n\PYZus{}boltzs} \PY{o}{=} \PY{p}{[}\PY{p}{]}
    \PY{n}{n\PYZus{}boltzs\PYZus{}acc} \PY{o}{=} \PY{p}{[}\PY{p}{]}
    \PY{n}{E\PYZus{}c\PYZus{}kts} \PY{o}{=} \PY{p}{[}\PY{p}{]}
    \PY{k}{for} \PY{n}{i} \PY{o+ow}{in} \PY{n}{np}\PY{o}{.}\PY{n}{linspace}\PY{p}{(}\PY{o}{\PYZhy{}}\PY{l+m+mi}{20}\PY{p}{,}\PY{l+m+mi}{20}\PY{p}{)}\PY{p}{:}
        \PY{n}{E\PYZus{}c} \PY{o}{=} \PY{n}{mat}\PY{o}{.}\PY{n}{kT\PYZus{}to\PYZus{}J}\PY{p}{(}\PY{n}{i}\PY{p}{)}
        \PY{n}{E\PYZus{}c\PYZus{}kts}\PY{o}{.}\PY{n}{append}\PY{p}{(}\PY{n}{i}\PY{p}{)}
        \PY{n}{ns}\PY{o}{.}\PY{n}{append}\PY{p}{(}\PY{n}{mat}\PY{o}{.}\PY{n}{n}\PY{p}{(}\PY{n}{E\PYZus{}c}\PY{p}{)}\PY{p}{[}\PY{l+m+mi}{0}\PY{p}{]}\PY{o}{/}\PY{n}{mat}\PY{o}{.}\PY{n}{nb}\PY{p}{)}
        \PY{n}{n\PYZus{}boltzs}\PY{o}{.}\PY{n}{append}\PY{p}{(}\PY{n}{n\PYZus{}boltzmann}\PY{p}{(}\PY{n}{mat}\PY{p}{,} \PY{n}{E\PYZus{}c}\PY{p}{)}\PY{o}{/}\PY{n}{mat}\PY{o}{.}\PY{n}{nb}\PY{p}{)}
        \PY{n}{n\PYZus{}boltzs\PYZus{}acc}\PY{o}{.}\PY{n}{append}\PY{p}{(}\PY{n}{n\PYZus{}boltzmann\PYZus{}acc}\PY{p}{(}\PY{n}{mat}\PY{p}{,} \PY{n}{E\PYZus{}c}\PY{p}{)}\PY{o}{/}\PY{n}{mat}\PY{o}{.}\PY{n}{nb}\PY{p}{)}
    
    
    \PY{n}{fermi\PYZus{}level\PYZus{}pos\PYZus{}kt} \PY{o}{=} \PY{o}{\PYZhy{}}\PY{n}{mat}\PY{o}{.}\PY{n}{J\PYZus{}to\PYZus{}kT}\PY{p}{(}\PY{n}{mat}\PY{o}{.}\PY{n}{Diff\PYZus{}EF\PYZus{}EC}\PY{p}{)}
    
    \PY{n}{fig}\PY{p}{,} \PY{n}{axe} \PY{o}{=} \PY{n}{subplots}\PY{p}{(}\PY{l+m+mi}{1}\PY{p}{,}\PY{n}{figsize} \PY{o}{=} \PY{p}{(}\PY{l+m+mi}{16}\PY{p}{,}\PY{l+m+mi}{9}\PY{p}{)}\PY{p}{)}

    \PY{n}{axe}\PY{o}{.}\PY{n}{plot}\PY{p}{(}\PY{n}{E\PYZus{}c\PYZus{}kts}\PY{p}{,} \PY{n}{ns}\PY{p}{,} \PY{n}{label}\PY{o}{=}\PY{l+s+s1}{\PYZsq{}}\PY{l+s+s1}{No approx. }\PY{l+s+s1}{\PYZsq{}}\PY{p}{)}
    \PY{n}{axe}\PY{o}{.}\PY{n}{plot}\PY{p}{(}\PY{n}{E\PYZus{}c\PYZus{}kts}\PY{p}{,} \PY{n}{n\PYZus{}boltzs}\PY{p}{,} \PY{l+s+s1}{\PYZsq{}}\PY{l+s+s1}{\PYZhy{}\PYZhy{}}\PY{l+s+s1}{\PYZsq{}}\PY{p}{,} \PY{n}{label}\PY{o}{=}\PY{l+s+s1}{\PYZsq{}}\PY{l+s+s1}{Boltzmann approx.}\PY{l+s+s1}{\PYZsq{}}\PY{p}{)}
    \PY{n}{axe}\PY{o}{.}\PY{n}{plot}\PY{p}{(}\PY{n}{E\PYZus{}c\PYZus{}kts}\PY{p}{,} \PY{n}{n\PYZus{}boltzs\PYZus{}acc}\PY{p}{,} \PY{l+s+s1}{\PYZsq{}}\PY{l+s+s1}{\PYZhy{}.}\PY{l+s+s1}{\PYZsq{}}\PY{p}{,} \PY{n}{label}\PY{o}{=}\PY{l+s+s1}{\PYZsq{}}\PY{l+s+s1}{Accumulation\PYZhy{}Boltzmann approx.}\PY{l+s+s1}{\PYZsq{}}\PY{p}{)}
    \PY{n}{axe}\PY{o}{.}\PY{n}{set\PYZus{}yscale}\PY{p}{(}\PY{l+s+s1}{\PYZsq{}}\PY{l+s+s1}{log}\PY{l+s+s1}{\PYZsq{}}\PY{p}{)}
    \PY{n}{axe}\PY{o}{.}\PY{n}{set\PYZus{}title}\PY{p}{(}\PY{l+s+s1}{\PYZsq{}}\PY{l+s+s1}{\PYZdl{}n(E\PYZus{}C)/n(0)\PYZdl{} as a function of the band bending \PYZdl{}E\PYZus{}C\PYZdl{}}\PY{l+s+s1}{\PYZsq{}}\PY{p}{)}
    \PY{n}{axe}\PY{o}{.}\PY{n}{set\PYZus{}xlabel}\PY{p}{(}\PY{l+s+s1}{\PYZsq{}}\PY{l+s+s1}{Band bending \PYZdl{}E\PYZus{}C\PYZdl{} [kT]}\PY{l+s+s1}{\PYZsq{}}\PY{p}{,} \PY{n}{fontsize}\PY{o}{=}\PY{l+m+mi}{20}\PY{p}{)}
    \PY{n}{axe}\PY{o}{.}\PY{n}{set\PYZus{}ylabel}\PY{p}{(}\PY{l+s+s1}{\PYZsq{}}\PY{l+s+s1}{\PYZdl{}}\PY{l+s+se}{\PYZbs{}\PYZbs{}}\PY{l+s+s1}{frac}\PY{l+s+s1}{\PYZob{}}\PY{l+s+s1}{n(E\PYZus{}C)\PYZcb{}}\PY{l+s+s1}{\PYZob{}}\PY{l+s+s1}{n(0)\PYZcb{}\PYZdl{}}\PY{l+s+s1}{\PYZsq{}}\PY{p}{,} \PY{n}{fontsize}\PY{o}{=}\PY{l+m+mi}{20}\PY{p}{)}
    \PY{n}{axe}\PY{o}{.}\PY{n}{axvline}\PY{p}{(}\PY{n}{fermi\PYZus{}level\PYZus{}pos\PYZus{}kt}\PY{p}{,} \PY{n}{label}\PY{o}{=}\PY{l+s+s1}{\PYZsq{}}\PY{l+s+s1}{Fermi Level}\PY{l+s+s1}{\PYZsq{}}\PY{p}{,} \PY{n}{alpha}\PY{o}{=}\PY{l+m+mf}{0.5}\PY{p}{)}
    \PY{n}{axe}\PY{o}{.}\PY{n}{legend}\PY{p}{(}\PY{p}{)}
    \PY{n}{axe}\PY{o}{.}\PY{n}{grid}\PY{p}{(}\PY{n}{b}\PY{o}{=}\PY{k+kc}{True}\PY{p}{)}
    \PY{k}{return} \PY{n}{fig} 

\PY{n}{fig} \PY{o}{=} \PY{n}{plot\PYZus{}material\PYZus{}char}\PY{p}{(}\PY{n}{material}\PY{p}{)}
\end{Verbatim}
\end{tcolorbox}

    \begin{center}
    \adjustimage{max size={0.9\linewidth}{0.9\paperheight}}{output_32_0.png}
    \end{center}
    { \hspace*{\fill} \\}
    
    The numerical solution and the approximations are inline with each other
in their specific regions of \(E_C\). The Boltzmann approximation is
identical to the numerical solution starting \textasciitilde3kT above
the Fermi energy level. The approximation for the accumulation layer has
its validity in a region where an accumulation is present. This was
already shown in \cite{Barsan2011a}.

    \hypertarget{numerical-description-of-the-semiconductor-grains}{%
\subsubsection{Numerical description of the semiconductor
grains}\label{numerical-description-of-the-semiconductor-grains}}

In this section we define a SMOX grain. We approximate the grain as a
sphere composed out of a material we previously defined. For one grain,
the Poisson equation with spherical symmetry is solved. The transfer of
the findings from a material to an actual grain are important for
multiple reasons. On the one side the ratio of the available surface
sites to react with the semiconductor and its bulk size play an
important role. Very small grains may have relatively high concentration
of surface sites but lack of electrons needed for the reaction a t the
surface. So a grain may get fully depleted which may have significant an
influence on the overall conduction. On the other side, the conduction
path through the grain differers depending on the free charge carrier
concentration. Those two relevant properties can only be analyzed, if
the transfer from a material to an actual grain is solved.

To solve the Poisson equation, we will need to supply the solver with
the initial values. Tow values need to be supplies. The goal of the
solver is to find the shape of the conduction band inside the grain.
This depends on the position at the surface. This value can
experimentally be measured with the Kelvin Probe method. The second
start parameter, which needs to be supplied is the slope of the curve at
the surface. With these two parameters, the solver iterates from the
starting condition stepwise though the grain and calculates for each
step new values based on the previous iteration.

This ``inital value problem'' is solved with the scipy tool solve\_ivp.

    \begin{tcolorbox}[breakable, size=fbox, boxrule=1pt, pad at break*=1mm,colback=cellbackground, colframe=cellborder]
\prompt{In}{incolor}{80}{\boxspacing}
\begin{Verbatim}[commandchars=\\\{\}]
\PY{k+kn}{from} \PY{n+nn}{scipy}\PY{n+nn}{.}\PY{n+nn}{integrate} \PY{k+kn}{import} \PY{n}{solve\PYZus{}ivp}
\PY{k}{class} \PY{n+nc}{Grain}\PY{p}{:}
    \PY{k}{def} \PY{n+nf+fm}{\PYZus{}\PYZus{}init\PYZus{}\PYZus{}}\PY{p}{(}\PY{n+nb+bp}{self}\PY{p}{,}\PY{n}{grainsize\PYZus{}radius}\PY{p}{,}\PY{n}{material}\PY{p}{,}\PY{n}{rPoints}\PY{o}{=}\PY{l+m+mi}{100}\PY{p}{)}\PY{p}{:}
        \PY{n+nb+bp}{self}\PY{o}{.}\PY{n}{R} \PY{o}{=} \PY{n}{grainsize\PYZus{}radius}
        \PY{n+nb+bp}{self}\PY{o}{.}\PY{n}{material} \PY{o}{=} \PY{n}{material}
        \PY{n+nb+bp}{self}\PY{o}{.}\PY{n}{rs} \PY{o}{=} \PY{n+nb+bp}{self}\PY{o}{.}\PY{n}{R}\PY{o}{*}\PY{p}{(}\PY{l+m+mf}{1.0}\PY{o}{\PYZhy{}}\PY{n}{np}\PY{o}{.}\PY{n}{logspace}\PY{p}{(}\PY{l+m+mi}{0}\PY{p}{,}\PY{l+m+mi}{3}\PY{p}{,}\PY{n}{num}\PY{o}{=}\PY{n}{rPoints}\PY{p}{)}\PY{o}{/}\PY{l+m+mf}{1e3}\PY{p}{)} \PY{c+c1}{\PYZsh{}calcualtion points from surface to center (not lnear spaced)}
        \PY{n+nb+bp}{self}\PY{o}{.}\PY{n}{rs} \PY{o}{=} \PY{n}{np}\PY{o}{.}\PY{n}{linspace}\PY{p}{(}\PY{n+nb+bp}{self}\PY{o}{.}\PY{n}{R}\PY{o}{/}\PY{l+m+mi}{1000}\PY{p}{,} \PY{n+nb+bp}{self}\PY{o}{.}\PY{n}{R}\PY{p}{,} \PY{l+m+mi}{1000}\PY{p}{)}
    
  
    \PY{k}{def} \PY{n+nf}{solve\PYZus{}with\PYZus{}values}\PY{p}{(}\PY{n+nb+bp}{self}\PY{p}{,}\PY{n}{E\PYZus{}init}\PY{p}{,} \PY{n}{E\PYZus{}dot\PYZus{}init}\PY{p}{)}\PY{p}{:}
        \PY{n}{r} \PY{o}{=} \PY{n+nb+bp}{self}\PY{o}{.}\PY{n}{rs}\PY{o}{/}\PY{n+nb+bp}{self}\PY{o}{.}\PY{n}{material}\PY{o}{.}\PY{n}{LD}
        \PY{n}{E\PYZus{}init\PYZus{}kT} \PY{o}{=} \PY{n+nb+bp}{self}\PY{o}{.}\PY{n}{material}\PY{o}{.}\PY{n}{J\PYZus{}to\PYZus{}kT}\PY{p}{(}\PY{n}{E\PYZus{}init}\PY{p}{)}
        \PY{n}{E\PYZus{}dot\PYZus{}init\PYZus{}kt} \PY{o}{=} \PY{n+nb+bp}{self}\PY{o}{.}\PY{n}{material}\PY{o}{.}\PY{n}{J\PYZus{}to\PYZus{}kT}\PY{p}{(}\PY{n}{E\PYZus{}dot\PYZus{}init}\PY{p}{)}

        \PY{c+c1}{\PYZsh{}the solver should stop, when the slope is zero. This is reasonable since if the slope is zero, this should be the lowest point of the graph}
        \PY{c+c1}{\PYZsh{}so, when we \PYZdq{}hit\PYZus{}ground\PYZdq{} the solver should stop, to save some computational time}
        \PY{k}{def} \PY{n+nf}{hit\PYZus{}ground}\PY{p}{(}\PY{n}{t}\PY{p}{,} \PY{n}{y}\PY{p}{)}\PY{p}{:} \PY{k}{return} \PY{n}{y}\PY{p}{[}\PY{l+m+mi}{1}\PY{p}{]}
        \PY{n}{hit\PYZus{}ground}\PY{o}{.}\PY{n}{terminal} \PY{o}{=} \PY{k+kc}{True}
        
        \PY{c+c1}{\PYZsh{}this is the solver}
        \PY{n}{data} \PY{o}{=} \PY{n}{solve\PYZus{}ivp}\PY{p}{(}\PY{n+nb+bp}{self}\PY{o}{.}\PY{n}{deriv\PYZus{}E\PYZus{}E\PYZus{}dot}\PY{p}{,}\PY{p}{(}\PY{n}{r}\PY{p}{[}\PY{o}{\PYZhy{}}\PY{l+m+mi}{1}\PY{p}{]}\PY{p}{,}\PY{n}{r}\PY{p}{[}\PY{l+m+mi}{0}\PY{p}{]}\PY{p}{)}\PY{p}{,}  \PY{p}{[}\PY{n}{E\PYZus{}init\PYZus{}kT}\PY{p}{,}\PY{n}{E\PYZus{}dot\PYZus{}init\PYZus{}kt}\PY{p}{]}\PY{p}{,} \PY{n}{events}\PY{o}{=}\PY{n}{hit\PYZus{}ground}\PY{p}{,} \PY{n}{method} \PY{o}{=} \PY{l+s+s1}{\PYZsq{}}\PY{l+s+s1}{Radau}\PY{l+s+s1}{\PYZsq{}}\PY{p}{,}\PY{n}{max\PYZus{}step}\PY{o}{=}\PY{n+nb}{max}\PY{p}{(}\PY{n}{r}\PY{p}{)}\PY{o}{/}\PY{l+m+mi}{1000}\PY{p}{)}
        
        \PY{c+c1}{\PYZsh{}since we start the iteration to solve the equation from the outside, the results have to be revered }
        \PY{n}{r} \PY{o}{=} \PY{n}{data}\PY{o}{.}\PY{n}{t}\PY{p}{[}\PY{p}{:}\PY{p}{:}\PY{o}{\PYZhy{}}\PY{l+m+mi}{1}\PY{p}{]}
        \PY{n}{v} \PY{o}{=} \PY{n}{data}\PY{o}{.}\PY{n}{y}\PY{p}{[}\PY{l+m+mi}{0}\PY{p}{]}\PY{p}{[}\PY{p}{:}\PY{p}{:}\PY{o}{\PYZhy{}}\PY{l+m+mi}{1}\PY{p}{]}
        \PY{n}{v\PYZus{}dot} \PY{o}{=} \PY{n}{data}\PY{o}{.}\PY{n}{y}\PY{p}{[}\PY{l+m+mi}{1}\PY{p}{]}\PY{p}{[}\PY{p}{:}\PY{p}{:}\PY{o}{\PYZhy{}}\PY{l+m+mi}{1}\PY{p}{]}

        \PY{k}{return} \PY{n}{r}\PY{p}{,}\PY{n}{v}\PY{p}{,} \PY{n}{v\PYZus{}dot}\PY{p}{,} \PY{n}{data}


    \PY{k}{def} \PY{n+nf}{deriv\PYZus{}E\PYZus{}E\PYZus{}dot}\PY{p}{(}\PY{n+nb+bp}{self}\PY{p}{,}\PY{n}{r\PYZus{}}\PY{p}{,} \PY{n}{U\PYZus{}U\PYZus{}dot}\PY{p}{)}\PY{p}{:}
        \PY{n}{U} \PY{o}{=} \PY{n}{U\PYZus{}U\PYZus{}dot}\PY{p}{[}\PY{l+m+mi}{0}\PY{p}{]}
        \PY{n}{U\PYZus{}dot} \PY{o}{=} \PY{n}{U\PYZus{}U\PYZus{}dot}\PY{p}{[}\PY{l+m+mi}{1}\PY{p}{]}
        \PY{n}{E} \PY{o}{=} \PY{n+nb+bp}{self}\PY{o}{.}\PY{n}{material}\PY{o}{.}\PY{n}{kT\PYZus{}to\PYZus{}J}\PY{p}{(}\PY{n}{U}\PY{p}{)}
        \PY{n}{n} \PY{o}{=} \PY{n+nb+bp}{self}\PY{o}{.}\PY{n}{material}\PY{o}{.}\PY{n}{n}\PY{p}{(}\PY{n}{E}\PY{p}{)}
        \PY{n}{U\PYZus{}dot\PYZus{}dot} \PY{o}{=} \PY{l+m+mi}{1}\PY{o}{\PYZhy{}}\PY{n}{n}\PY{p}{[}\PY{l+m+mi}{0}\PY{p}{]}\PY{o}{/}\PY{n+nb+bp}{self}\PY{o}{.}\PY{n}{material}\PY{o}{.}\PY{n}{nb} \PY{o}{\PYZhy{}}\PY{l+m+mi}{2}\PY{o}{/}\PY{n}{r\PYZus{}}\PY{o}{*}\PY{n}{U\PYZus{}dot}
        \PY{k}{return} \PY{p}{[}\PY{n}{U\PYZus{}dot}\PY{p}{,} \PY{n}{U\PYZus{}dot\PYZus{}dot}\PY{p}{]}

\PY{n}{grain} \PY{o}{=} \PY{n}{Grain}\PY{p}{(}\PY{l+m+mf}{100e\PYZhy{}9}\PY{p}{,}\PY{n}{material}\PY{p}{)}
\end{Verbatim}
\end{tcolorbox}

    In the following graph an example is given, where the initial value of
\(V^*\) is fixed, an different values of the surface slope are used to
solve the equation.

In the previous cells we already initialized the \texttt{material} and
\texttt{grain} object with reasonable parameters. First we initialize
the grain with a material and calculate the solutions of the Poisson
equation for multiple start parameters.

    \begin{tcolorbox}[breakable, size=fbox, boxrule=1pt, pad at break*=1mm,colback=cellbackground, colframe=cellborder]
\prompt{In}{incolor}{21}{\boxspacing}
\begin{Verbatim}[commandchars=\\\{\}]
\PY{n}{fig}\PY{p}{,} \PY{n}{axe} \PY{o}{=} \PY{n}{subplots}\PY{p}{(}\PY{n}{figsize} \PY{o}{=} \PY{p}{(}\PY{l+m+mi}{16}\PY{p}{,}\PY{l+m+mi}{9}\PY{p}{)}\PY{p}{)}
\PY{n}{axe}\PY{o}{.}\PY{n}{set\PYZus{}ylim}\PY{p}{(}\PY{o}{\PYZhy{}}\PY{l+m+mi}{8}\PY{p}{,}\PY{l+m+mi}{8}\PY{p}{)}

\PY{n}{E\PYZus{}init} \PY{o}{=} \PY{n}{grain}\PY{o}{.}\PY{n}{material}\PY{o}{.}\PY{n}{kT\PYZus{}to\PYZus{}J}\PY{p}{(}\PY{l+m+mi}{8}\PY{p}{)}
\PY{k}{for} \PY{n}{E\PYZus{}dot\PYZus{}init\PYZus{}kt} \PY{o+ow}{in} \PY{p}{[}\PY{l+m+mf}{0.8}\PY{p}{,}\PY{l+m+mf}{0.9}\PY{p}{,}\PY{l+m+mi}{1}\PY{p}{]}\PY{p}{:}
    \PY{n}{E\PYZus{}dot\PYZus{}init} \PY{o}{=} \PY{n}{grain}\PY{o}{.}\PY{n}{material}\PY{o}{.}\PY{n}{kT\PYZus{}to\PYZus{}J}\PY{p}{(}\PY{n}{E\PYZus{}dot\PYZus{}init\PYZus{}kt}\PY{p}{)}
    \PY{n}{r}\PY{p}{,}\PY{n}{v}\PY{p}{,} \PY{n}{v\PYZus{}dot}\PY{p}{,} \PY{n}{data} \PY{o}{=} \PY{n}{grain}\PY{o}{.}\PY{n}{solve\PYZus{}with\PYZus{}values}\PY{p}{(}\PY{n}{E\PYZus{}init}\PY{p}{,} \PY{n}{E\PYZus{}dot\PYZus{}init}\PY{p}{)}


    \PY{n}{axe}\PY{o}{.}\PY{n}{plot}\PY{p}{(}\PY{n}{r}\PY{p}{,}\PY{n}{v}\PY{p}{,} \PY{n}{label}\PY{o}{=}\PY{l+s+s1}{\PYZsq{}}\PY{l+s+s1}{Initial slope: E\PYZus{}dot\PYZus{}init\PYZus{}kt [kT/\PYZdl{}L\PYZus{}D\PYZdl{}])}
    \PY{n}{axe}\PY{o}{.}\PY{n}{set\PYZus{}xlabel}
\PY{n}{axe}\PY{o}{.}\PY{n}{set\PYZus{}ylabel}\PY{p}{(}\PY{l+s+s1}{\PYZsq{}}\PY{l+s+s1}{\PYZdl{}E\PYZus{}C(r\PYZca{}*)\PYZdl{} [\PYZdl{}k\PYZus{}BT\PYZdl{}]}\PY{l+s+s1}{\PYZsq{}}\PY{p}{,} \PY{n}{fontsize} \PY{o}{=}\PY{l+m+mi}{22}\PY{p}{)}
\PY{n}{axe}\PY{o}{.}\PY{n}{set\PYZus{}xlabel}\PY{p}{(}\PY{l+s+sa}{f}\PY{l+s+s1}{\PYZsq{}}\PY{l+s+s1}{Position in grain [\PYZdl{}L\PYZus{}D\PYZdl{} = }\PY{l+s+s1}{\PYZob{}}\PY{l+s+s1}{grain.material.LD*1e9:.2f\PYZcb{}nm], R=}\PY{l+s+s1}{\PYZob{}}\PY{l+s+s1}{grain.R*1e9:.2f\PYZcb{}nm}\PY{l+s+s1}{\PYZsq{}}\PY{p}{,} \PY{n}{fontsize} \PY{o}{=}\PY{l+m+mi}{22}\PY{p}{)}
\PY{n}{leg} \PY{o}{=} \PY{n}{axe}\PY{o}{.}\PY{n}{legend}\PY{p}{(}\PY{p}{)}

    
\end{Verbatim}
\end{tcolorbox}

    \begin{center}
    \adjustimage{max size={0.9\linewidth}{0.9\paperheight}}{output_37_0.png}
    \end{center}
    { \hspace*{\fill} \\}
    
    From this graph it is obvious, that the initial slope has a major
influence on the result.

Unfortunately this value is not known. On the other side, I can relay on
an equation to check my solution. We will use the derived equation
(\ref{second_derivative}) from the Poisson equation:

\begin{align}
\frac{d²V^{*}}{dr^{*2}}=1-n^{*}(V^{*})-\frac{2}{r^{*}}\frac{dV^{*}}{dr^{*}}
\end{align}

This equation can be transformed into following form:

\begin{align}
\intop_{0}^{R}\frac{d²V^{*}}{dr^{*2}}dr^*=\left[\frac{dV^{*}}{dr^{*}}\right]_{0}^{R}=\left.\frac{dV^*}{dr^{*}}\right|_{R}=\intop_{0}^{R}1-n^{*}(V^{*})-\frac{2}{r^{*}}\frac{dV^{*}}{dr^{*}}dr^*\label{second_derivative}\tag{Surface slope}
\end{align}

With this relation we can check each solution. For each starting
condition the solution should should also be valid for this equation.
The right side will again be evaluated numerically. The expression
\(n^{*}(V^{*})\) can be calculated for each \(V^{*}\) with the function
defined in \texttt{class\ material}. From the solver of the differential
equation \(\frac{dV^*}{dr^{*}}\) is known inside the grain for each
\(r^*\). Since all the elements of the integral are known, the numerical
evaluation is not difficult. Since the elements of the integral in this
case are not functions, which can be calculated individually for each
point, but rather a list of values, the integration is slightly
different. For the numerical integration of a list of values \(y\)
corresponding to a set of \(x\) values, the \texttt{numpy} function
\texttt{trapz} is used:

\begin{quote}
numpy.trapz(y, x=None, dx=1.0, axis=-1)

Integrate along the given axis using the composite trapezoidal rule.
\end{quote}

    \hypertarget{example-usage-of-np.trapz}{%
\paragraph{\texorpdfstring{Example usage of
\texttt{np.trapz}}{Example usage of np.trapz}}\label{example-usage-of-np.trapz}}

    \begin{tcolorbox}[breakable, size=fbox, boxrule=1pt, pad at break*=1mm,colback=cellbackground, colframe=cellborder]
\prompt{In}{incolor}{84}{\boxspacing}
\begin{Verbatim}[commandchars=\\\{\}]
\PY{n}{x} \PY{o}{=} \PY{p}{[}\PY{l+m+mi}{0}\PY{p}{,}\PY{l+m+mi}{1}\PY{p}{,}\PY{l+m+mi}{1}\PY{p}{,}\PY{l+m+mi}{2}\PY{p}{,}\PY{l+m+mi}{3}\PY{p}{,}\PY{l+m+mi}{4}\PY{p}{,}\PY{l+m+mi}{5}\PY{p}{,}\PY{l+m+mi}{5}\PY{p}{,}\PY{l+m+mi}{6}\PY{p}{]}
\PY{n}{y} \PY{o}{=} \PY{p}{[}\PY{l+m+mi}{0}\PY{p}{,}\PY{l+m+mi}{0}\PY{p}{,}\PY{l+m+mi}{1}\PY{p}{,}\PY{l+m+mi}{1}\PY{p}{,}\PY{l+m+mi}{1}\PY{p}{,}\PY{l+m+mi}{1}\PY{p}{,}\PY{l+m+mi}{1}\PY{p}{,}\PY{l+m+mi}{0}\PY{p}{,}\PY{l+m+mi}{0}\PY{p}{]}
\PY{n}{fig}\PY{p}{,} \PY{n}{axe} \PY{o}{=} \PY{n}{subplots}\PY{p}{(}\PY{p}{)}
\PY{n}{axe}\PY{o}{.}\PY{n}{plot}\PY{p}{(}\PY{n}{x}\PY{p}{,}\PY{n}{y}\PY{p}{)}
\PY{n}{axe}\PY{o}{.}\PY{n}{set\PYZus{}xlabel}\PY{p}{(}\PY{l+s+s1}{\PYZsq{}}\PY{l+s+s1}{X}\PY{l+s+s1}{\PYZsq{}}\PY{p}{)}
\PY{n}{axe}\PY{o}{.}\PY{n}{set\PYZus{}ylabel}\PY{p}{(}\PY{l+s+s1}{\PYZsq{}}\PY{l+s+s1}{Y}\PY{l+s+s1}{\PYZsq{}}\PY{p}{)}
\PY{n}{numerical\PYZus{}integral} \PY{o}{=} \PY{n}{np}\PY{o}{.}\PY{n}{trapz}\PY{p}{(}\PY{n}{y}\PY{p}{,}\PY{n}{x}\PY{p}{)}
\PY{n}{axe}\PY{o}{.}\PY{n}{text}\PY{p}{(}\PY{l+m+mi}{3}\PY{p}{,} \PY{l+m+mf}{0.5}\PY{p}{,} \PY{l+s+sa}{f}\PY{l+s+s1}{\PYZsq{}}\PY{l+s+s1}{Integral: }\PY{l+s+si}{\PYZob{}numerical\PYZus{}integral\PYZcb{}}\PY{l+s+s1}{\PYZsq{}}\PY{p}{)}
\end{Verbatim}
\end{tcolorbox}

            \begin{tcolorbox}[breakable, size=fbox, boxrule=.5pt, pad at break*=1mm, opacityfill=0]
\prompt{Out}{outcolor}{84}{\boxspacing}
\begin{Verbatim}[commandchars=\\\{\}]
Text(3, 0.5, 'Integral: 4.0')
\end{Verbatim}
\end{tcolorbox}
        
    \begin{center}
    \adjustimage{max size={0.9\linewidth}{0.9\paperheight}}{output_40_1.png}
    \end{center}
    { \hspace*{\fill} \\}
    
    From the latter relation between the inital slope a the surface and
charge distribution on the center (\(\ref{second_derivative}\)) the
following boundary condition for the solution is defined: \begin{align}
\left.\frac{dV^*}{dr^{*}}\right|_{R^*}-\intop_{0}^{R^*}\left(1-n^{*}(V^{*})-\frac{2}{r^{*}}\frac{dV^{*}}{dr^{*}}\right)dr^* = 0.
\end{align}

Tle left side of the equation can be calculated for multiple values of
\(\left.\frac{dV^*}{dr^{*}}\right|_{R^*}\). The right value of
\(\left.\frac{dV^*}{dr^{*}}\right|_{R^*}\) needs to be found to minimize
the left side of the equation. Definitely similar problems have been
done before, Python/SciPy has already a solution for this ready. The
tools needed to solve this problems can be found in the
\texttt{scipy.optimize} package. The function \texttt{minimize\_scalar}
will be used to minimize the left side of the equation by varying the
scalar parameter \(\left.\frac{dV^*}{dr^{*}}\right|_{R^*}\).

The following line is used to load the required function:
\texttt{from\ scipy.optimize\ import\ minimize\_scalar}

To use this function, function needs to be defined, which is then
minimized by changing the input parameter. This function, which takes
the inital slope as an argument, returns the ``error'' based on the
previous equation. The algorithm should then find the best initial slope
parameter to have a valid solution.

    \begin{tcolorbox}[breakable, size=fbox, boxrule=1pt, pad at break*=1mm,colback=cellbackground, colframe=cellborder]
\prompt{In}{incolor}{23}{\boxspacing}
\begin{Verbatim}[commandchars=\\\{\}]
\PY{k+kn}{from} \PY{n+nn}{scipy}\PY{n+nn}{.}\PY{n+nn}{optimize} \PY{k+kn}{import} \PY{n}{minimize\PYZus{}scalar}


\PY{k}{def} \PY{n+nf}{min\PYZus{}vdot}\PY{p}{(}\PY{n}{vdot\PYZus{}init}\PY{p}{,} \PY{n}{grain}\PY{p}{,} \PY{n}{vinit}\PY{p}{)}\PY{p}{:}
    \PY{n}{r}\PY{p}{,}\PY{n}{v}\PY{p}{,}\PY{n}{vdot}\PY{p}{,} \PY{n}{data} \PY{o}{=} \PY{n}{grain}\PY{o}{.}\PY{n}{solve\PYZus{}with\PYZus{}values}\PY{p}{(}\PY{n}{grain}\PY{o}{.}\PY{n}{material}\PY{o}{.}\PY{n}{kT\PYZus{}to\PYZus{}J}\PY{p}{(}\PY{n}{vinit}\PY{p}{)}\PY{p}{,}\PY{n}{grain}\PY{o}{.}\PY{n}{material}\PY{o}{.}\PY{n}{kT\PYZus{}to\PYZus{}J}\PY{p}{(}\PY{n}{vdot\PYZus{}init}\PY{p}{)}\PY{p}{)}
    \PY{c+c1}{\PYZsh{}the integration stops when the slope is zero. This might not be in the center (r*=0) of the grain. So we will expand the datapoints appropriatly}
    \PY{c+c1}{\PYZsh{}for numerical reasons, we will not choose r*=0, but rather a very small r*.}
    \PY{n}{r} \PY{o}{=} \PY{n}{np}\PY{o}{.}\PY{n}{concatenate}\PY{p}{(}\PY{p}{[}\PY{p}{[}\PY{l+m+mf}{0.01}\PY{p}{]}\PY{p}{,}\PY{n}{r}\PY{p}{]}\PY{p}{,} \PY{n}{axis}\PY{o}{=}\PY{l+m+mi}{0}\PY{p}{)}
    \PY{c+c1}{\PYZsh{}If the slope is zero, the slope should stay zero until the center}
    \PY{n}{vdot} \PY{o}{=} \PY{n}{np}\PY{o}{.}\PY{n}{concatenate}\PY{p}{(}\PY{p}{[}\PY{p}{[}\PY{l+m+mi}{0}\PY{p}{]}\PY{p}{,}\PY{n}{vdot}\PY{p}{]}\PY{p}{,} \PY{n}{axis}\PY{o}{=}\PY{l+m+mi}{0}\PY{p}{)}
    
    \PY{c+c1}{\PYZsh{}and since the slope is zero, also the value should not change anymore}
    \PY{n}{v} \PY{o}{=} \PY{n}{np}\PY{o}{.}\PY{n}{concatenate}\PY{p}{(}\PY{p}{[}\PY{p}{[}\PY{n}{v}\PY{p}{[}\PY{l+m+mi}{0}\PY{p}{]}\PY{p}{]}\PY{p}{,}\PY{n}{v}\PY{p}{]}\PY{p}{,} \PY{n}{axis}\PY{o}{=}\PY{l+m+mi}{0}\PY{p}{)}
    
    \PY{c+c1}{\PYZsh{}for each point of the solution the element in the integral is calculated}
    \PY{n}{integrand} \PY{o}{=} \PY{p}{[}\PY{p}{(}\PY{l+m+mi}{1}\PY{o}{\PYZhy{}}\PY{n}{grain}\PY{o}{.}\PY{n}{material}\PY{o}{.}\PY{n}{n}\PY{p}{(}\PY{n}{grain}\PY{o}{.}\PY{n}{material}\PY{o}{.}\PY{n}{kT\PYZus{}to\PYZus{}J}\PY{p}{(}\PY{n}{v\PYZus{}i}\PY{p}{)}\PY{p}{)}\PY{p}{[}\PY{l+m+mi}{0}\PY{p}{]}\PY{o}{/}\PY{n}{grain}\PY{o}{.}\PY{n}{material}\PY{o}{.}\PY{n}{nb}\PY{p}{)}\PY{o}{\PYZhy{}}\PY{l+m+mi}{2}\PY{o}{/}\PY{n}{r\PYZus{}i}\PY{o}{*}\PY{n}{vdot\PYZus{}i} \PY{k}{for} \PY{n}{r\PYZus{}i}\PY{p}{,} \PY{n}{v\PYZus{}i}\PY{p}{,} \PY{n}{vdot\PYZus{}i} \PY{o+ow}{in} \PY{n+nb}{zip}\PY{p}{(}\PY{n}{r}\PY{p}{,} \PY{n}{v}\PY{p}{,} \PY{n}{vdot}\PY{p}{)}\PY{p}{]}
    
    \PY{c+c1}{\PYZsh{}the integral is numerically calculated}
    \PY{n}{dV} \PY{o}{=} \PY{n}{np}\PY{o}{.}\PY{n}{trapz}\PY{p}{(}\PY{n}{y}\PY{o}{=}\PY{n}{integrand}\PY{p}{,}\PY{n}{x}\PY{o}{=}\PY{n}{r}\PY{p}{)}
    
    \PY{c+c1}{\PYZsh{}The integral should be the same as the slope at the surface, the differenc is the error to be minimized}
    \PY{n}{res} \PY{o}{=} \PY{n+nb}{abs}\PY{p}{(}\PY{n}{dV}\PY{o}{\PYZhy{}}\PY{n}{vdot}\PY{p}{[}\PY{o}{\PYZhy{}}\PY{l+m+mi}{1}\PY{p}{]}\PY{p}{)}
    
    \PY{c+c1}{\PYZsh{}Debug output}
    \PY{c+c1}{\PYZsh{}print(res, vdot[\PYZhy{}1])}
    
    \PY{c+c1}{\PYZsh{}return the \PYZsq{}error\PYZsq{} to be minimized}
    \PY{k}{return} \PY{n}{res}

\PY{k}{def} \PY{n+nf}{find\PYZus{}best\PYZus{}v\PYZus{}dot\PYZus{}init}\PY{p}{(}\PY{n}{E\PYZus{}init\PYZus{}kT}\PY{p}{,} \PY{n}{grain}\PY{p}{)}\PY{p}{:}
    \PY{c+c1}{\PYZsh{}find for a given inital value E\PYZus{}init\PYZus{}kT, the correspoding E\PYZus{}init\PYZus{}dot\PYZus{}kT}
    \PY{n}{bounds} \PY{o}{=} \PY{n+nb}{sorted}\PY{p}{(}\PY{p}{[}\PY{n}{E\PYZus{}init\PYZus{}kT}\PY{o}{*}\PY{l+m+mi}{10}\PY{p}{,} \PY{l+m+mi}{0}\PY{p}{]}\PY{p}{)}
    \PY{n}{res} \PY{o}{=} \PY{n}{minimize\PYZus{}scalar}\PY{p}{(}\PY{n}{min\PYZus{}vdot}\PY{p}{,}  \PY{n}{bounds}\PY{o}{=}\PY{n}{bounds}\PY{p}{,} \PY{n}{tol}\PY{o}{=}\PY{n+nb}{abs}\PY{p}{(}\PY{n}{E\PYZus{}init\PYZus{}kT}\PY{o}{/}\PY{l+m+mi}{10000}\PY{p}{)}\PY{p}{,} \PY{n}{method}\PY{o}{=}\PY{l+s+s1}{\PYZsq{}}\PY{l+s+s1}{Bounded}\PY{l+s+s1}{\PYZsq{}}\PY{p}{,} \PY{n}{args}\PY{o}{=}\PY{p}{(}\PY{n}{grain}\PY{p}{,} \PY{n}{E\PYZus{}init\PYZus{}kT}\PY{p}{)}\PY{p}{)}
    \PY{k}{return} \PY{n}{res}

\PY{n}{res} \PY{o}{=} \PY{n}{find\PYZus{}best\PYZus{}v\PYZus{}dot\PYZus{}init}\PY{p}{(}\PY{l+m+mi}{8}\PY{p}{,} \PY{n}{grain}\PY{p}{)}
\end{Verbatim}
\end{tcolorbox}

    So the previous example with randomly guessed initial values can be
extended with a better guess.

    \begin{tcolorbox}[breakable, size=fbox, boxrule=1pt, pad at break*=1mm,colback=cellbackground, colframe=cellborder]
\prompt{In}{incolor}{87}{\boxspacing}
\begin{Verbatim}[commandchars=\\\{\}]
\PY{n}{fig}\PY{p}{,} \PY{n}{axe} \PY{o}{=} \PY{n}{subplots}\PY{p}{(}\PY{n}{figsize} \PY{o}{=} \PY{p}{(}\PY{l+m+mi}{16}\PY{p}{,}\PY{l+m+mi}{9}\PY{p}{)}\PY{p}{)}
\PY{n}{axe}\PY{o}{.}\PY{n}{set\PYZus{}ylim}\PY{p}{(}\PY{o}{\PYZhy{}}\PY{l+m+mi}{8}\PY{p}{,}\PY{l+m+mi}{8}\PY{p}{)}
\PY{n}{E\PYZus{}init\PYZus{}kT} \PY{o}{=} \PY{l+m+mi}{8}
\PY{n}{E\PYZus{}init} \PY{o}{=} \PY{n}{grain}\PY{o}{.}\PY{n}{material}\PY{o}{.}\PY{n}{kT\PYZus{}to\PYZus{}J}\PY{p}{(}\PY{n}{E\PYZus{}init\PYZus{}kT}\PY{p}{)}
\PY{k}{for} \PY{n}{E\PYZus{}dot\PYZus{}init\PYZus{}kT} \PY{o+ow}{in} \PY{p}{[}\PY{l+m+mf}{0.8}\PY{p}{,}\PY{l+m+mf}{0.9}\PY{p}{,}\PY{l+m+mi}{1}\PY{p}{]}\PY{p}{:}
    \PY{n}{E\PYZus{}dot\PYZus{}init} \PY{o}{=} \PY{n}{grain}\PY{o}{.}\PY{n}{material}\PY{o}{.}\PY{n}{kT\PYZus{}to\PYZus{}J}\PY{p}{(}\PY{n}{E\PYZus{}dot\PYZus{}init\PYZus{}kT}\PY{p}{)}
    \PY{n}{r}\PY{p}{,}\PY{n}{v}\PY{p}{,} \PY{n}{v\PYZus{}dot}\PY{p}{,} \PY{n}{data} \PY{o}{=} \PY{n}{grain}\PY{o}{.}\PY{n}{solve\PYZus{}with\PYZus{}values}\PY{p}{(}\PY{n}{E\PYZus{}init}\PY{p}{,} \PY{n}{E\PYZus{}dot\PYZus{}init}\PY{p}{)}


    \PY{n}{axe}\PY{o}{.}\PY{n}{plot}\PY{p}{(}\PY{n}{r}\PY{p}{,}\PY{n}{v}\PY{p}{,} \PY{n}{label}\PY{o}{=}\PY{n}{E\PYZus{}dot\PYZus{}init\PYZus{}kT}\PY{p}{)}
    
\PY{n}{res} \PY{o}{=} \PY{n}{find\PYZus{}best\PYZus{}v\PYZus{}dot\PYZus{}init}\PY{p}{(}\PY{n}{E\PYZus{}init\PYZus{}kT}\PY{p}{,} \PY{n}{grain}\PY{p}{)}
\PY{n}{E\PYZus{}dot\PYZus{}init\PYZus{}kT} \PY{o}{=} \PY{n}{res}\PY{o}{.}\PY{n}{x}
\PY{n}{E\PYZus{}dot\PYZus{}init} \PY{o}{=} \PY{n}{grain}\PY{o}{.}\PY{n}{material}\PY{o}{.}\PY{n}{kT\PYZus{}to\PYZus{}J}\PY{p}{(}\PY{n}{E\PYZus{}dot\PYZus{}init\PYZus{}kT}\PY{p}{)}
\PY{n}{r}\PY{p}{,}\PY{n}{v}\PY{p}{,} \PY{n}{v\PYZus{}dot}\PY{p}{,} \PY{n}{data} \PY{o}{=} \PY{n}{grain}\PY{o}{.}\PY{n}{solve\PYZus{}with\PYZus{}values}\PY{p}{(}\PY{n}{E\PYZus{}init}\PY{p}{,} \PY{n}{E\PYZus{}dot\PYZus{}init}\PY{p}{)}

\PY{n}{axe}\PY{o}{.}\PY{n}{plot}\PY{p}{(}\PY{n}{r}\PY{p}{,}\PY{n}{v}\PY{p}{,} \PY{l+s+s1}{\PYZsq{}}\PY{l+s+s1}{\PYZhy{}*}\PY{l+s+s1}{\PYZsq{}}\PY{p}{,} \PY{n}{label}\PY{o}{=}\PY{l+s+sa}{f}\PY{l+s+s1}{\PYZsq{}}\PY{l+s+si}{\PYZob{}E\PYZus{}dot\PYZus{}init\PYZus{}kT:.2f\PYZcb{}}\PY{l+s+s1}{ minimized}\PY{l+s+s1}{\PYZsq{}}\PY{p}{)}
    
\PY{n}{axe}\PY{o}{.}\PY{n}{set\PYZus{}ylabel}\PY{p}{(}\PY{l+s+s1}{\PYZsq{}}\PY{l+s+s1}{\PYZdl{}E\PYZus{}C(r\PYZca{}*)\PYZdl{} [\PYZdl{}k\PYZus{}BT\PYZdl{}]}\PY{l+s+s1}{\PYZsq{}}\PY{p}{,} \PY{n}{fontsize} \PY{o}{=}\PY{l+m+mi}{22}\PY{p}{)}
\PY{n}{axe}\PY{o}{.}\PY{n}{set\PYZus{}xlabel}\PY{p}{(}\PY{l+s+sa}{f}\PY{l+s+s2}{\PYZdq{}\PYZdq{}\PYZdq{}}\PY{l+s+s2}{Position in grain [\PYZdl{}L\PYZus{}D\PYZdl{} = }\PY{l+s+s2}{\PYZob{}}\PY{l+s+s2}{grain.material.LD*1e9:.2f\PYZcb{}nm],}\PY{l+s+se}{\PYZbs{}n}
\PY{l+s+s2}{                    R=}\PY{l+s+s2}{\PYZob{}}\PY{l+s+s2}{grain.R*1e9:.2f\PYZcb{}nm}\PY{l+s+s2}{\PYZdq{}\PYZdq{}\PYZdq{}}\PY{p}{,} \PY{n}{fontsize} \PY{o}{=}\PY{l+m+mi}{22}\PY{p}{)}
\PY{n}{axe}\PY{o}{.}\PY{n}{legend}\PY{p}{(}\PY{p}{)}\PY{p}{;}
\end{Verbatim}
\end{tcolorbox}

    \begin{center}
    \adjustimage{max size={0.9\linewidth}{0.9\paperheight}}{output_44_0.png}
    \end{center}
    { \hspace*{\fill} \\}
    
    \hypertarget{putting-the-pieces-together}{%
\subsubsection{Putting the pieces
together}\label{putting-the-pieces-together}}

With a description of the semiconductor itself by the class
\texttt{material} and the semiconductor grain by the class
\texttt{grain} the screening of multiple parameters can start. For the
beginning I will define one material and calculate the resulting
conduction band bending for multiple surface potential. Additionally I
would like to investigate the influence of different grain sizes on the
sensing performance.

Those results will lead in a second step to an understanding of the
relation between surface reaction, resistance change and grain size. But
for now lets concentrate on generating data for further analysis.

As we would like to do the time consuming calculations (finding the
right start conditions) only once, we will save the correct solution in
a \texttt{DataFrame}. As mentioned earlier, a \texttt{DataFrame} is a
data structure to organize information similar to Excel Worksheets. As
in ``Excel Worksheets'' data can be stored, accessed and manipulated. A
\texttt{Dataframe} is a part of the \texttt{pandas} Python library. To
shorten the command for pandas I will import it and add an alias to it.
The following code part import \texttt{pandas} and creates a
\texttt{Dataframe}, where all our results will be stored.

    \begin{tcolorbox}[breakable, size=fbox, boxrule=1pt, pad at break*=1mm,colback=cellbackground, colframe=cellborder]
\prompt{In}{incolor}{91}{\boxspacing}
\begin{Verbatim}[commandchars=\\\{\}]
\PY{k+kn}{import} \PY{n+nn}{pandas} \PY{k}{as} \PY{n+nn}{pd}
\PY{c+c1}{\PYZsh{}The results will be saved in a DataFrame}
\PY{n}{dF\PYZus{}calc} \PY{o}{=} \PY{n}{pd}\PY{o}{.}\PY{n}{DataFrame}\PY{p}{(}\PY{p}{)}
\end{Verbatim}
\end{tcolorbox}

    Next, we need to create the numerical representation of the material.

    \begin{tcolorbox}[breakable, size=fbox, boxrule=1pt, pad at break*=1mm,colback=cellbackground, colframe=cellborder]
\prompt{In}{incolor}{115}{\boxspacing}
\begin{Verbatim}[commandchars=\\\{\}]
\PY{c+c1}{\PYZsh{}define a grain with a specific material}
\PY{k}{def} \PY{n+nf}{create\PYZus{}grain}\PY{p}{(}\PY{n}{grainsize}\PY{p}{)}\PY{p}{:}
    \PY{n}{T\PYZus{}C} \PY{o}{=} \PY{l+m+mi}{300}
    \PY{n}{ND} \PY{o}{=} \PY{l+m+mf}{9e21}
    \PY{n}{mass\PYZus{}e\PYZus{}eff\PYZus{}factor} \PY{o}{=}\PY{l+m+mf}{0.3}

    \PY{n}{EDCF\PYZus{}eV} \PY{o}{=} \PY{n}{calc\PYZus{}EDCF\PYZus{}by\PYZus{}temp}\PY{p}{(}\PY{n}{T\PYZus{}C}\PY{p}{,} \PY{n}{ND}\PY{p}{,} \PY{n}{mass\PYZus{}e\PYZus{}eff\PYZus{}factor}\PY{p}{)}

    \PY{n+nb}{print}\PY{p}{(}\PY{l+s+sa}{f}\PY{l+s+s1}{\PYZsq{}}\PY{l+s+s1}{For SnO2 at }\PY{l+s+si}{\PYZob{}T\PYZus{}C\PYZcb{}}\PY{l+s+s1}{°C with a defect concentration of }\PY{l+s+si}{\PYZob{}ND\PYZcb{}}\PY{l+s+s1}{ 1/m³, the value of EDCF\PYZus{}eV is }\PY{l+s+si}{\PYZob{}EDCF\PYZus{}eV\PYZcb{}}\PY{l+s+s1}{ eV}\PY{l+s+s1}{\PYZsq{}}\PY{p}{)}    
    \PY{n}{material} \PY{o}{=} \PY{n}{Material}\PY{p}{(}\PY{n}{T\PYZus{}C}\PY{p}{,}\PY{n}{DIFF\PYZus{}EF\PYZus{}EC\PYZus{}evolt}\PY{o}{=}\PY{n}{EDCF\PYZus{}eV}\PY{p}{)}
    \PY{n}{grain} \PY{o}{=} \PY{n}{Grain}\PY{p}{(}\PY{n}{grainsize\PYZus{}radius}\PY{o}{=}\PY{n}{grainsize}\PY{p}{,}\PY{n}{material}\PY{o}{=}\PY{n}{material}\PY{p}{)}
    \PY{k}{return} \PY{n}{grain}

\PY{n}{grain} \PY{o}{=} \PY{n}{create\PYZus{}grain}\PY{p}{(}\PY{l+m+mf}{100e\PYZhy{}9}\PY{p}{)}
\end{Verbatim}
\end{tcolorbox}

    \begin{Verbatim}[commandchars=\\\{\}]
For SnO2 at 300°C with a defect concentration of 9e+21 1/m³, the value of
EDCF\_eV is 0.3163543980626456 eV
    \end{Verbatim}

    \begin{tcolorbox}[breakable, size=fbox, boxrule=1pt, pad at break*=1mm,colback=cellbackground, colframe=cellborder]
\prompt{In}{incolor}{126}{\boxspacing}
\begin{Verbatim}[commandchars=\\\{\}]
\PY{k}{def} \PY{n+nf}{calcualte\PYZus{}conduction\PYZus{}band}\PY{p}{(}\PY{n}{grain}\PY{p}{)}\PY{p}{:}
    \PY{n}{dF\PYZus{}calc\PYZus{}temp} \PY{o}{=} \PY{n}{pd}\PY{o}{.}\PY{n}{DataFrame}\PY{p}{(}\PY{p}{)}
    \PY{k}{for} \PY{n}{vinit} \PY{o+ow}{in} \PY{p}{[}\PY{o}{\PYZhy{}}\PY{l+m+mi}{8}\PY{p}{,}\PY{o}{\PYZhy{}}\PY{l+m+mi}{4}\PY{p}{,}\PY{o}{\PYZhy{}}\PY{l+m+mi}{2}\PY{p}{,}\PY{o}{\PYZhy{}}\PY{l+m+mi}{1}\PY{p}{,}\PY{l+m+mi}{0}\PY{p}{,}\PY{l+m+mi}{1}\PY{p}{,}\PY{l+m+mi}{2}\PY{p}{,}\PY{l+m+mi}{4}\PY{p}{,}\PY{l+m+mi}{8}\PY{p}{]}\PY{p}{:}
        \PY{n+nb}{print}\PY{p}{(}\PY{l+s+sa}{f}\PY{l+s+s1}{\PYZsq{}}\PY{l+s+s1}{Solving for }\PY{l+s+si}{\PYZob{}vinit\PYZcb{}}\PY{l+s+s1}{\PYZsq{}}\PY{p}{)}
        \PY{k}{if} \PY{n}{vinit} \PY{o+ow}{in} \PY{n}{dF\PYZus{}calc}\PY{o}{.}\PY{n}{index}\PY{p}{:}
            \PY{k}{pass}
        \PY{n}{bounds} \PY{o}{=} \PY{n+nb}{sorted}\PY{p}{(}\PY{p}{[}\PY{n}{vinit}\PY{o}{*}\PY{l+m+mi}{10}\PY{p}{,} \PY{l+m+mi}{0}\PY{p}{]}\PY{p}{)}
        \PY{n}{res} \PY{o}{=} \PY{n}{find\PYZus{}best\PYZus{}v\PYZus{}dot\PYZus{}init}\PY{p}{(}\PY{n}{vinit}\PY{p}{,} \PY{n}{grain}\PY{p}{)}
        \PY{c+c1}{\PYZsh{}res = minimize\PYZus{}scalar(min\PYZus{}vdot,  bounds=bounds, tol=abs(vinit/10000), method=\PYZsq{}Bounded\PYZsq{})}
        \PY{n}{ser\PYZus{}temp} \PY{o}{=} \PY{n}{pd}\PY{o}{.}\PY{n}{Series}\PY{p}{(}\PY{p}{)}
        \PY{n}{ser\PYZus{}temp}\PY{o}{.}\PY{n}{name} \PY{o}{=} \PY{n}{vinit}
        \PY{n}{ser\PYZus{}temp}\PY{p}{[}\PY{l+s+s1}{\PYZsq{}}\PY{l+s+s1}{Einit\PYZus{}kT}\PY{l+s+s1}{\PYZsq{}}\PY{p}{]} \PY{o}{=} \PY{n}{vinit}
        \PY{n}{ser\PYZus{}temp}\PY{p}{[}\PY{l+s+s1}{\PYZsq{}}\PY{l+s+s1}{E\PYZus{}dor\PYZus{}init\PYZus{}kT}\PY{l+s+s1}{\PYZsq{}}\PY{p}{]} \PY{o}{=} \PY{n}{res}\PY{o}{.}\PY{n}{x}
        \PY{n}{ser\PYZus{}temp}\PY{p}{[}\PY{l+s+s1}{\PYZsq{}}\PY{l+s+s1}{res}\PY{l+s+s1}{\PYZsq{}}\PY{p}{]} \PY{o}{=} \PY{n}{res}\PY{o}{.}\PY{n}{fun}



        \PY{n}{r}\PY{p}{,}\PY{n}{v}\PY{p}{,}\PY{n}{vdot}\PY{p}{,} \PY{n}{data} \PY{o}{=} \PY{n}{grain}\PY{o}{.}\PY{n}{solve\PYZus{}with\PYZus{}values}\PY{p}{(}\PY{n}{grain}\PY{o}{.}\PY{n}{material}\PY{o}{.}\PY{n}{kT\PYZus{}to\PYZus{}J}\PY{p}{(}\PY{n}{vinit}\PY{p}{)}\PY{p}{,}\PY{n}{grain}\PY{o}{.}\PY{n}{material}\PY{o}{.}\PY{n}{kT\PYZus{}to\PYZus{}J}\PY{p}{(}\PY{n}{res}\PY{o}{.}\PY{n}{x}\PY{p}{)}\PY{p}{,} \PY{p}{)}
        \PY{n}{r} \PY{o}{=} \PY{n}{np}\PY{o}{.}\PY{n}{concatenate}\PY{p}{(}\PY{p}{[}\PY{p}{[}\PY{l+m+mf}{0.01}\PY{p}{]}\PY{p}{,}\PY{n}{r}\PY{p}{]}\PY{p}{,} \PY{n}{axis}\PY{o}{=}\PY{l+m+mi}{0}\PY{p}{)}
        \PY{n}{v} \PY{o}{=} \PY{n}{np}\PY{o}{.}\PY{n}{concatenate}\PY{p}{(}\PY{p}{[}\PY{p}{[}\PY{n}{v}\PY{p}{[}\PY{l+m+mi}{0}\PY{p}{]}\PY{p}{]}\PY{p}{,}\PY{n}{v}\PY{p}{]}\PY{p}{,} \PY{n}{axis}\PY{o}{=}\PY{l+m+mi}{0}\PY{p}{)}
        \PY{n}{vdot} \PY{o}{=} \PY{n}{np}\PY{o}{.}\PY{n}{concatenate}\PY{p}{(}\PY{p}{[}\PY{p}{[}\PY{l+m+mi}{0}\PY{p}{]}\PY{p}{,}\PY{n}{vdot}\PY{p}{]}\PY{p}{,} \PY{n}{axis}\PY{o}{=}\PY{l+m+mi}{0}\PY{p}{)}
        \PY{n}{ser\PYZus{}temp}\PY{p}{[}\PY{l+s+s1}{\PYZsq{}}\PY{l+s+s1}{R}\PY{l+s+s1}{\PYZsq{}}\PY{p}{]} \PY{o}{=} \PY{n}{grain}\PY{o}{.}\PY{n}{R}
        \PY{n}{ser\PYZus{}temp}\PY{p}{[}\PY{l+s+s1}{\PYZsq{}}\PY{l+s+s1}{temp}\PY{l+s+s1}{\PYZsq{}}\PY{p}{]} \PY{o}{=} \PY{n}{grain}\PY{o}{.}\PY{n}{material}\PY{o}{.}\PY{n}{T\PYZus{}C}
        \PY{n}{ser\PYZus{}temp}\PY{p}{[}\PY{l+s+s1}{\PYZsq{}}\PY{l+s+s1}{v}\PY{l+s+s1}{\PYZsq{}}\PY{p}{]} \PY{o}{=} \PY{n}{v}
        \PY{n}{ser\PYZus{}temp}\PY{p}{[}\PY{l+s+s1}{\PYZsq{}}\PY{l+s+s1}{v\PYZus{}dot}\PY{l+s+s1}{\PYZsq{}}\PY{p}{]} \PY{o}{=} \PY{n}{vdot}
        \PY{n}{ser\PYZus{}temp}\PY{p}{[}\PY{l+s+s1}{\PYZsq{}}\PY{l+s+s1}{r}\PY{l+s+s1}{\PYZsq{}}\PY{p}{]} \PY{o}{=} \PY{n}{r}
        \PY{n}{ser\PYZus{}temp}\PY{p}{[}\PY{l+s+s1}{\PYZsq{}}\PY{l+s+s1}{mass\PYZus{}eff}\PY{l+s+s1}{\PYZsq{}}\PY{p}{]} \PY{o}{=} \PY{n}{grain}\PY{o}{.}\PY{n}{material}\PY{o}{.}\PY{n}{MASS\PYZus{}E\PYZus{}EFF}
        \PY{n}{ser\PYZus{}temp}\PY{p}{[}\PY{l+s+s1}{\PYZsq{}}\PY{l+s+s1}{ND}\PY{l+s+s1}{\PYZsq{}}\PY{p}{]} \PY{o}{=} \PY{n}{grain}\PY{o}{.}\PY{n}{material}\PY{o}{.}\PY{n}{ND}
        \PY{n}{ser\PYZus{}temp}\PY{p}{[}\PY{l+s+s1}{\PYZsq{}}\PY{l+s+s1}{EPSILON}\PY{l+s+s1}{\PYZsq{}}\PY{p}{]} \PY{o}{=} \PY{n}{grain}\PY{o}{.}\PY{n}{material}\PY{o}{.}\PY{n}{EPSILON}
        \PY{n}{ser\PYZus{}temp}\PY{p}{[}\PY{l+s+s1}{\PYZsq{}}\PY{l+s+s1}{nb}\PY{l+s+s1}{\PYZsq{}}\PY{p}{]} \PY{o}{=} \PY{n}{grain}\PY{o}{.}\PY{n}{material}\PY{o}{.}\PY{n}{nb}
        
        \PY{n}{n} \PY{o}{=} \PY{p}{[}\PY{n}{grain}\PY{o}{.}\PY{n}{material}\PY{o}{.}\PY{n}{n}\PY{p}{(}\PY{n}{v\PYZus{}J}\PY{p}{)}\PY{p}{[}\PY{l+m+mi}{0}\PY{p}{]} \PY{k}{for} \PY{n}{v\PYZus{}J} \PY{o+ow}{in} \PY{n}{grain}\PY{o}{.}\PY{n}{material}\PY{o}{.}\PY{n}{kT\PYZus{}to\PYZus{}J}\PY{p}{(}\PY{n}{v}\PY{p}{)}\PY{p}{]}
        \PY{n}{ser\PYZus{}temp}\PY{p}{[}\PY{l+s+s1}{\PYZsq{}}\PY{l+s+s1}{n}\PY{l+s+s1}{\PYZsq{}}\PY{p}{]} \PY{o}{=} \PY{n}{n}

        \PY{n}{dF\PYZus{}calc\PYZus{}temp} \PY{o}{=} \PY{n}{dF\PYZus{}calc\PYZus{}temp}\PY{o}{.}\PY{n}{append}\PY{p}{(}\PY{n}{ser\PYZus{}temp}\PY{p}{)}
    \PY{k}{return} \PY{n}{dF\PYZus{}calc\PYZus{}temp}
    
\PY{k}{for} \PY{n}{grainsize} \PY{o+ow}{in} \PY{p}{[}\PY{l+m+mf}{50e\PYZhy{}9}\PY{p}{,} \PY{l+m+mf}{100e\PYZhy{}9}\PY{p}{,} \PY{l+m+mf}{200e\PYZhy{}9}\PY{p}{,} \PY{l+m+mf}{400e\PYZhy{}9}\PY{p}{]}\PY{p}{:}
    \PY{n}{grain} \PY{o}{=} \PY{n}{create\PYZus{}grain}\PY{p}{(}\PY{n}{grainsize}\PY{p}{)}
    \PY{n}{dF\PYZus{}calc\PYZus{}temp} \PY{o}{=} \PY{n}{calcualte\PYZus{}conduction\PYZus{}band}\PY{p}{(}\PY{n}{grain}\PY{p}{)}
    \PY{n}{dF\PYZus{}calc} \PY{o}{=} \PY{n}{dF\PYZus{}calc}\PY{o}{.}\PY{n}{append}\PY{p}{(}\PY{n}{dF\PYZus{}calc\PYZus{}temp}\PY{p}{)}
    
\end{Verbatim}
\end{tcolorbox}

    \hypertarget{exportimport-data}{%
\paragraph{Export/Import data}\label{exportimport-data}}

The date will be saved for later use and to avoid a recalcualtion. It is
helpful to directly reimport the data to see if any mistakes in saving
the date have happened.

    \begin{tcolorbox}[breakable, size=fbox, boxrule=1pt, pad at break*=1mm,colback=cellbackground, colframe=cellborder]
\prompt{In}{incolor}{109}{\boxspacing}
\begin{Verbatim}[commandchars=\\\{\}]
\PY{n}{dF\PYZus{}calc}\PY{o}{.}\PY{n}{to\PYZus{}hdf}\PY{p}{(}\PY{l+s+s1}{\PYZsq{}}\PY{l+s+s1}{results.h5}\PY{l+s+s1}{\PYZsq{}}\PY{p}{,} \PY{l+s+s1}{\PYZsq{}}\PY{l+s+s1}{raw}\PY{l+s+s1}{\PYZsq{}}\PY{p}{)}
\end{Verbatim}
\end{tcolorbox}

    \begin{Verbatim}[commandchars=\\\{\}]
/usr/lib/python3.7/site-packages/pandas/core/generic.py:2530:
PerformanceWarning:
your performance may suffer as PyTables will pickle object types that it cannot
map directly to c-types [inferred\_type->mixed,key->block1\_values] [items->['n',
'r', 'v', 'v\_dot']]

  pytables.to\_hdf(path\_or\_buf, key, self, **kwargs)
    \end{Verbatim}

    \begin{tcolorbox}[breakable, size=fbox, boxrule=1pt, pad at break*=1mm,colback=cellbackground, colframe=cellborder]
\prompt{In}{incolor}{127}{\boxspacing}
\begin{Verbatim}[commandchars=\\\{\}]
\PY{n}{calc\PYZus{}dF} \PY{o}{=} \PY{n}{pd}\PY{o}{.}\PY{n}{read\PYZus{}hdf}\PY{p}{(}\PY{l+s+s1}{\PYZsq{}}\PY{l+s+s1}{results.h5}\PY{l+s+s1}{\PYZsq{}}\PY{p}{,} \PY{l+s+s1}{\PYZsq{}}\PY{l+s+s1}{raw}\PY{l+s+s1}{\PYZsq{}}\PY{p}{)}
\end{Verbatim}
\end{tcolorbox}

    \hypertarget{represent-the-results}{%
\subsubsection{Represent the results}\label{represent-the-results}}

    \begin{tcolorbox}[breakable, size=fbox, boxrule=1pt, pad at break*=1mm,colback=cellbackground, colframe=cellborder]
\prompt{In}{incolor}{131}{\boxspacing}
\begin{Verbatim}[commandchars=\\\{\}]
\PY{k}{for} \PY{n}{R}\PY{p}{,} \PY{n}{calc\PYZus{}dF\PYZus{}grainsize} \PY{o+ow}{in} \PY{n}{calc\PYZus{}dF}\PY{o}{.}\PY{n}{groupby}\PY{p}{(}\PY{l+s+s1}{\PYZsq{}}\PY{l+s+s1}{R}\PY{l+s+s1}{\PYZsq{}}\PY{p}{)}\PY{p}{:}
    \PY{n}{T\PYZus{}C} \PY{o}{=} \PY{n}{calc\PYZus{}dF\PYZus{}grainsize}\PY{o}{.}\PY{n}{iloc}\PY{p}{[}\PY{l+m+mi}{0}\PY{p}{]}\PY{p}{[}\PY{l+s+s1}{\PYZsq{}}\PY{l+s+s1}{temp}\PY{l+s+s1}{\PYZsq{}}\PY{p}{]}
    \PY{n}{ND} \PY{o}{=} \PY{n}{calc\PYZus{}dF\PYZus{}grainsize}\PY{o}{.}\PY{n}{iloc}\PY{p}{[}\PY{l+m+mi}{0}\PY{p}{]}\PY{p}{[}\PY{l+s+s1}{\PYZsq{}}\PY{l+s+s1}{ND}\PY{l+s+s1}{\PYZsq{}}\PY{p}{]}
    \PY{n}{mass\PYZus{}e\PYZus{}eff\PYZus{}factor} \PY{o}{=} \PY{n}{calc\PYZus{}dF\PYZus{}grainsize}\PY{o}{.}\PY{n}{iloc}\PY{p}{[}\PY{l+m+mi}{0}\PY{p}{]}\PY{p}{[}\PY{l+s+s1}{\PYZsq{}}\PY{l+s+s1}{mass\PYZus{}eff}\PY{l+s+s1}{\PYZsq{}}\PY{p}{]}\PY{o}{/}\PY{n}{CONST}\PY{o}{.}\PY{n}{MASS\PYZus{}E} \PY{c+c1}{\PYZsh{}this is a hack because the effective mass was saved, not the factor}

    \PY{n}{EDCF\PYZus{}eV} \PY{o}{=} \PY{n}{calc\PYZus{}EDCF\PYZus{}by\PYZus{}temp}\PY{p}{(}\PY{n}{T\PYZus{}C}\PY{p}{,} \PY{n}{ND}\PY{p}{,} \PY{n}{mass\PYZus{}e\PYZus{}eff\PYZus{}factor}\PY{p}{)}

    \PY{n}{material} \PY{o}{=} \PY{n}{Material}\PY{p}{(}\PY{n}{T\PYZus{}C}\PY{p}{,}\PY{n}{DIFF\PYZus{}EF\PYZus{}EC\PYZus{}evolt}\PY{o}{=}\PY{n}{EDCF\PYZus{}eV}\PY{p}{)}
    \PY{n}{temp\PYZus{}grain} \PY{o}{=} \PY{n}{Grain}\PY{p}{(}\PY{n}{grainsize\PYZus{}radius}\PY{o}{=}\PY{n}{R}\PY{p}{,}\PY{n}{material}\PY{o}{=}\PY{n}{material}\PY{p}{)}
    
    \PY{n}{fig}\PY{p}{,} \PY{n}{axe}\PY{o}{=} \PY{n}{subplots}\PY{p}{(}\PY{n}{figsize} \PY{o}{=} \PY{p}{(}\PY{l+m+mi}{16}\PY{p}{,}\PY{l+m+mi}{9}\PY{p}{)}\PY{p}{)}
    \PY{n}{axe}\PY{o}{.}\PY{n}{axhline}\PY{p}{(}\PY{o}{\PYZhy{}}\PY{n}{temp\PYZus{}grain}\PY{o}{.}\PY{n}{material}\PY{o}{.}\PY{n}{J\PYZus{}to\PYZus{}kT}\PY{p}{(}\PY{n}{temp\PYZus{}grain}\PY{o}{.}\PY{n}{material}\PY{o}{.}\PY{n}{Diff\PYZus{}EF\PYZus{}EC}\PY{p}{)}\PY{p}{,}\PY{n}{linestyle}\PY{o}{=}\PY{l+s+s1}{\PYZsq{}}\PY{l+s+s1}{\PYZhy{}\PYZhy{}}\PY{l+s+s1}{\PYZsq{}}\PY{p}{,}\PY{n}{color}\PY{o}{=}\PY{l+s+s1}{\PYZsq{}}\PY{l+s+s1}{r}\PY{l+s+s1}{\PYZsq{}}\PY{p}{,} \PY{n}{label}\PY{o}{=}\PY{l+s+s1}{\PYZsq{}}\PY{l+s+s1}{Fermi Level}\PY{l+s+s1}{\PYZsq{}}\PY{p}{)}
    \PY{n}{axe}\PY{o}{.}\PY{n}{set\PYZus{}ylim}\PY{p}{(}\PY{o}{\PYZhy{}}\PY{l+m+mi}{10}\PY{p}{,}\PY{l+m+mi}{10}\PY{p}{)}

    \PY{k}{for} \PY{n}{vinit}\PY{p}{,} \PY{n}{ser\PYZus{}temp} \PY{o+ow}{in} \PY{n}{calc\PYZus{}dF\PYZus{}grainsize}\PY{o}{.}\PY{n}{iterrows}\PY{p}{(}\PY{p}{)}\PY{p}{:}

        \PY{n}{r} \PY{o}{=} \PY{n}{ser\PYZus{}temp}\PY{p}{[}\PY{l+s+s1}{\PYZsq{}}\PY{l+s+s1}{r}\PY{l+s+s1}{\PYZsq{}}\PY{p}{]}
        \PY{n}{v} \PY{o}{=} \PY{n}{ser\PYZus{}temp}\PY{p}{[}\PY{l+s+s1}{\PYZsq{}}\PY{l+s+s1}{v}\PY{l+s+s1}{\PYZsq{}}\PY{p}{]}
        \PY{n}{vdot} \PY{o}{=} \PY{n}{ser\PYZus{}temp}\PY{p}{[}\PY{l+s+s1}{\PYZsq{}}\PY{l+s+s1}{v\PYZus{}dot}\PY{l+s+s1}{\PYZsq{}}\PY{p}{]}

        \PY{n}{axe}\PY{o}{.}\PY{n}{set\PYZus{}title}\PY{p}{(}\PY{l+s+sa}{f}\PY{l+s+s1}{\PYZsq{}}\PY{l+s+s1}{Grain radius = }\PY{l+s+s1}{\PYZob{}}\PY{l+s+s1}{temp\PYZus{}grain.R*1e9:.2f\PYZcb{}nm}\PY{l+s+s1}{\PYZsq{}}\PY{p}{,} \PY{n}{fontsize}\PY{o}{=}\PY{l+m+mi}{22}\PY{p}{)}
        
        \PY{n}{axe}\PY{o}{.}\PY{n}{plot}\PY{p}{(}\PY{n}{r}\PY{p}{,}\PY{n}{v}\PY{p}{,} \PY{l+s+s1}{\PYZsq{}}\PY{l+s+s1}{\PYZhy{}}\PY{l+s+s1}{\PYZsq{}}\PY{p}{,} \PY{n}{label} \PY{o}{=} \PY{l+s+s2}{\PYZdq{}}\PY{l+s+s2}{\PYZdq{}}\PY{p}{)}
        \PY{n}{axe}\PY{o}{.}\PY{n}{set\PYZus{}ylabel}\PY{p}{(}\PY{l+s+s1}{\PYZsq{}}\PY{l+s+s1}{\PYZdl{}E\PYZus{}C(r)\PYZdl{} [\PYZdl{}k\PYZus{}BT\PYZdl{}]}\PY{l+s+s1}{\PYZsq{}}\PY{p}{,} \PY{n}{fontsize} \PY{o}{=}\PY{l+m+mi}{22}\PY{p}{)}
        \PY{n}{axe}\PY{o}{.}\PY{n}{set\PYZus{}xlabel}\PY{p}{(}\PY{l+s+sa}{f}\PY{l+s+s1}{\PYZsq{}}\PY{l+s+s1}{Position in grain [\PYZdl{}L\PYZus{}D\PYZdl{} = }\PY{l+s+s1}{\PYZob{}}\PY{l+s+s1}{temp\PYZus{}grain.material.LD*1e9:.2f\PYZcb{}nm], R=}\PY{l+s+s1}{\PYZob{}}\PY{l+s+s1}{temp\PYZus{}grain.R*1e9:.2f\PYZcb{}nm}\PY{l+s+s1}{\PYZsq{}}\PY{p}{,} \PY{n}{fontsize} \PY{o}{=}\PY{l+m+mi}{22}\PY{p}{)}
    \PY{n}{axe}\PY{o}{.}\PY{n}{legend}\PY{p}{(}\PY{p}{)}
    \PY{n}{axe}\PY{o}{.}\PY{n}{grid}\PY{p}{(}\PY{n}{b}\PY{o}{=}\PY{k+kc}{True}\PY{p}{)}
\end{Verbatim}
\end{tcolorbox}

    \begin{center}
    \adjustimage{max size={0.9\linewidth}{0.9\paperheight}}{output_54_0.png}
    \end{center}
    { \hspace*{\fill} \\}
    
    \begin{center}
    \adjustimage{max size={0.9\linewidth}{0.9\paperheight}}{output_54_1.png}
    \end{center}
    { \hspace*{\fill} \\}
    
    \begin{center}
    \adjustimage{max size={0.9\linewidth}{0.9\paperheight}}{output_54_2.png}
    \end{center}
    { \hspace*{\fill} \\}
    
    \begin{center}
    \adjustimage{max size={0.9\linewidth}{0.9\paperheight}}{output_54_3.png}
    \end{center}
    { \hspace*{\fill} \\}
    
    \hypertarget{summary}{%
\section{Summary}\label{summary}}

In this notebook the flowowing steps have been accomplished:

\begin{itemize}
\tightlist
\item
  numerically calculate the charge density in a semiconductor
\item
  solve the Poisson equation for spherical grains
\item
  Calculate the solutions for multiple grain sizes and surface
  potentials
\end{itemize}

Those calculations have been derived with a standard set of Python
tools. By using mainly the \texttt{numpy}, \texttt{scipy},
\texttt{matplotlib} and \texttt{pandas} these results have been
achieved.

To avoid a to large blocks of information in one notebook I like to
introduce a breakpoint here. At such breakpoints it is helpful to save
all the relevant gathered data in a \texttt{DataFrame}, save it to the
filesystem, and pick it up again in a fresh notebook. This keeps each
notebooks close to one topic and and introduces directly structure in
the data.

In the next notebook this calculated date will be used derive the total
resistance of a grain under different conditions. The anisotropic charge
carrier distribution inside the grain has a mayor influence on the total
resistance. For two extreme cases, the conduction path inside the grain
differs a lot. Those cases are:

\begin{enumerate}
\def\labelenumi{\arabic{enumi}.}
\tightlist
\item
  Accumulation layer at the surface
\item
  Depletion layer at the surface
\end{enumerate}

In the case of 1., the current will most likely run along the highly
conductive surface of the grain. In the second case, the current will
need to overcome a highly resistive surface layer and then propagate
through the inside of the relatively low resistive bulk of the grain.

Since all information to numerically derive the effects are now
pre-calcualted, the next notebook will start at this point and continue
to calcualte the total
resistance.\href{./2.1-From_smox_grain_to_resistance.ipynb}{Non-PDF
readers, could use this link to guide them to the next notebook.}

    This graph shows how a surface potential is shielded by the remaining
ionized donors. In the case of on deletion layer (
\(E_{C_{Surface}}>0 )\)), the total number of charges shielding the
surface potential is rather small compared to the amount of charges in
an accumulation layer ( \(E_{C_{Surface}}<0 )\)). The result of such an
asymmetry is visible in the graph. The width of the accumulation layer
is by far smaller then the width of the depleted are.

    \hypertarget{bibliography-section}{%
\section{Bibliography section}\label{bibliography-section}}

    \bibliographystyle{alphadin}
\bibliography{ipython}


    % Add a bibliography block to the postdoc
    
    
    
\end{document}
