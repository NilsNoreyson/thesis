\documentclass[11pt]{article}

    \usepackage[breakable]{tcolorbox}
    \usepackage{parskip} % Stop auto-indenting (to mimic markdown behaviour)
    
    \usepackage{iftex}
    \ifPDFTeX
    	\usepackage[T1]{fontenc}
    	\usepackage{mathpazo}
    \else
    	\usepackage{fontspec}
    \fi

    % Basic figure setup, for now with no caption control since it's done
    % automatically by Pandoc (which extracts ![](path) syntax from Markdown).
    \usepackage{graphicx}
    % Maintain compatibility with old templates. Remove in nbconvert 6.0
    \let\Oldincludegraphics\includegraphics
    % Ensure that by default, figures have no caption (until we provide a
    % proper Figure object with a Caption API and a way to capture that
    % in the conversion process - todo).
    \usepackage{caption}
    \DeclareCaptionFormat{nocaption}{}
    \captionsetup{format=nocaption,aboveskip=0pt,belowskip=0pt}

    \usepackage[Export]{adjustbox} % Used to constrain images to a maximum size
    \adjustboxset{max size={0.9\linewidth}{0.9\paperheight}}
    \usepackage{float}
    \floatplacement{figure}{H} % forces figures to be placed at the correct location
    \usepackage{xcolor} % Allow colors to be defined
    \usepackage{enumerate} % Needed for markdown enumerations to work
    \usepackage{geometry} % Used to adjust the document margins
    \usepackage{amsmath} % Equations
    \usepackage{amssymb} % Equations
    \usepackage{textcomp} % defines textquotesingle
    % Hack from http://tex.stackexchange.com/a/47451/13684:
    \AtBeginDocument{%
        \def\PYZsq{\textquotesingle}% Upright quotes in Pygmentized code
    }
    \usepackage{upquote} % Upright quotes for verbatim code
    \usepackage{eurosym} % defines \euro
    \usepackage[mathletters]{ucs} % Extended unicode (utf-8) support
    \usepackage{fancyvrb} % verbatim replacement that allows latex
    \usepackage{grffile} % extends the file name processing of package graphics 
                         % to support a larger range
    \makeatletter % fix for grffile with XeLaTeX
    \def\Gread@@xetex#1{%
      \IfFileExists{"\Gin@base".bb}%
      {\Gread@eps{\Gin@base.bb}}%
      {\Gread@@xetex@aux#1}%
    }
    \makeatother

    % The hyperref package gives us a pdf with properly built
    % internal navigation ('pdf bookmarks' for the table of contents,
    % internal cross-reference links, web links for URLs, etc.)
    \usepackage{hyperref}
    % The default LaTeX title has an obnoxious amount of whitespace. By default,
    % titling removes some of it. It also provides customization options.
    \usepackage{titling}
    \usepackage{longtable} % longtable support required by pandoc >1.10
    \usepackage{booktabs}  % table support for pandoc > 1.12.2
    \usepackage[inline]{enumitem} % IRkernel/repr support (it uses the enumerate* environment)
    \usepackage[normalem]{ulem} % ulem is needed to support strikethroughs (\sout)
                                % normalem makes italics be italics, not underlines
    \usepackage{mathrsfs}
    

    
    % Colors for the hyperref package
    \definecolor{urlcolor}{rgb}{0,.145,.698}
    \definecolor{linkcolor}{rgb}{.71,0.21,0.01}
    \definecolor{citecolor}{rgb}{.12,.54,.11}

    % ANSI colors
    \definecolor{ansi-black}{HTML}{3E424D}
    \definecolor{ansi-black-intense}{HTML}{282C36}
    \definecolor{ansi-red}{HTML}{E75C58}
    \definecolor{ansi-red-intense}{HTML}{B22B31}
    \definecolor{ansi-green}{HTML}{00A250}
    \definecolor{ansi-green-intense}{HTML}{007427}
    \definecolor{ansi-yellow}{HTML}{DDB62B}
    \definecolor{ansi-yellow-intense}{HTML}{B27D12}
    \definecolor{ansi-blue}{HTML}{208FFB}
    \definecolor{ansi-blue-intense}{HTML}{0065CA}
    \definecolor{ansi-magenta}{HTML}{D160C4}
    \definecolor{ansi-magenta-intense}{HTML}{A03196}
    \definecolor{ansi-cyan}{HTML}{60C6C8}
    \definecolor{ansi-cyan-intense}{HTML}{258F8F}
    \definecolor{ansi-white}{HTML}{C5C1B4}
    \definecolor{ansi-white-intense}{HTML}{A1A6B2}
    \definecolor{ansi-default-inverse-fg}{HTML}{FFFFFF}
    \definecolor{ansi-default-inverse-bg}{HTML}{000000}

    % commands and environments needed by pandoc snippets
    % extracted from the output of `pandoc -s`
    \providecommand{\tightlist}{%
      \setlength{\itemsep}{0pt}\setlength{\parskip}{0pt}}
    \DefineVerbatimEnvironment{Highlighting}{Verbatim}{commandchars=\\\{\}}
    % Add ',fontsize=\small' for more characters per line
    \newenvironment{Shaded}{}{}
    \newcommand{\KeywordTok}[1]{\textcolor[rgb]{0.00,0.44,0.13}{\textbf{{#1}}}}
    \newcommand{\DataTypeTok}[1]{\textcolor[rgb]{0.56,0.13,0.00}{{#1}}}
    \newcommand{\DecValTok}[1]{\textcolor[rgb]{0.25,0.63,0.44}{{#1}}}
    \newcommand{\BaseNTok}[1]{\textcolor[rgb]{0.25,0.63,0.44}{{#1}}}
    \newcommand{\FloatTok}[1]{\textcolor[rgb]{0.25,0.63,0.44}{{#1}}}
    \newcommand{\CharTok}[1]{\textcolor[rgb]{0.25,0.44,0.63}{{#1}}}
    \newcommand{\StringTok}[1]{\textcolor[rgb]{0.25,0.44,0.63}{{#1}}}
    \newcommand{\CommentTok}[1]{\textcolor[rgb]{0.38,0.63,0.69}{\textit{{#1}}}}
    \newcommand{\OtherTok}[1]{\textcolor[rgb]{0.00,0.44,0.13}{{#1}}}
    \newcommand{\AlertTok}[1]{\textcolor[rgb]{1.00,0.00,0.00}{\textbf{{#1}}}}
    \newcommand{\FunctionTok}[1]{\textcolor[rgb]{0.02,0.16,0.49}{{#1}}}
    \newcommand{\RegionMarkerTok}[1]{{#1}}
    \newcommand{\ErrorTok}[1]{\textcolor[rgb]{1.00,0.00,0.00}{\textbf{{#1}}}}
    \newcommand{\NormalTok}[1]{{#1}}
    
    % Additional commands for more recent versions of Pandoc
    \newcommand{\ConstantTok}[1]{\textcolor[rgb]{0.53,0.00,0.00}{{#1}}}
    \newcommand{\SpecialCharTok}[1]{\textcolor[rgb]{0.25,0.44,0.63}{{#1}}}
    \newcommand{\VerbatimStringTok}[1]{\textcolor[rgb]{0.25,0.44,0.63}{{#1}}}
    \newcommand{\SpecialStringTok}[1]{\textcolor[rgb]{0.73,0.40,0.53}{{#1}}}
    \newcommand{\ImportTok}[1]{{#1}}
    \newcommand{\DocumentationTok}[1]{\textcolor[rgb]{0.73,0.13,0.13}{\textit{{#1}}}}
    \newcommand{\AnnotationTok}[1]{\textcolor[rgb]{0.38,0.63,0.69}{\textbf{\textit{{#1}}}}}
    \newcommand{\CommentVarTok}[1]{\textcolor[rgb]{0.38,0.63,0.69}{\textbf{\textit{{#1}}}}}
    \newcommand{\VariableTok}[1]{\textcolor[rgb]{0.10,0.09,0.49}{{#1}}}
    \newcommand{\ControlFlowTok}[1]{\textcolor[rgb]{0.00,0.44,0.13}{\textbf{{#1}}}}
    \newcommand{\OperatorTok}[1]{\textcolor[rgb]{0.40,0.40,0.40}{{#1}}}
    \newcommand{\BuiltInTok}[1]{{#1}}
    \newcommand{\ExtensionTok}[1]{{#1}}
    \newcommand{\PreprocessorTok}[1]{\textcolor[rgb]{0.74,0.48,0.00}{{#1}}}
    \newcommand{\AttributeTok}[1]{\textcolor[rgb]{0.49,0.56,0.16}{{#1}}}
    \newcommand{\InformationTok}[1]{\textcolor[rgb]{0.38,0.63,0.69}{\textbf{\textit{{#1}}}}}
    \newcommand{\WarningTok}[1]{\textcolor[rgb]{0.38,0.63,0.69}{\textbf{\textit{{#1}}}}}
    
    
    % Define a nice break command that doesn't care if a line doesn't already
    % exist.
    \def\br{\hspace*{\fill} \\* }
    % Math Jax compatibility definitions
    \def\gt{>}
    \def\lt{<}
    \let\Oldtex\TeX
    \let\Oldlatex\LaTeX
    \renewcommand{\TeX}{\textrm{\Oldtex}}
    \renewcommand{\LaTeX}{\textrm{\Oldlatex}}
    % Document parameters
    % Document title
    \title{Jupyter for scientific research}
    
    
    
    
    
% Pygments definitions
\makeatletter
\def\PY@reset{\let\PY@it=\relax \let\PY@bf=\relax%
    \let\PY@ul=\relax \let\PY@tc=\relax%
    \let\PY@bc=\relax \let\PY@ff=\relax}
\def\PY@tok#1{\csname PY@tok@#1\endcsname}
\def\PY@toks#1+{\ifx\relax#1\empty\else%
    \PY@tok{#1}\expandafter\PY@toks\fi}
\def\PY@do#1{\PY@bc{\PY@tc{\PY@ul{%
    \PY@it{\PY@bf{\PY@ff{#1}}}}}}}
\def\PY#1#2{\PY@reset\PY@toks#1+\relax+\PY@do{#2}}

\expandafter\def\csname PY@tok@w\endcsname{\def\PY@tc##1{\textcolor[rgb]{0.73,0.73,0.73}{##1}}}
\expandafter\def\csname PY@tok@c\endcsname{\let\PY@it=\textit\def\PY@tc##1{\textcolor[rgb]{0.25,0.50,0.50}{##1}}}
\expandafter\def\csname PY@tok@cp\endcsname{\def\PY@tc##1{\textcolor[rgb]{0.74,0.48,0.00}{##1}}}
\expandafter\def\csname PY@tok@k\endcsname{\let\PY@bf=\textbf\def\PY@tc##1{\textcolor[rgb]{0.00,0.50,0.00}{##1}}}
\expandafter\def\csname PY@tok@kp\endcsname{\def\PY@tc##1{\textcolor[rgb]{0.00,0.50,0.00}{##1}}}
\expandafter\def\csname PY@tok@kt\endcsname{\def\PY@tc##1{\textcolor[rgb]{0.69,0.00,0.25}{##1}}}
\expandafter\def\csname PY@tok@o\endcsname{\def\PY@tc##1{\textcolor[rgb]{0.40,0.40,0.40}{##1}}}
\expandafter\def\csname PY@tok@ow\endcsname{\let\PY@bf=\textbf\def\PY@tc##1{\textcolor[rgb]{0.67,0.13,1.00}{##1}}}
\expandafter\def\csname PY@tok@nb\endcsname{\def\PY@tc##1{\textcolor[rgb]{0.00,0.50,0.00}{##1}}}
\expandafter\def\csname PY@tok@nf\endcsname{\def\PY@tc##1{\textcolor[rgb]{0.00,0.00,1.00}{##1}}}
\expandafter\def\csname PY@tok@nc\endcsname{\let\PY@bf=\textbf\def\PY@tc##1{\textcolor[rgb]{0.00,0.00,1.00}{##1}}}
\expandafter\def\csname PY@tok@nn\endcsname{\let\PY@bf=\textbf\def\PY@tc##1{\textcolor[rgb]{0.00,0.00,1.00}{##1}}}
\expandafter\def\csname PY@tok@ne\endcsname{\let\PY@bf=\textbf\def\PY@tc##1{\textcolor[rgb]{0.82,0.25,0.23}{##1}}}
\expandafter\def\csname PY@tok@nv\endcsname{\def\PY@tc##1{\textcolor[rgb]{0.10,0.09,0.49}{##1}}}
\expandafter\def\csname PY@tok@no\endcsname{\def\PY@tc##1{\textcolor[rgb]{0.53,0.00,0.00}{##1}}}
\expandafter\def\csname PY@tok@nl\endcsname{\def\PY@tc##1{\textcolor[rgb]{0.63,0.63,0.00}{##1}}}
\expandafter\def\csname PY@tok@ni\endcsname{\let\PY@bf=\textbf\def\PY@tc##1{\textcolor[rgb]{0.60,0.60,0.60}{##1}}}
\expandafter\def\csname PY@tok@na\endcsname{\def\PY@tc##1{\textcolor[rgb]{0.49,0.56,0.16}{##1}}}
\expandafter\def\csname PY@tok@nt\endcsname{\let\PY@bf=\textbf\def\PY@tc##1{\textcolor[rgb]{0.00,0.50,0.00}{##1}}}
\expandafter\def\csname PY@tok@nd\endcsname{\def\PY@tc##1{\textcolor[rgb]{0.67,0.13,1.00}{##1}}}
\expandafter\def\csname PY@tok@s\endcsname{\def\PY@tc##1{\textcolor[rgb]{0.73,0.13,0.13}{##1}}}
\expandafter\def\csname PY@tok@sd\endcsname{\let\PY@it=\textit\def\PY@tc##1{\textcolor[rgb]{0.73,0.13,0.13}{##1}}}
\expandafter\def\csname PY@tok@si\endcsname{\let\PY@bf=\textbf\def\PY@tc##1{\textcolor[rgb]{0.73,0.40,0.53}{##1}}}
\expandafter\def\csname PY@tok@se\endcsname{\let\PY@bf=\textbf\def\PY@tc##1{\textcolor[rgb]{0.73,0.40,0.13}{##1}}}
\expandafter\def\csname PY@tok@sr\endcsname{\def\PY@tc##1{\textcolor[rgb]{0.73,0.40,0.53}{##1}}}
\expandafter\def\csname PY@tok@ss\endcsname{\def\PY@tc##1{\textcolor[rgb]{0.10,0.09,0.49}{##1}}}
\expandafter\def\csname PY@tok@sx\endcsname{\def\PY@tc##1{\textcolor[rgb]{0.00,0.50,0.00}{##1}}}
\expandafter\def\csname PY@tok@m\endcsname{\def\PY@tc##1{\textcolor[rgb]{0.40,0.40,0.40}{##1}}}
\expandafter\def\csname PY@tok@gh\endcsname{\let\PY@bf=\textbf\def\PY@tc##1{\textcolor[rgb]{0.00,0.00,0.50}{##1}}}
\expandafter\def\csname PY@tok@gu\endcsname{\let\PY@bf=\textbf\def\PY@tc##1{\textcolor[rgb]{0.50,0.00,0.50}{##1}}}
\expandafter\def\csname PY@tok@gd\endcsname{\def\PY@tc##1{\textcolor[rgb]{0.63,0.00,0.00}{##1}}}
\expandafter\def\csname PY@tok@gi\endcsname{\def\PY@tc##1{\textcolor[rgb]{0.00,0.63,0.00}{##1}}}
\expandafter\def\csname PY@tok@gr\endcsname{\def\PY@tc##1{\textcolor[rgb]{1.00,0.00,0.00}{##1}}}
\expandafter\def\csname PY@tok@ge\endcsname{\let\PY@it=\textit}
\expandafter\def\csname PY@tok@gs\endcsname{\let\PY@bf=\textbf}
\expandafter\def\csname PY@tok@gp\endcsname{\let\PY@bf=\textbf\def\PY@tc##1{\textcolor[rgb]{0.00,0.00,0.50}{##1}}}
\expandafter\def\csname PY@tok@go\endcsname{\def\PY@tc##1{\textcolor[rgb]{0.53,0.53,0.53}{##1}}}
\expandafter\def\csname PY@tok@gt\endcsname{\def\PY@tc##1{\textcolor[rgb]{0.00,0.27,0.87}{##1}}}
\expandafter\def\csname PY@tok@err\endcsname{\def\PY@bc##1{\setlength{\fboxsep}{0pt}\fcolorbox[rgb]{1.00,0.00,0.00}{1,1,1}{\strut ##1}}}
\expandafter\def\csname PY@tok@kc\endcsname{\let\PY@bf=\textbf\def\PY@tc##1{\textcolor[rgb]{0.00,0.50,0.00}{##1}}}
\expandafter\def\csname PY@tok@kd\endcsname{\let\PY@bf=\textbf\def\PY@tc##1{\textcolor[rgb]{0.00,0.50,0.00}{##1}}}
\expandafter\def\csname PY@tok@kn\endcsname{\let\PY@bf=\textbf\def\PY@tc##1{\textcolor[rgb]{0.00,0.50,0.00}{##1}}}
\expandafter\def\csname PY@tok@kr\endcsname{\let\PY@bf=\textbf\def\PY@tc##1{\textcolor[rgb]{0.00,0.50,0.00}{##1}}}
\expandafter\def\csname PY@tok@bp\endcsname{\def\PY@tc##1{\textcolor[rgb]{0.00,0.50,0.00}{##1}}}
\expandafter\def\csname PY@tok@fm\endcsname{\def\PY@tc##1{\textcolor[rgb]{0.00,0.00,1.00}{##1}}}
\expandafter\def\csname PY@tok@vc\endcsname{\def\PY@tc##1{\textcolor[rgb]{0.10,0.09,0.49}{##1}}}
\expandafter\def\csname PY@tok@vg\endcsname{\def\PY@tc##1{\textcolor[rgb]{0.10,0.09,0.49}{##1}}}
\expandafter\def\csname PY@tok@vi\endcsname{\def\PY@tc##1{\textcolor[rgb]{0.10,0.09,0.49}{##1}}}
\expandafter\def\csname PY@tok@vm\endcsname{\def\PY@tc##1{\textcolor[rgb]{0.10,0.09,0.49}{##1}}}
\expandafter\def\csname PY@tok@sa\endcsname{\def\PY@tc##1{\textcolor[rgb]{0.73,0.13,0.13}{##1}}}
\expandafter\def\csname PY@tok@sb\endcsname{\def\PY@tc##1{\textcolor[rgb]{0.73,0.13,0.13}{##1}}}
\expandafter\def\csname PY@tok@sc\endcsname{\def\PY@tc##1{\textcolor[rgb]{0.73,0.13,0.13}{##1}}}
\expandafter\def\csname PY@tok@dl\endcsname{\def\PY@tc##1{\textcolor[rgb]{0.73,0.13,0.13}{##1}}}
\expandafter\def\csname PY@tok@s2\endcsname{\def\PY@tc##1{\textcolor[rgb]{0.73,0.13,0.13}{##1}}}
\expandafter\def\csname PY@tok@sh\endcsname{\def\PY@tc##1{\textcolor[rgb]{0.73,0.13,0.13}{##1}}}
\expandafter\def\csname PY@tok@s1\endcsname{\def\PY@tc##1{\textcolor[rgb]{0.73,0.13,0.13}{##1}}}
\expandafter\def\csname PY@tok@mb\endcsname{\def\PY@tc##1{\textcolor[rgb]{0.40,0.40,0.40}{##1}}}
\expandafter\def\csname PY@tok@mf\endcsname{\def\PY@tc##1{\textcolor[rgb]{0.40,0.40,0.40}{##1}}}
\expandafter\def\csname PY@tok@mh\endcsname{\def\PY@tc##1{\textcolor[rgb]{0.40,0.40,0.40}{##1}}}
\expandafter\def\csname PY@tok@mi\endcsname{\def\PY@tc##1{\textcolor[rgb]{0.40,0.40,0.40}{##1}}}
\expandafter\def\csname PY@tok@il\endcsname{\def\PY@tc##1{\textcolor[rgb]{0.40,0.40,0.40}{##1}}}
\expandafter\def\csname PY@tok@mo\endcsname{\def\PY@tc##1{\textcolor[rgb]{0.40,0.40,0.40}{##1}}}
\expandafter\def\csname PY@tok@ch\endcsname{\let\PY@it=\textit\def\PY@tc##1{\textcolor[rgb]{0.25,0.50,0.50}{##1}}}
\expandafter\def\csname PY@tok@cm\endcsname{\let\PY@it=\textit\def\PY@tc##1{\textcolor[rgb]{0.25,0.50,0.50}{##1}}}
\expandafter\def\csname PY@tok@cpf\endcsname{\let\PY@it=\textit\def\PY@tc##1{\textcolor[rgb]{0.25,0.50,0.50}{##1}}}
\expandafter\def\csname PY@tok@c1\endcsname{\let\PY@it=\textit\def\PY@tc##1{\textcolor[rgb]{0.25,0.50,0.50}{##1}}}
\expandafter\def\csname PY@tok@cs\endcsname{\let\PY@it=\textit\def\PY@tc##1{\textcolor[rgb]{0.25,0.50,0.50}{##1}}}

\def\PYZbs{\char`\\}
\def\PYZus{\char`\_}
\def\PYZob{\char`\{}
\def\PYZcb{\char`\}}
\def\PYZca{\char`\^}
\def\PYZam{\char`\&}
\def\PYZlt{\char`\<}
\def\PYZgt{\char`\>}
\def\PYZsh{\char`\#}
\def\PYZpc{\char`\%}
\def\PYZdl{\char`\$}
\def\PYZhy{\char`\-}
\def\PYZsq{\char`\'}
\def\PYZdq{\char`\"}
\def\PYZti{\char`\~}
% for compatibility with earlier versions
\def\PYZat{@}
\def\PYZlb{[}
\def\PYZrb{]}
\makeatother


    % For linebreaks inside Verbatim environment from package fancyvrb. 
    \makeatletter
        \newbox\Wrappedcontinuationbox 
        \newbox\Wrappedvisiblespacebox 
        \newcommand*\Wrappedvisiblespace {\textcolor{red}{\textvisiblespace}} 
        \newcommand*\Wrappedcontinuationsymbol {\textcolor{red}{\llap{\tiny$\m@th\hookrightarrow$}}} 
        \newcommand*\Wrappedcontinuationindent {3ex } 
        \newcommand*\Wrappedafterbreak {\kern\Wrappedcontinuationindent\copy\Wrappedcontinuationbox} 
        % Take advantage of the already applied Pygments mark-up to insert 
        % potential linebreaks for TeX processing. 
        %        {, <, #, %, $, ' and ": go to next line. 
        %        _, }, ^, &, >, - and ~: stay at end of broken line. 
        % Use of \textquotesingle for straight quote. 
        \newcommand*\Wrappedbreaksatspecials {% 
            \def\PYGZus{\discretionary{\char`\_}{\Wrappedafterbreak}{\char`\_}}% 
            \def\PYGZob{\discretionary{}{\Wrappedafterbreak\char`\{}{\char`\{}}% 
            \def\PYGZcb{\discretionary{\char`\}}{\Wrappedafterbreak}{\char`\}}}% 
            \def\PYGZca{\discretionary{\char`\^}{\Wrappedafterbreak}{\char`\^}}% 
            \def\PYGZam{\discretionary{\char`\&}{\Wrappedafterbreak}{\char`\&}}% 
            \def\PYGZlt{\discretionary{}{\Wrappedafterbreak\char`\<}{\char`\<}}% 
            \def\PYGZgt{\discretionary{\char`\>}{\Wrappedafterbreak}{\char`\>}}% 
            \def\PYGZsh{\discretionary{}{\Wrappedafterbreak\char`\#}{\char`\#}}% 
            \def\PYGZpc{\discretionary{}{\Wrappedafterbreak\char`\%}{\char`\%}}% 
            \def\PYGZdl{\discretionary{}{\Wrappedafterbreak\char`\$}{\char`\$}}% 
            \def\PYGZhy{\discretionary{\char`\-}{\Wrappedafterbreak}{\char`\-}}% 
            \def\PYGZsq{\discretionary{}{\Wrappedafterbreak\textquotesingle}{\textquotesingle}}% 
            \def\PYGZdq{\discretionary{}{\Wrappedafterbreak\char`\"}{\char`\"}}% 
            \def\PYGZti{\discretionary{\char`\~}{\Wrappedafterbreak}{\char`\~}}% 
        } 
        % Some characters . , ; ? ! / are not pygmentized. 
        % This macro makes them "active" and they will insert potential linebreaks 
        \newcommand*\Wrappedbreaksatpunct {% 
            \lccode`\~`\.\lowercase{\def~}{\discretionary{\hbox{\char`\.}}{\Wrappedafterbreak}{\hbox{\char`\.}}}% 
            \lccode`\~`\,\lowercase{\def~}{\discretionary{\hbox{\char`\,}}{\Wrappedafterbreak}{\hbox{\char`\,}}}% 
            \lccode`\~`\;\lowercase{\def~}{\discretionary{\hbox{\char`\;}}{\Wrappedafterbreak}{\hbox{\char`\;}}}% 
            \lccode`\~`\:\lowercase{\def~}{\discretionary{\hbox{\char`\:}}{\Wrappedafterbreak}{\hbox{\char`\:}}}% 
            \lccode`\~`\?\lowercase{\def~}{\discretionary{\hbox{\char`\?}}{\Wrappedafterbreak}{\hbox{\char`\?}}}% 
            \lccode`\~`\!\lowercase{\def~}{\discretionary{\hbox{\char`\!}}{\Wrappedafterbreak}{\hbox{\char`\!}}}% 
            \lccode`\~`\/\lowercase{\def~}{\discretionary{\hbox{\char`\/}}{\Wrappedafterbreak}{\hbox{\char`\/}}}% 
            \catcode`\.\active
            \catcode`\,\active 
            \catcode`\;\active
            \catcode`\:\active
            \catcode`\?\active
            \catcode`\!\active
            \catcode`\/\active 
            \lccode`\~`\~ 	
        }
    \makeatother

    \let\OriginalVerbatim=\Verbatim
    \makeatletter
    \renewcommand{\Verbatim}[1][1]{%
        %\parskip\z@skip
        \sbox\Wrappedcontinuationbox {\Wrappedcontinuationsymbol}%
        \sbox\Wrappedvisiblespacebox {\FV@SetupFont\Wrappedvisiblespace}%
        \def\FancyVerbFormatLine ##1{\hsize\linewidth
            \vtop{\raggedright\hyphenpenalty\z@\exhyphenpenalty\z@
                \doublehyphendemerits\z@\finalhyphendemerits\z@
                \strut ##1\strut}%
        }%
        % If the linebreak is at a space, the latter will be displayed as visible
        % space at end of first line, and a continuation symbol starts next line.
        % Stretch/shrink are however usually zero for typewriter font.
        \def\FV@Space {%
            \nobreak\hskip\z@ plus\fontdimen3\font minus\fontdimen4\font
            \discretionary{\copy\Wrappedvisiblespacebox}{\Wrappedafterbreak}
            {\kern\fontdimen2\font}%
        }%
        
        % Allow breaks at special characters using \PYG... macros.
        \Wrappedbreaksatspecials
        % Breaks at punctuation characters . , ; ? ! and / need catcode=\active 	
        \OriginalVerbatim[#1,codes*=\Wrappedbreaksatpunct]%
    }
    \makeatother

    % Exact colors from NB
    \definecolor{incolor}{HTML}{303F9F}
    \definecolor{outcolor}{HTML}{D84315}
    \definecolor{cellborder}{HTML}{CFCFCF}
    \definecolor{cellbackground}{HTML}{F7F7F7}
    
    % prompt
    \makeatletter
    \newcommand{\boxspacing}{\kern\kvtcb@left@rule\kern\kvtcb@boxsep}
    \makeatother
    \newcommand{\prompt}[4]{
        \ttfamily\llap{{\color{#2}[#3]:\hspace{3pt}#4}}\vspace{-\baselineskip}
    }
    

    
    % Prevent overflowing lines due to hard-to-break entities
    \sloppy 
    % Setup hyperref package
    \hypersetup{
      breaklinks=true,  % so long urls are correctly broken across lines
      colorlinks=true,
      urlcolor=urlcolor,
      linkcolor=linkcolor,
      citecolor=citecolor,
      }
    % Slightly bigger margins than the latex defaults
    
    \geometry{verbose,tmargin=1in,bmargin=1in,lmargin=1in,rmargin=1in}
    
    

\begin{document}
    
    \maketitle
    
    

    
    \tableofcontents 
\setcounter{section}{0}
\setcounter{page}{1}

    \hypertarget{introduction}{%
\section{Introduction}\label{introduction}}

\hypertarget{motivation}{%
\subsection{Motivation}\label{motivation}}

In the beginning of my academical career I was aiming an educational
degree i math and physics. While studying at the University of Tübingen
I was also worked as a research assistant in the Institute of Physical
Chemistry. In the course of my studies I was allowed to spend 4 month in
a school to gain first experiences in teaching physics and math to
students between 12 and 20 years. Even though it was a great experience
and I enjoyed it a lot to bring new ideas and concepts to the students,
I also felt a strong urge to continue learning and explore. At this time
the possibilities I saw in focusing on scientific research and
development seemed more attractive to me. My reasoning was, that I might
be able to combine my preference for teaching and research rather at the
university than a college. Long story short, after some years I ended up
doing my Phd. in the ``Institute of Theoretical and Physical Chemistry''
at the University of Tübingen. Lucky as I was I directly got the
possibility to work in a project with an industrial partner. The focus
was on building a new state of the art gas sensors. Besides the benefits
of working in a great teams, having enough financial support for
research activities and learning to present and report my results on a
regular basis, it also imposed some new problems needed to be solved.
One of the biggest challenges was the continuous increase of
experimental data in always shorter intervals. With the increasing
industrialization and automatization of sensor production and testing,
the quantity of high quality data increased dramatically.

Luck as I was, I was not the only one facing problems with the
increasing amount of data needed to be processed and analyzed. In the
beginning of each analysis, the specific task was not fully defined. It
was rather an exploratory research to gain an idea about possible hidden
opportunities. In this case the the traditional way of creating well
defined software algorithms for a specific tasks did not fit too well.

I rather had the need for efficient tools to reduce time consuming parts
regarding the data analysis, which had been performed manually until
now. I was looking for a platform to combine multiple of these tools
efficiently. So at this point of my career with my very basic knowledge
in the area of programming/coding and the need to solve urgent tasks, I
was looking for an efficient solutions with a steep learning curve.

Until then my standard procedure (and some others in our lab) of working
with data was mainly based on manual feature extraction and analysis.
When Working with a limited amount of samples, manually doing these
steps was acceptable and efficient. With the industrial cooperation
accelerated, the number of samples increased also rapidly. Soon we
reached the point where the pre-processing of the data would easily
consume a large amount of time. And that without even having started
with a deeper analysis. Luckily I was not the only one facing such
problems and a project from the Python community gained more and more
interest. Articles like \cite{Unpingco2014} pointed me to a new way of
dealing with data and indicated, that the Python programming language
might be very helpful. Regarding the fact, that Python was already a
well established programming language, the introduction of the ``IPtyhon
notebook ecosystem'' was making the use of Python for in an scientific
work flow very attractive. As mentioned in this article:
\cite{Osipov2016}

\begin{quote}
``\ldots{} what they do offer is an environment for exploration,
collaboration, and visualization.''
\end{quote}

I also realized the large potential for my working field. By learning
Python I got efficient tools for calculations, analysis and
representations. Additionally the new tools have been build specially
with the focus on the goal to easy report result including the way they
have been gained. The environment around the so called ``Juypter
notebooks'' was the ideal piece, which I experienced as a missing block
in the scientific work I was doing.

Besides my work for our industrial partner I also did fundamental
research about semiconducting metaloxide gas sensors. Based on great
research done before my time on gas sensors, my focus was now on
numerical calculations of the gas sensors. Typically a theoretical model
of a gas sensor is developed and the predictions are compared with
experimental results. When publishing the results the peer review system
assures, that the publications are well written, documented the on a
solid and proven basis. But when reading such papers, I had the
experience, many of them presented excellent ideas, but unfortunately
practical instructions on how to implement the presented models were
often not given.

The work of rebuilding the model and recalculating the results was
therefore often not possible in a reasonable amount of time. In my
experience this limits the direct comparison of new model with
experimental data from other sensors. It is also worth to mention, that
in my experience the average experimental oriented researcher does not
have the required programming knowledge to easily implement algorithm
based on the presented work. But I am confident, that if an algorithm
was given in an appropriate way, most research would benefit from the
presented code.

With the results I gained while my Phd. thesis, I was facing the same
problem. Others understood the value of my work but could not transfer
it to their particular problem. At this point my graphical
representation in the form of presentations, my detailed description and
the corresponding algorithms of my work had been three separated parts.
An appropriated way of representing those results in unified matter did
not exists yet. Facing this problem more and more, I decided not to
ignore it anymore. I made the choice to try to combine representation,
description and algorithm in one unified way.

I decided to orient myself back to my ``educational roots'' and use the
powerful ecosystem around the ``Jupyter notebooks'' to calculate,
represent and describe the results for my thesis for this task. This is
why my thesis is written with the large focus on suppling a introduction
to this excellent toolbox.

This work will (hopefully) allow more people to gain insides in the
sensing properties of semi conducting metal oxide sensors
(SMOX).Additionally this thesis is intended to be a useful introduction
into Python, specially IPython notebooks, for scientific work.

These hopes are not unfunded. In my time working at the ``Eberhard Karls
University of Tübingen'' i was also assigned to give multiple lectures
and classes. One lecture branch has been the introduction to data
mining. Often the lack of programming knowledge and the limited amount
of lecture time did not allow a usage of Python as a supporting
programming language. In these cases classical tools like Excel or
Origin have been used to analyze example datasets for hidden facts. Most
attendees have been very fast and understood the general concept of data
mining. Simply not being able to translate the general concept into
machine understandable instructions stopped them from using more
advanced tools.

In cases, where enough time was available, a short introduction into
programming with Python gained a lot of interest and was generally seen
as a positive experience. With just a short introduction already many
advanced tasks can be performed, which would often even have been
possible with the ``traditional'' tools.

This thesis is therefore structured in such a way, that I will present
my research results from the past years in a condensed form.
Additionally I will use this opportunity to introduce and explain the
importance of applying programming tools in the common work flow of
scientific work.

The potential of ``outsourcing'' repetitive tasks to machine executable
scripts is huge. By focusing on investing more time in creativity and
intelligence solutions, the overall quality of the results will
increase. My hope is to bring with this thesis not only a deeper insight
in the understanding of SMOX based gas sensors, but also help others to
start a interesting journey into the wide area of data mining and
machine learning with python.

The thesis is structured in three major ``work packages'', each of them
dealing with discrete elements of a typical scientific tasks. In a
condensed form the following areas will be covered:

\begin{enumerate}
\def\labelenumi{\arabic{enumi}.}
\tightlist
\item
  Solving differential equations and integral numerically
\item
  Using numerical models for simulation
\item
  Fitting numerical results with experimental data
\end{enumerate}

I typically finish my introductions to Python by letting the students
run their first commands. For other introductions found around the
globe, this is a program which outputs ``Hello World.''. For Python I
prefer to execute an other commands with boils almost down to the
philosophical essence on how ``instructions'' should be. In a ``Jupyter
notebook'', the next cell represents a ``code block'' and can be
executed along the text of the document. The output of the executed code
is then visible along with the text of the document.

    \begin{tcolorbox}[breakable, size=fbox, boxrule=1pt, pad at break*=1mm,colback=cellbackground, colframe=cellborder]
\prompt{In}{incolor}{1}{\boxspacing}
\begin{Verbatim}[commandchars=\\\{\}]
\PY{k+kn}{import} \PY{n+nn}{this}
\end{Verbatim}
\end{tcolorbox}

    \begin{Verbatim}[commandchars=\\\{\}]
The Zen of Python, by Tim Peters

Beautiful is better than ugly.
Explicit is better than implicit.
Simple is better than complex.
Complex is better than complicated.
Flat is better than nested.
Sparse is better than dense.
Readability counts.
Special cases aren't special enough to break the rules.
Although practicality beats purity.
Errors should never pass silently.
Unless explicitly silenced.
In the face of ambiguity, refuse the temptation to guess.
There should be one-- and preferably only one --obvious way to do it.
Although that way may not be obvious at first unless you're Dutch.
Now is better than never.
Although never is often better than *right* now.
If the implementation is hard to explain, it's a bad idea.
If the implementation is easy to explain, it may be a good idea.
Namespaces are one honking great idea -- let's do more of those!
    \end{Verbatim}

    \hypertarget{jupyter-notebooks-juypter}{%
\subsection[Jupyter Notebooks ]{\texorpdfstring{Jupyter Notebooks
\protect\includegraphics{media/icons/jupyter.png}}{Jupyter Notebooks Juypter}}\label{jupyter-notebooks-juypter}}

``The Zen of Python'' might not always be the primary directive of each
developer, but the Python community consists most probably of many
people how would consider the latter points as important. So did also
the inventors of the IPython and Jupyter. A quick search will reveal
multiple sources in the world wide web giving a detailed picture about
what ``IPython Notebooks'' are and how Jupyter in connected with them.
Here I will not try to give an general overview about this tool and
rather stick to the phrase ``Learning by doing.''. A minimal
understanding of working with Python will definitely be helpful for this
thesis, but in not necessarily required. I will try to explain the used
features this notebooks to guide an interested reader to the point
where:

\begin{itemize}
\tightlist
\item
  understanding the fundamental instructions of Python
\item
  using the basic functionality of the Notebooks
\item
  fundamental understanding of SMOX based gas sensors
\end{itemize}

is gained.

It is worth mentioning, that the intention of such notebooks is to merge
the essential tools of scientific work flows together. Data acquisition,
preparation, analysis, representation and documentation all available in
one place. The strength of not just sharing final conclusions in a
nicely formatted way, but also being able to share the full stack of
steps necessary to reach the final conclusion is essentially the
strength of the Jupyter notebooks. This feature is already changing the
way how scientific results are shared/published and was intentionally
designed this way \cite{Randles2017}.

The default format of representing anything in a notebook is based on
``Markdown''. Wikipedia summarizes Markdown like this:

\begin{quote}
Markdown is a lightweight markup language with plain text formatting
syntax. Wikipedia
\end{quote}

This means a document is formatted by writing plain text and special
text blocks are interpreted as formatting commands. E.g. \textbf{BOLD}
letters are generated by encapsulating the text with
\texttt{**\ Text\ here\ **}, headings are generated by starting the
heading with \texttt{\#}. Depending on the number of \texttt{\#},
subsections are created. I will not go into detail here about the
features of Markdown. Many features are used in this notebook and are
directly accessible by double-clicking the text element. By doing this,
the formatted text will switch back to the ``plain text representation''
and will reveal the way it was created. By again executing the cell with
\texttt{CTRL+ENTER}, its Markdown formatted representation is again
rendered.

One other handy feature of the Jupyter ecosystem I use is the ability to
transform notebooks into multiple other formats. Just to name some:
HTML, WORD (DOC, DOCX), Latex. The tool \texttt{nbconvert} is used
internally to convert the Markdown formatted representation into other
formats. For this thesis the default option ``Export as PDF'' under the
File option generates a Latex based PDF file.
\href{https://guides.github.com/features/mastering-markdown/}{Mastering-markdown}
is a web page, where I found some hints on how to format my notebook.
This example on web page demonstrates for instance how to generate block
quotes:

As Kanye West said:

\begin{quote}
We're living the future so the present is our past.
\end{quote}

To learn how to use notebooks it is best to use them in an interactive
environment. The next section will explain how to obtain one for free!

    \hypertarget{installation-guide}{%
\subsubsection{Installation guide}\label{installation-guide}}

The easiest way to get started would be to use the Anaconda
distribution. Anaconda bundles multiple different tools and installs
them in the operation system. Anaconda will take care of cross
dependencies and handle the update process of the software. This is not
the only way to get started with ``Jupyter Notebooks'' but surely an
very fast and easy one.
\href{media/presentations/InstallAnaconda/Anaconda-Python.pdf}{HERE} are
the slides I use for my lectures to guide students into the world of
Python and
\href{media/presentations/InstallAnaconda/PythonÜbung.zip}{here one
example} of it's usage.

    \hypertarget{example-notebook---sneak-preview}{%
\subsubsection{Example Notebook - Sneak
preview}\label{example-notebook---sneak-preview}}

Besides the first example \texttt{import\ this}, here is a very basic
example to demonstrate some features of the notebook.

    \hypertarget{simple-is-better-than-complex.} and are followed with an instruction. I will
demonstrate in this example the use of the \texttt{\%pylab\ inline}
instruction. This modifies the current programming space to become a lab
nicely equipped for scientific work tasks. For instance like a chemistry
lab is commonly equipped with a balance, a water tab and a fire
extinguisher, in this case a ``pylab'' is equipped with a data handing,
a plotting and a calculation tool (besides many others). The additional
parameter \texttt{inline} makes sure, that the figures will be along
with this document. So let's setup a ``pylab'' and run some lines of
code. (The plotting is handled in the background by Matplotlib
\cite{Hunter2007})

    \begin{tcolorbox}[breakable, size=fbox, boxrule=1pt, pad at break*=1mm,colback=cellbackground, colframe=cellborder]
\prompt{In}{incolor}{3}{\boxspacing}
\begin{Verbatim}[commandchars=\\\{\}]
\PY{o}{\PYZpc{}}\PY{k}{pylab} inline

\PY{c+c1}{\PYZsh{}get a list of 5000 points between \PYZhy{}50 and 50}
\PY{n}{xs} \PY{o}{=} \PY{n}{linspace}\PY{p}{(}\PY{n}{start}\PY{o}{=}\PY{o}{\PYZhy{}}\PY{l+m+mi}{50}\PY{p}{,} \PY{n}{stop}\PY{o}{=}\PY{l+m+mi}{50}\PY{p}{,} \PY{n}{num}\PY{o}{=}\PY{l+m+mi}{5000}\PY{p}{)}

\PY{c+c1}{\PYZsh{} Python way of adding a comment is the \PYZsq{}\PYZsh{}\PYZsq{} character}
\PY{c+c1}{\PYZsh{} Python way of writer \PYZsq{}x to the power of y\PYZsq{} is x**y}
\PY{c+c1}{\PYZsh{} here I am calculating thethe third power of x for 5000 points between \PYZhy{}50 and 50}
\PY{n}{ys} \PY{o}{=} \PY{n}{xs}\PY{o}{*}\PY{o}{*}\PY{l+m+mi}{3}

\PY{c+c1}{\PYZsh{}plot it}
\PY{n}{fig} \PY{o}{=} \PY{n}{plt}\PY{o}{.}\PY{n}{plot}\PY{p}{(}\PY{n}{xs}\PY{p}{,}\PY{n}{ys}\PY{p}{,}\PY{l+s+s1}{\PYZsq{}}\PY{l+s+s1}{b}\PY{l+s+s1}{\PYZsq{}}\PY{p}{)}
\end{Verbatim}
\end{tcolorbox}

    \begin{Verbatim}[commandchars=\\\{\}]
Populating the interactive namespace from numpy and matplotlib
    \end{Verbatim}

    \begin{center}
    \adjustimage{max size={0.9\linewidth}{0.9\paperheight}}{1_Introduction_files/1_Introduction_7_1.png}
    \end{center}
    { \hspace*{\fill} \\}
    
    The \texttt{linspace} function generates a 5000 linear distributed
points in the interval {[}-50,50) and saves those points in the xs
variable. ys is a list holding the third power of each point in xs.
\texttt{plt.} is a submodule defined by the magic command
\texttt{\%pylab} which handels data plotting in a very simple way.
\texttt{plt.plot(x,y,\textquotesingle{}r-\textquotesingle{})} for
example plots x vs.~y with a red line. The thesis is done in the
notebook to offer the reader the possibility to directly work with newly
gained knowledge. Therefore the upper block is a good opportunity to do
the fist steps in Python. For instance change the line format to `b'
(blue). Or to `b-.'. Change the exponent from 3 to 3.1. To do so:

\begin{itemize}
\tightlist
\item
  click on the code box above
\item
  edit the code
\item
  Add the following line
  \texttt{plt.title(\textquotesingle{}The\ power\ law\textquotesingle{})}
\item
  go back with CTRL-Z to correct your mistakes
\item
  hit CTRL-ENTER to execute the code
\end{itemize}

    \hypertarget{smox-based-gas-sensors}{%
\subsection{SMOX based gas sensors}\label{smox-based-gas-sensors}}

In this section a very short summary about \textbf{S}emi conducting
\textbf{M}etal \textbf{OX}ide (SMOX) based gas sensors is provided.
Relevant aspects for this thesis will be presented in a general way and
additional literature references are be provided. The intention is to
provided the minimum amount of information needed to follow the this
work. The supplied references are a good staring point to gain a
detailed understanding of SMOX based gas sensors.

\hypertarget{smox-material}{%
\subsubsection{SMOX material}\label{smox-material}}

A typical definition of a sensors describes two elements. The receptor
and the transducer. The receptor interacts with the external stimulus,
while the transducer translates this stimulus into a measurable
response. In the case of SMOX sensors those two ``parts'' can't be
separated directly, since the SMOX material takes over the role of both.
Generally the gas sensing with SMOX materials is based on the
interaction of target molecules at its surface and a resulting change of
the semiconductor itself \cite{Barsan2003}.

Nevertheless the chemistry at the surface, like the
adsorption/desorption of molecules, may be seen as the receptor of the
sensor. In the process of adsorption/desorption at the surfaces,
typically charges from the inside of the material are involved. In
general this results in a modification of the charge distribution inside
the semiconductor. As a consequence of an increase/decrease of the free
charge carrier concentration the overall resistance of the material
changes. The processes related to changes inside the semiconductor and
hence the resistance, could be associated to the ``transducer'' part of
the sensor. A detailed description of the physical and chemical
properties that make tin oxide a suitable material gas sensing is
described here:"\cite{Batzill2005}.

For most SMOX the intrinsic electrical conductivity is very low due to a
high band gab of the semiconductor. Typical preparation methods of the
SMOX material generate defects inside the crystal structure due to
missing oxygen atoms. For \(SnO_2\), these defects act as additional
electron donors and increase the number of free charge carriers and
hence increase the conductivity.

Typically density of additional donors \(N_D\) of acceptors \(N_A\)
plays a significant role in the sensing properties. The availability of
additional free charge carries resulting from such defects depend also
on the overall temperature of the sensor. Typical operation temperatures
are around 300°C, but also higher and lower operations temperatures may
be suitable depending on the use case. Besides effecting the
semiconductors properties, the reactions at the surface of the sensor
are also strongly dependent on the temperature.

\hypertarget{smox-based-thick-film-sensors}{%
\subsubsection{SMOX based thick film
sensors}\label{smox-based-thick-film-sensors}}

The model of this work focuses on describing thick film SMOX gas
sensors. In the case of such sensors the SMOX material is deposited as
an porous thick film on top of a electric structure suitable to measure
the resistance of the film. The film itself consists of multiple
nano-sized grains in direct contact with each other. To measure the
resistance of the film, a bias voltage is applied and the resulting
current which passes through the network of multiple grains is measured.
Depending on the percolation path, the total resistance of the sensor
\(R_{Sensor}\) can be approximated as the product of a geometrical
factor \(G\) associated to the film morphology and the resistance of one
grain \(R_{Grain}\) as:

\begin{align}
R_{Sensor}  = G * R_{Grain} \label{r_sensor_r_grain}\tag{Resistance of sensor}
\end{align} The following sketch represents the typical setup of a thick
film sensor: \includegraphics{media/grains_sensor.png}

\hypertarget{surface-potential}{%
\subsubsection{Surface potential}\label{surface-potential}}

To better understand the processes related to the changes in the free
charge carrier concentration inside the semiconductor, the semiconductor
is typically described in a energy bands representation. For instance
for \(SnO_2\), trapping electrons at the surface results in an upward
bending of the conduction band. The position of the conduction band at
the surface is typically expresses in units \(E_S = -eV_S\) (\(e\) is
the charge of an electron). \(V_S\) is called the surface potential. In
regard to a non-bended conduction band, the number of electrons are
reduced in regions, where the band in elevated.

Experimentally the Kelvin probe method can be used to measure the change
in the surface potential \(\Delta V_S\). The experimental setup and the
requirements to obtain the value of \(\Delta V_S\) are described in
detail here: \cite{Oprea2009a}

The relation between \(\Delta V_S\) and the resulting change in the
resistance of the semiconductor \(R\) is then only determined by the
semiconductor. The relation between \(\Delta V_S\) and the resistance
\(R\) of the semiconductor is defined independently of the actual
surface chemistry. The relation depends strongly on the temperature
\(T\), the defect concentration \(N_D\) and the shape of the underlaying
material. This work focuses on the deriving a model to predict the
theoretical value of \(\Delta R_{V_S} = \frac {R_{V_S}} {R_{V_S=0}}\)
from a change in the surface potential \(\Delta V_S\).

With the definition of (\ref{r_sensor_r_grain}) the value of
\(\Delta R_{V_S}\) of one grain can be directly calculated by the ratio
of the experimentally measured value of the sensors \(R_{Sensor}\).
\begin{align}
\frac{R_{Sensor_{V_S}}}{R_{Sensor_{V_S=0}}}  = \frac{G*R_{Grain_{V_S}}}{G*R_{Grain_{V_S=0}}} = \Delta R_{V_S}
\end{align}

\hypertarget{numerical-model}{%
\subsubsection{Numerical model}\label{numerical-model}}

Based on the example of \(SnO_2\) as SMOX material, a numerical model
describing effects of shape, size, defect concentration and temperature
is developed for spherical grains. With such a model multiple
conclusions about the dependency of these parameters on the sensitivity
and performance of a sensor can be derived. Additionally the numerical
model is able to provide the number of charges involved in the sensing
process. This number is a important parameter involved in the
description of the chemical reaction at the surface.

When solving this problem numerically, some assumptions will need to be
introduction. These assumptions simplify the problem to a level where a
numerical calculation is possible, but will reduce the validity of the
results only to a small subset of all possible situations. The
calculations in this thesis are also packed with assumptions and
boundary conditions. My intention is to supply enough information to
understand the relevance of the assumption and it's implications along
with the supplied code. The way of presenting this knowledge should
lead/motivate others to adapt the presented work to individual other
cases with different boundary conditions.

    \hypertarget{conclusion}{%
\section{Conclusion}\label{conclusion}}

Since the motivation for this ``interactive'' thesis should be clear
now, I would like to come in the next section now to my actual research
topic: ``Numerical calculation of semiconducting metal oxide (SMOX)
based gas sensors''. In the \href{2-Grain-SMOX.ipynb}{next chapter} I
will demonstrate how theoretical numerical calculations are used to for
chemical sensors are used to better understand experimental results. In
the \href{3-Applying_theorie-to-practice.ipynb}{following chapter} I
will demonstrate how such knowledge could be used to improve the
performance of a sensor regarding it's sensitivity and selectivity.

\href{2-Grain-SMOX.ipynb}{Follow this link to come to the next section.}

    \hypertarget{about-the-pdf-version-of-this-work}{%
\section{About the PDF-Version of this
work}\label{about-the-pdf-version-of-this-work}}

This notebook was not intended to be used as a printed hard copy or as a
PDF. Its main intention is to serve as an easy entry point to Python
supported science. Many of the implemented features in these notebooks
like interactive widgets, animated data representations and live code
examples will not work in the PDF-version. Only a static snapshot of the
mutable representation can be represented outside the notebook, at best.
The benefits of interacting with the presented work in a notebook should
motivate the reader to use the notebook.

Nevertheless the integrated function to export a notebook to a latex
based version offers a very nice way to publish results in a printable
way. So please keep in mind, that the PDF-verison may not be able to
represent all the features of the notebook as intended and some links
might not work as expected. You are strongly encouraged to switch to the
Jupyter presentation of this work and experience the full potential of
such notebooks.

\hypertarget{equations}{%
\subsection{Equations}\label{equations}}

In a typical scientific thesis and textbook, relevant equations are
referred by numbers of identifiers. In Latex this is done, by assigning
a label and a tag to an equation. If this equation needs to be referred
to, a pointer to the reference is added, and the tag is used for the
representation of this equation. As an example, an arbitrary equation
from this thesis is used. The (internal) reference of this equation is
`example', while the printed representation (tag) is ``Example
Equation''.

\begin{align}
\frac{d²V^{*}}{dr^{*2}}=1-n^{*}(V^{*})-\frac{2}{r^{*}}\frac{dV^{*}}{dr^{*}}\label{example}\tag{Example Equation}
\end{align}

To refer to this equation a reference can be added which will look like
this (\ref{example}). The underlying mechanisms will link the reference
with the equation, print the label, and add an hyperlink allowing to
directly jump to the equation. This feature works either in notebook or
PDF representations. One drawback of the notebook representation is,
that the equation and reference need to be in the same code block/cell.
Otherwise the reference is not working. Instead of linking the equation
correctly, ??? will be represented. My opinion is, that this will change
with future versions of Jupyter. Since the PDF version is processed by a
Latex interpreter in total, the ``one code block'' limitation does not
exist there. To demonstrate this, I will try to reference the equation
in the next code block.

    Here the same reference to the equation: \ref{second_derivative}. In the
PDF-Version this will work correctly and in the Notebook-Version only
??? should be visible (until now).

    \hypertarget{tables}{%
\subsection{Tables}\label{tables}}

With the numerical calculations performed in this thesis, data will need
to be represented also. One way of doing this is by plotting the the
data. This is also suitable for a static PDF-Version of this thesis,
which often is useful/necessary to have. This feature is well
implemented and the transfer from Notebook to PDF version works well.
Another way to display data are tables. Similar to the well known,
omnipresent tool for working with tables, Excel, the Python universe has
its own, but similar, tool. It's called Pandas. As a primer I will give
here a very short introduction to Pandas. What sheets are for Excel, are
Dataframes for Pandas. Here a simple example how to create a Dataframe
in analogy to the previous programming example:

    \begin{tcolorbox}[breakable, size=fbox, boxrule=1pt, pad at break*=1mm,colback=cellbackground, colframe=cellborder]
\prompt{In}{incolor}{6}{\boxspacing}
\begin{Verbatim}[commandchars=\\\{\}]
\PY{o}{\PYZpc{}}\PY{k}{pylab}
\PY{k+kn}{import} \PY{n+nn}{pandas}

\PY{c+c1}{\PYZsh{}create some x values}
\PY{n}{x} \PY{o}{=} \PY{n}{linspace}\PY{p}{(}\PY{n}{start}\PY{o}{=}\PY{l+m+mi}{0}\PY{p}{,}\PY{n}{stop}\PY{o}{=}\PY{l+m+mi}{5}\PY{p}{,}\PY{n}{num} \PY{o}{=} \PY{l+m+mi}{10}\PY{p}{)}

\PY{c+c1}{\PYZsh{}the y values will the the square of the x values}
\PY{n}{y} \PY{o}{=} \PY{n}{x}\PY{o}{*}\PY{o}{*}\PY{l+m+mi}{2}

\PY{c+c1}{\PYZsh{}Put them in a Dataframe and reference this new Dataframe with the variable `dF`}
\PY{n}{dF} \PY{o}{=} \PY{n}{pandas}\PY{o}{.}\PY{n}{DataFrame}\PY{p}{(}\PY{p}{\PYZob{}}\PY{l+s+s1}{\PYZsq{}}\PY{l+s+s1}{x}\PY{l+s+s1}{\PYZsq{}}\PY{p}{:}\PY{n}{x}\PY{p}{,} \PY{l+s+s1}{\PYZsq{}}\PY{l+s+s1}{y}\PY{l+s+s1}{\PYZsq{}}\PY{p}{:}\PY{n}{y}\PY{p}{\PYZcb{}}\PY{p}{)}
\end{Verbatim}
\end{tcolorbox}

    \begin{Verbatim}[commandchars=\\\{\}]
Using matplotlib backend: Qt5Agg
Populating the interactive namespace from numpy and matplotlib
    \end{Verbatim}

    This simple example of a Dataframe should demonstrate the basic concept
of Dataframes. Once data is in the \texttt{DataFrame} format, the are
infinit ways to transform/slice/merge/\ldots{} it, to bring it into the
desired shape. Along this thesis, some of the functionalities of
Dataframes will be used and explained. Since the Jupyter environment is
still ``under construction'', the transfer from Juypter to PDF for
notebooks does not work to well until now. This does not mean, that is
is not possible, it is just not yet implemented in the default/vanilla
environment. This is way I want to highlight the small tweak that is at
this point still needed to have a comparable output in Juypter notebooks
and printed PDFs. To bypass this obstacle the default behavior of pandas
needs to be altered ( similar to this post:
\href{https://stackoverflow.com/a/24167756}{Latex-Tables Monkey Patch}).
This following patch brings DataFrames in the appropriate shape to by
nicely represented in both representations.

    \hypertarget{before-the-patch}{%
\subsubsection{Before the patch}\label{before-the-patch}}

    \begin{tcolorbox}[breakable, size=fbox, boxrule=1pt, pad at break*=1mm,colback=cellbackground, colframe=cellborder]
\prompt{In}{incolor}{7}{\boxspacing}
\begin{Verbatim}[commandchars=\\\{\}]
\PY{n}{display}\PY{p}{(}\PY{n}{dF}\PY{p}{)}
\end{Verbatim}
\end{tcolorbox}

    
    \begin{verbatim}
          x          y
0  0.000000   0.000000
1  0.555556   0.308642
2  1.111111   1.234568
3  1.666667   2.777778
4  2.222222   4.938272
5  2.777778   7.716049
6  3.333333  11.111111
7  3.888889  15.123457
8  4.444444  19.753086
9  5.000000  25.000000
    \end{verbatim}

    
    \hypertarget{the-patch}{%
\subsubsection{The patch}\label{the-patch}}

    \begin{tcolorbox}[breakable, size=fbox, boxrule=1pt, pad at break*=1mm,colback=cellbackground, colframe=cellborder]
\prompt{In}{incolor}{8}{\boxspacing}
\begin{Verbatim}[commandchars=\\\{\}]
\PY{k+kn}{import} \PY{n+nn}{pandas}
\PY{n}{pandas}\PY{o}{.}\PY{n}{set\PYZus{}option}\PY{p}{(}\PY{l+s+s1}{\PYZsq{}}\PY{l+s+s1}{display.notebook\PYZus{}repr\PYZus{}html}\PY{l+s+s1}{\PYZsq{}}\PY{p}{,} \PY{k+kc}{True}\PY{p}{)}

\PY{k}{def} \PY{n+nf}{\PYZus{}repr\PYZus{}latex\PYZus{}}\PY{p}{(}\PY{n+nb+bp}{self}\PY{p}{)}\PY{p}{:}
    \PY{k}{return} \PY{l+s+sa}{r}\PY{l+s+s2}{\PYZdq{}\PYZdq{}\PYZdq{}}
\PY{l+s+s2}{    }\PY{l+s+s2}{\PYZbs{}}\PY{l+s+s2}{begin}\PY{l+s+si}{\PYZob{}center\PYZcb{}}
\PY{l+s+s2}{    }\PY{l+s+s2}{\PYZob{}}\PY{l+s+si}{\PYZpc{}s}\PY{l+s+s2}{\PYZcb{}}
\PY{l+s+s2}{    }\PY{l+s+s2}{\PYZbs{}}\PY{l+s+s2}{end}\PY{l+s+si}{\PYZob{}center\PYZcb{}}
\PY{l+s+s2}{    }\PY{l+s+s2}{\PYZdq{}\PYZdq{}\PYZdq{}} \PY{o}{\PYZpc{}} \PY{n+nb+bp}{self}\PY{o}{.}\PY{n}{to\PYZus{}latex}\PY{p}{(}\PY{p}{)}

\PY{n}{pandas}\PY{o}{.}\PY{n}{DataFrame}\PY{o}{.}\PY{n}{\PYZus{}repr\PYZus{}latex\PYZus{}} \PY{o}{=} \PY{n}{\PYZus{}repr\PYZus{}latex\PYZus{}}  \PY{c+c1}{\PYZsh{} monkey patch pandas DataFrame}
\end{Verbatim}
\end{tcolorbox}

    \hypertarget{prettier-tables-after-the-patch}{%
\subsubsection{Prettier tables after the
patch}\label{prettier-tables-after-the-patch}}

    \begin{tcolorbox}[breakable, size=fbox, boxrule=1pt, pad at break*=1mm,colback=cellbackground, colframe=cellborder]
\prompt{In}{incolor}{9}{\boxspacing}
\begin{Verbatim}[commandchars=\\\{\}]
\PY{n}{display}\PY{p}{(}\PY{n}{dF}\PY{p}{)}
\PY{c+c1}{\PYZsh{}\PYZdq{}Test Tabel\PYZdq{}}
\end{Verbatim}
\end{tcolorbox}

    
    \begin{center}
    {\begin{tabular}{lrr}
\toprule
{} &         x &          y \\
\midrule
0 &  0.000000 &   0.000000 \\
1 &  0.555556 &   0.308642 \\
2 &  1.111111 &   1.234568 \\
3 &  1.666667 &   2.777778 \\
4 &  2.222222 &   4.938272 \\
5 &  2.777778 &   7.716049 \\
6 &  3.333333 &  11.111111 \\
7 &  3.888889 &  15.123457 \\
8 &  4.444444 &  19.753086 \\
9 &  5.000000 &  25.000000 \\
\bottomrule
\end{tabular}
}
    \end{center}
    

    
    In Jupyter notebooks the output will look very similar. In the PDF
version of this thesis, those two output will differ(, unless a newer
version of Jupyter fixed this issue.) To have this thesis in a PDF
printable form, I will use the presented patch to unify the output. On
the other side I suggest not to use this patch, and rather work with the
Jupyter notebooks and its default representation, since the patch has
also its downsides. But this would go beyond the scope of this remark
about the patch.

    \hypertarget{bibliography-section}{%
\section{Bibliography section}\label{bibliography-section}}

    \bibliographystyle{alphadin}
\bibliography{ipython}

    \newpage

This page is intentionally left blank.


    % Add a bibliography block to the postdoc
    
    
    
\end{document}
